%%%%%%%%%%%%%%%%%%%%%%%%%%%%%%%%%%%%%%%%%%%%%%%%%%%%%%%%%%%%%%%
%% OXFORD THESIS TEMPLATE

% Use this template to produce a standard thesis that meets the Oxford University requirements for DPhil submission
%
% Originally by Keith A. Gillow (gillow@maths.ox.ac.uk), 1997
% Modified by Sam Evans (sam@samuelevansresearch.org), 2007
% Modified by John McManigle (john@oxfordechoes.com), 2015
% Modified by Ulrik Lyngs (ulrik.lyngs@cs.ox.ac.uk), 2018, for use with R Markdown
%
% Ulrik Lyngs, 25 Nov 2018: Following John McManigle, broad permissions are granted to use, modify, and distribute this software
% as specified in the MIT License included in this distribution's LICENSE file.
%
% John tried to comment this file extensively, so read through it to see how to use the various options.  Remember
% that in LaTeX, any line starting with a % is NOT executed.  Several places below, you have a choice of which line to use
% out of multiple options (eg draft vs final, for PDF vs for binding, etc.)  When you pick one, add a % to the beginning of
% the lines you don't want.


%%%%% CHOOSE PAGE LAYOUT
% The most common choices should be below.  You can also do other things, like replacing "a4paper" with "letterpaper", etc.

% This one will format for two-sided binding (ie left and right pages have mirror margins; blank pages inserted where needed):
%\documentclass[a4paper,twoside]{templates/ociamthesis}
% This one will format for one-sided binding (ie left margin > right margin; no extra blank pages):
%\documentclass[a4paper]{ociamthesis}
% This one will format for PDF output (ie equal margins, no extra blank pages):
%\documentclass[a4paper,nobind]{templates/ociamthesis}
%UL 2 Dec 2018: pass this in from YAML
\documentclass[a4paper, nobind]{templates/ociamthesis}

% UL 5 January 2021 - add packages used by kableExtra
\usepackage{booktabs}
\usepackage{longtable}
\usepackage{array}
\usepackage{multirow}
\usepackage{wrapfig}
\usepackage{colortbl}
\usepackage{pdflscape}
\usepackage{tabu}
\usepackage{threeparttable}
\usepackage{threeparttablex}
\usepackage[normalem]{ulem}
\usepackage{makecell}
\usepackage[colorlinks=false,pdfpagelabels,hidelinks=]{hyperref}
\usepackage{float}


%UL set section header spacing
\usepackage{titlesec}
% 
\titlespacing\subsubsection{0pt}{24pt plus 4pt minus 2pt}{0pt plus 2pt minus 2pt}

% UL 30 Nov 2018 pandoc puts lists in 'tightlist' command when no space between bullet points in Rmd file
\providecommand{\tightlist}{%
  \setlength{\itemsep}{0pt}\setlength{\parskip}{0pt}}
 
% UL 1 Dec 2018, fix to include code in shaded environments
\usepackage{color}
\usepackage{fancyvrb}
\newcommand{\VerbBar}{|}
\newcommand{\VERB}{\Verb[commandchars=\\\{\}]}
\DefineVerbatimEnvironment{Highlighting}{Verbatim}{commandchars=\\\{\}}
% Add ',fontsize=\small' for more characters per line
\usepackage{framed}
\definecolor{shadecolor}{RGB}{248,248,248}
\newenvironment{Shaded}{\begin{snugshade}}{\end{snugshade}}
\newcommand{\AlertTok}[1]{\textcolor[rgb]{0.94,0.16,0.16}{#1}}
\newcommand{\AnnotationTok}[1]{\textcolor[rgb]{0.56,0.35,0.01}{\textbf{\textit{#1}}}}
\newcommand{\AttributeTok}[1]{\textcolor[rgb]{0.77,0.63,0.00}{#1}}
\newcommand{\BaseNTok}[1]{\textcolor[rgb]{0.00,0.00,0.81}{#1}}
\newcommand{\BuiltInTok}[1]{#1}
\newcommand{\CharTok}[1]{\textcolor[rgb]{0.31,0.60,0.02}{#1}}
\newcommand{\CommentTok}[1]{\textcolor[rgb]{0.56,0.35,0.01}{\textit{#1}}}
\newcommand{\CommentVarTok}[1]{\textcolor[rgb]{0.56,0.35,0.01}{\textbf{\textit{#1}}}}
\newcommand{\ConstantTok}[1]{\textcolor[rgb]{0.00,0.00,0.00}{#1}}
\newcommand{\ControlFlowTok}[1]{\textcolor[rgb]{0.13,0.29,0.53}{\textbf{#1}}}
\newcommand{\DataTypeTok}[1]{\textcolor[rgb]{0.13,0.29,0.53}{#1}}
\newcommand{\DecValTok}[1]{\textcolor[rgb]{0.00,0.00,0.81}{#1}}
\newcommand{\DocumentationTok}[1]{\textcolor[rgb]{0.56,0.35,0.01}{\textbf{\textit{#1}}}}
\newcommand{\ErrorTok}[1]{\textcolor[rgb]{0.64,0.00,0.00}{\textbf{#1}}}
\newcommand{\ExtensionTok}[1]{#1}
\newcommand{\FloatTok}[1]{\textcolor[rgb]{0.00,0.00,0.81}{#1}}
\newcommand{\FunctionTok}[1]{\textcolor[rgb]{0.00,0.00,0.00}{#1}}
\newcommand{\ImportTok}[1]{#1}
\newcommand{\InformationTok}[1]{\textcolor[rgb]{0.56,0.35,0.01}{\textbf{\textit{#1}}}}
\newcommand{\KeywordTok}[1]{\textcolor[rgb]{0.13,0.29,0.53}{\textbf{#1}}}
\newcommand{\NormalTok}[1]{#1}
\newcommand{\OperatorTok}[1]{\textcolor[rgb]{0.81,0.36,0.00}{\textbf{#1}}}
\newcommand{\OtherTok}[1]{\textcolor[rgb]{0.56,0.35,0.01}{#1}}
\newcommand{\PreprocessorTok}[1]{\textcolor[rgb]{0.56,0.35,0.01}{\textit{#1}}}
\newcommand{\RegionMarkerTok}[1]{#1}
\newcommand{\SpecialCharTok}[1]{\textcolor[rgb]{0.00,0.00,0.00}{#1}}
\newcommand{\SpecialStringTok}[1]{\textcolor[rgb]{0.31,0.60,0.02}{#1}}
\newcommand{\StringTok}[1]{\textcolor[rgb]{0.31,0.60,0.02}{#1}}
\newcommand{\VariableTok}[1]{\textcolor[rgb]{0.00,0.00,0.00}{#1}}
\newcommand{\VerbatimStringTok}[1]{\textcolor[rgb]{0.31,0.60,0.02}{#1}}
\newcommand{\WarningTok}[1]{\textcolor[rgb]{0.56,0.35,0.01}{\textbf{\textit{#1}}}}

%UL set white space before and after code blocks
\renewenvironment{Shaded}
{
  \vspace{10pt}%
  \begin{snugshade}%
}{%
  \end{snugshade}%
  \vspace{8pt}%
}

%%%CSL
%%%%

%UL set whitespace around verbatim environments
\usepackage{etoolbox}
\makeatletter
\preto{\@verbatim}{\topsep=0pt \partopsep=0pt }
\makeatother

%UL 26 Mar 2019, enable strikethrough
\usepackage[normalem]{ulem}

%UL use soul package for correction highlighting
\usepackage{color, soul}
\usepackage{xcolor}
\definecolor{correctioncolor}{HTML}{CCCCFF}
\sethlcolor{correctioncolor}
\newcommand{\ctext}[3][RGB]{%
  \begingroup
  \definecolor{hlcolor}{#1}{#2}\sethlcolor{hlcolor}%
  \hl{#3}%
  \endgroup
}
\soulregister\ref7
\soulregister\cite7
\soulregister\autocite7
\soulregister\textcite7
\soulregister\pageref7

%%%%%%% PAGE HEADERS AND FOOTERS %%%%%%%%%
\usepackage{fancyhdr}
\setlength{\headheight}{15pt}
\fancyhf{} % clear the header and footers
\pagestyle{fancy}
\renewcommand{\chaptermark}[1]{\markboth{\thechapter. #1}{\thechapter. #1}}
\renewcommand{\sectionmark}[1]{\markright{\thesection. #1}} 
\renewcommand{\headrulewidth}{0pt}

\fancyhead[LO]{\emph{\leftmark}} 
\fancyhead[RE]{\emph{\rightmark}} 

% UL page number position 
\fancyfoot[C]{\emph{\thepage}} %regular pages
\fancypagestyle{plain}{\fancyhf{}\fancyfoot[C]{\emph{\thepage}}} %chapter pages

% JEM fix header on cleared pages for openright
\def\cleardoublepage{\clearpage\if@twoside \ifodd\c@page\else
   \hbox{}
   \fancyfoot[C]{}
   \newpage
   \if@twocolumn\hbox{}\newpage
   \fi
   \fancyhead[LO]{\emph{\leftmark}} 
   \fancyhead[RE]{\emph{\rightmark}} 
   \fi\fi}


%%%%% SELECT YOUR DRAFT OPTIONS
% This adds a "DRAFT" footer to every normal page.  (The first page of each chapter is not a "normal" page.)

% This highlights (in blue) corrections marked with (for words) \mccorrect{blah} or (for whole
% paragraphs) \begin{mccorrection} . . . \end{mccorrection}.  This can be useful for sending a PDF of
% your corrected thesis to your examiners for review.  Turn it off, and the blue disappears.

% IP feb 2021: option to include line numbers in PDF

%%%%% BIBLIOGRAPHY SETUP
% Note that your bibliography will require some tweaking depending on your department, preferred format, etc.
% If you've not used LaTeX before, I recommend reading a little about biblatex/biber and getting started with it.
% If you're already a LaTeX pro and are used to natbib or something, modify as necessary.
% Either way, you'll have to choose and configure an appropriate bibliography format...


\usepackage[style=authoryear, sorting=nyt, backend=biber, maxcitenames=2, useprefix, doi=true, isbn=false, uniquename=false]{biblatex}
\newcommand*{\bibtitle}{References}

\addbibresource{references.bib}


% This makes the bibliography left-aligned (not 'justified') and slightly smaller font.
\renewcommand*{\bibfont}{\raggedright\small}


% Uncomment this if you want equation numbers per section (2.3.12), instead of per chapter (2.18):
%\numberwithin{equation}{subsection}


%%%%% THESIS / TITLE PAGE INFORMATION
% Everybody needs to complete the following:
\title{Empirical Evaluation of the Possible\\
Impacts of the Transformation of\\
Microfinance Institutions in Africa}
\author{John King'athia Karuitha}
\college{Graduate School of Business Administration}

% Master's candidates who require the alternate title page (with candidate number and word count)
% must also un-comment and complete the following three lines:

% Uncomment the following line if your degree also includes exams (eg most masters):
%\renewcommand{\submittedtext}{Submitted in partial completion of the}
% Your full degree name.  (But remember that DPhils aren't "in" anything.  They're just DPhils.)
\degree{Doctor of Philosophy in Finance}
% Term and year of submission, or date if your board requires (eg most masters)
\degreedate{August 2021}


%%%%% YOUR OWN PERSONAL MACROS
% This is a good place to dump your own LaTeX macros as they come up.

% To make text superscripts shortcuts
	\renewcommand{\th}{\textsuperscript{th}} % ex: I won 4\th place
	\newcommand{\nd}{\textsuperscript{nd}}
	\renewcommand{\st}{\textsuperscript{st}}
	\newcommand{\rd}{\textsuperscript{rd}}

%%%%% THE ACTUAL DOCUMENT STARTS HERE
\begin{document}

%%%%% CHOOSE YOUR LINE SPACING HERE
% This is the official option.  Use it for your submission copy and library copy:
\setlength{\textbaselineskip}{22pt plus2pt}
% This is closer spacing (about 1.5-spaced) that you might prefer for your personal copies:
%\setlength{\textbaselineskip}{18pt plus2pt minus1pt}

% You can set the spacing here for the roman-numbered pages (acknowledgements, table of contents, etc.)
\setlength{\frontmatterbaselineskip}{17pt plus1pt minus1pt}

% UL: You can set the line and paragraph spacing here for the separate abstract page to be handed in to Examination schools
\setlength{\abstractseparatelineskip}{13pt plus1pt minus1pt}
\setlength{\abstractseparateparskip}{0pt plus 1pt}

% UL: You can set the general paragraph spacing here - I've set it to 2pt (was 0) so
% it's less claustrophobic
\setlength{\parskip}{2pt plus 1pt}

%
% Oxford University logo on title page
%
\def\crest{{\includegraphics{templates/wits.jpg}}}
\renewcommand{\university}{University of the Witwatersrand, Johannesburg}
\renewcommand{\submittedtext}{A thesis submitted for the degree of}


% Leave this line alone; it gets things started for the real document.
\setlength{\baselineskip}{\textbaselineskip}


%%%%% CHOOSE YOUR SECTION NUMBERING DEPTH HERE
% You have two choices.  First, how far down are sections numbered?  (Below that, they're named but
% don't get numbers.)  Second, what level of section appears in the table of contents?  These don't have
% to match: you can have numbered sections that don't show up in the ToC, or unnumbered sections that
% do.  Throughout, 0 = chapter; 1 = section; 2 = subsection; 3 = subsubsection, 4 = paragraph...

% The level that gets a number:
\setcounter{secnumdepth}{3}
% The level that shows up in the ToC:
\setcounter{tocdepth}{3}


%%%%% ABSTRACT SEPARATE
% This is used to create the separate, one-page abstract that you are required to hand into the Exam
% Schools.  You can comment it out to generate a PDF for printing or whatnot.

% JEM: Pages are roman numbered from here, though page numbers are invisible until ToC.  This is in
% keeping with most typesetting conventions.
\begin{romanpages}

% Title page is created here
\maketitle

%%%%% DEDICATION -- If you'd like one, un-comment the following.
\begin{dedication}
  For my parents Justus King'athia and Veronicah Wanjiku; My daughter Veronicah Wanjiku; My siblings Margaret Njeri, Martha Wangari, Ann Nyaguthii, Caroline Wothaya, and Thomas Kimondo.
\end{dedication}

%%%%% ACKNOWLEDGEMENTS -- Nothing to do here except comment out if you don't want it.
\begin{acknowledgements}
 	This is where you will normally thank your advisor, colleagues, family and friends, as well as funding and institutional support. In our case, we will give our praises to the people who developed the ideas and tools that allow us to push open science a little step forward by writing plain-text, transparent, and reproducible theses in R Markdown.

  We must be grateful to John Gruber for inventing the original version of Markdown, to John MacFarlane for creating Pandoc (\url{http://pandoc.org}) which converts Markdown to a large number of output formats, and to Yihui Xie for creating \texttt{knitr} which introduced R Markdown as a way of embedding code in Markdown documents, and \texttt{bookdown} which added tools for technical and longer-form writing.

  Special thanks to \href{http://chester.rbind.io}{Chester Ismay}, who created the \texttt{thesisdown} package that helped many a PhD student write their theses in R Markdown. And a very special thanks to John McManigle, whose adaption of Sam Evans' adaptation of Keith Gillow's original maths template for writing an Oxford University DPhil thesis in LaTeX provided the template that I in turn adapted for R Markdown.

  Finally, profuse thanks to JJ Allaire, the founder and CEO of \href{http://rstudio.com}{RStudio}, and Hadley Wickham, the mastermind of the tidyverse without whom we'd all just given up and done data science in Python instead. Thanks for making data science easier, more accessible, and more fun for us all.

  \begin{flushright}
  Ulrik Lyngs \\
  Linacre College, Oxford \\
  2 December 2018
  \end{flushright}
\end{acknowledgements}

%%%%% ABSTRACT -- Nothing to do here except comment out if you don't want it.
\begin{abstract}
	This \emph{R Markdown} template is for writing an Oxford University thesis. The template is built using Yihui Xie's \texttt{bookdown} package, with heavy inspiration from Chester Ismay's \texttt{thesisdown} and the \texttt{OxThesis} \LaTeX~template (most recently adapted by John McManigle).

 This template's sample content include illustrations of how to write a thesis in R Markdown, and largely follows the structure from \href{https://ulyngs.github.io/rmarkdown-workshop-2019/}{this R Markdown workshop}.

 Congratulations for taking a step further into the lands of open, reproducible science by writing your thesis using a tool that allows you to transparently include tables and dynamically generated plots directly from the underlying data. Hip hooray!
\end{abstract}

%%%%% DECLARATION -- Nothing to do here except comment out if you don't want it.
 	\vspace{45mm}

  This is my original work and has not been presented before for a degree in this or any other University.

  \vspace{25mm}

  Name: John King'athia Karuitha \hfill~Signature: \ldots \ldots \ldots \ldots \ldots \ldots \ldots

  \vspace{3mm}

  Student Number: 855810 \hfill~Date: \today

  \vspace{10mm}

  ORCID: 0000-0002-8204-7034

%%%%% MINI TABLES
% This lays the groundwork for per-chapter, mini tables of contents.  Comment the following line
% (and remove \minitoc from the chapter files) if you don't want this.  Un-comment either of the
% next two lines if you want a per-chapter list of figures or tables.
  \dominitoc % include a mini table of contents

% This aligns the bottom of the text of each page.  It generally makes things look better.
\flushbottom

% This is where the whole-document ToC appears:
\tableofcontents

\listoffigures
	\mtcaddchapter
  	% \mtcaddchapter is needed when adding a non-chapter (but chapter-like) entity to avoid confusing minitoc

% Uncomment to generate a list of tables:
\listoftables
  \mtcaddchapter
%%%%% LIST OF ABBREVIATIONS
% This example includes a list of abbreviations.  Look at text/abbreviations.tex to see how that file is
% formatted.  The template can handle any kind of list though, so this might be a good place for a
% glossary, etc.
% First parameter can be changed eg to "Glossary" or something.
% Second parameter is the max length of bold terms.
\begin{mclistof}{List of Abbreviations}{3.2cm}

\item[CGAP] Consultative Group to Assist the Poor.

\item[DEA] Data Envelopment Analysis.

\item[DMFI] Deposit Taking Microfinance Institution.

\item[EAP] East Asia and the Pacific.

\item[ECA] East and Central Asia.

\item[FinTech] Financial Technology.

\item[FSB] Financial Stability Board.

\item[FSPs] Financial Services Providers.

\item[GNI] Gross National Income.

\item[IFC] International Finance Corporation.

\item[IMF] International Monetary Fund.

\item[IPOs] Initial Public Offers.

\item[KIVA] KIVA Microfund.

\item[KREP] Kenya Rural Enterprise program.

\item[LAC] Latin America and the Caribbean.

\item[MENA] Middle East and North Africa.

\item[MF] Microfinance.

\item[MFI(s)] Microfinance Institution(s).

\item[MIS] Management Information Systems.

\item[MIX] Microfinance Information Exchange.

\item[NBFIs] Non-Bank Financial Institutions.

\item[NDMFI] Non-Deposit Taking Microfinance Institution.

\item[NGO] Non-Governmental Organisation.

\item[OIBM] Opportunity International Bank of Malawi.

\item[OLS] Ordinary Least Squares.

\item[OL-SASL] Opportunity International Savings and Loans LTD.

\item[OSS] Operational Self-Sufficiency.

\item[RCT] Randomized Control Trials.

\item[RFI] Regulated Financial Institution.

\item[SAPs] Structural Adjustment Programs.

\item[SET] Social Exchange Theory.

\item[SFA] Stochastic Frontier Analysis.

\item[SHGs] Self-Help Groups.

\item[SMEs] Small and Medium Enterprises.

\item[SSA] Sub-Saharan Africa.

\item[USAID] United States Agency for International Development.

\end{mclistof} 


% The Roman pages, like the Roman Empire, must come to its inevitable close.
\end{romanpages}

%%%%% CHAPTERS
% Add or remove any chapters you'd like here, by file name (excluding '.tex'):
\flushbottom

% all your chapters and appendices will appear here
\hypertarget{introduction}{%
\chapter*{Introduction}\label{introduction}}
\addcontentsline{toc}{chapter}{Introduction}

\adjustmtc
\markboth{Introduction}{}

Welcome to the \emph{R Markdown} Oxford University thesis template.
This sample content is adapted from \href{https://github.com/ismayc/thesisdown}{\texttt{thesisdown}} and the formatting of PDF output is adapted from the \href{https://github.com/mcmanigle/OxThesis}{OxThesis LaTeX template}.
Hopefully, writing your thesis in R Markdown will provide a nicer interface to the OxThesis template if you haven't used TeX or LaTeX before.
More importantly, using \emph{R Markdown} allows you to embed chunks of code directly into your thesis and generate plots and tables directly from the underlying data, avoiding copy-paste steps.
This will get you into the habit of doing reproducible research, which benefits you long-term as a researcher, but also will greatly help anyone that is trying to reproduce or build upon your results down the road.

Using LaTeX together with \emph{Markdown} is more consistent than the output of a word processor, much less prone to corruption or crashing, and the resulting file is smaller than a Word file.
While you may never have had problems using Word in the past, your thesis is likely going to be about twice as large and complex as anything you've written before, taxing Word's capabilities.

\hypertarget{why-use-it}{%
\section*{Why use it?}\label{why-use-it}}
\addcontentsline{toc}{section}{Why use it?}

\emph{R Markdown} creates a simple and straightforward way to interface with the beauty of LaTeX.
Packages have been written in \textbf{R} to work directly with LaTeX to produce nicely formatting tables and paragraphs.
In addition to creating a user friendly interface to LaTeX, \emph{R Markdown} allows you to read in your data, analyze it and to visualize it using \textbf{R}, \textbf{Python} or other languages, and provide documentation and commentary on the results of your project.\\
Further, it allows for results of code output to be passed inline to the commentary of your results.
You'll see more on this later, focusing on \textbf{R}. If you are more into \textbf{Python} or something else, you can still use \emph{R Markdown} - see \href{https://bookdown.org/yihui/rmarkdown/language-engines.html}{`Other language engines'} in Yihui Xie's \href{https://bookdown.org/yihui/rmarkdown/language-engines.html}{\emph{R Markdown: The Definitive Guide}}.

\hypertarget{who-should-use-it}{%
\section*{Who should use it?}\label{who-should-use-it}}
\addcontentsline{toc}{section}{Who should use it?}

Anyone who needs to use data analysis, math, tables, a lot of figures, complex cross-references, or who just cares about reproducibility in research can benefit from using \emph{R Markdown}.
If you are working in `softer' fields, the user-friendly nature of the \emph{Markdown} syntax and its ability to keep track of and easily include figures, automatically generate a table of contents, index, references, table of figures, etc. should still make it of great benefit to your thesis project.

\begin{savequote}
``Asking for investors to come is the wrong direction completely.
\ldots{} If you are inviting investment from the market, they are
looking for their return. That is the wrong message. Micro-credit should
not be presented to investors as a ground for making a lot of money out
of the poor people -- that is a shame.''
\qauthor{--- Prof.~Muhammad Yunus.}\end{savequote}



\hypertarget{Introduction}{%
\chapter{Introduction}\label{Introduction}}

\minitoc 

\hypertarget{background-to-the-study}{%
\section{Background to the Study}\label{background-to-the-study}}

\noindent Advocates of Microfinance institutions (MFIs) hail the industry for availing financial services to the poor and the financially excluded. Data from 2015, for example, shows that MFIs availed \$92.4 billion to 116.6 million borrowers and accepted \$58.9 billion from 98.4 million depositors \autocite{market2014global}. Supporters of Microfinance (MF) further associate it with improved household welfare \autocite{meador2017food,you2013role}, increased purchasing power and a higher employment rate \autocite{raihan2017macro,lopatta2016microfinance}. Also, MF supporters contend that it leads to improved gender parity \autocite{mafukata2017reciprocal,zhang2017microfinance}, and enables families to cope with the effects of climate change \autocite{fenton2017role} among other benefits \footnote{The quotation comes from an interview of Professor Muhammad Yunus. The video is available at \url{https://nextbillion.net/an-interview-with-muhammad-yunus/}}.

Other researchers, however, have uncovered mixed outcomes from Microfinance (MF) interventions. For example, \textcite{ganle2015microcredit} found that while some women indeed get empowered as a result of access to credit, most have little control over the subsequent spending while a significant proportion suffers harassment from MF agents' for failing to repay the loans. \textcite{van2012impact} also arrive at a similar conclusion.

On the other extreme, some scholars dispute the benefits of MF altogether. For instance, some researchers posit that MF does not boost employment and education among the rural poor \autocite{bauchet2013micro}. Other researchers link micro-credit to increased child labour \autocite{hazarika2008household}, increased gender inequalities in access to finance \autocite{zulfiqar2017does}, and reduced entrepreneurial spirit among the poor \autocite{field2013does}. The apprehension around the high-interest rates charged by MFIs and inappropriate lending practices that fail to account for the social, cultural, and economic context of the target clients is also gaining prominence \autocite{chester2016one}. Taken together, the case against MF paints a bleak future of MF. Some studies recommend a re-examination of the MF business model and call for better regulation of the industry \autocite{johnson2013microfinance,ghosh2013microfinance}. Without reforms, conclude \textcite{chester2016one}, ``the MF industry could not only ruin the lives of many borrowers but also ruin itself'' (pp.~28).

Despite these contradicting views, MFIs continue to attract interest from governments, state agencies, donor organisations, philanthropists and increasingly commercial providers of capital. Initially, MFIs ran on the non-governmental organisation (NGO) model relying chiefly on donors to finance their operations \autocite{d2017ngos}. The NGO oriented model has been emphasising the social mission of MFIs- that is, availing financial services to the poor and the financially excluded \autocite{ashta2012compartamos}. The model played down the profit motive \autocite{ashta2012compartamos}.

However, in the last two decades, some MFIs have been transforming their institutional structure from NGOs to commercial entities \autocite{d2017ngos}. This study seeks to evaluate the possible impacts of the transformation of MFIs on their financial sustainability and outreach to the financially excluded. Transformation is a process where an MFI converts from a donor-funded NGO to a regulated financial institution (RFI) that derives its capital primarily from commercial sources. The rationale for the transformation is that it would not only enable MFIs to access commercial sources of capital but also lead to improved financial sustainability, efficiency and social performance \autocite{louis2013financial}.

Other benefits of the transformation include improved customer service, a more extensive range of products, and enhanced control and governance\autocite{srnec2008transformation}. Consumers and other stakeholders would also reap the benefits of the regulation of MFIs both directly \autocite{meagher2006microfinance}, and indirectly \autocite{hartarska2007regulated}. Nevertheless, the process of transformation is complicated and dependent on the country-specific regulatory framework. Thus, although most research delineates a point in time when an MFI ceases to be an NGO and becomes a commercial entity, there are essential preparations before the transformation that are often overlooked \autocite{d2017ngos}.

The transformation of MFIs leads to a change in their capital structure and hence, governance. As research in corporate finance indicates, there is a link between the capital structure of corporations and their financial performance. For example, family-owned businesses that have lower leverage exhibit higher profitability \autocite{hamid2015capital} in line with the pecking order theory of capital structure. Other studies indicate that leverage is positively related to performance \autocite{fosu2013capital,berger2006capital}, including that of MFIs \autocite{kar2012does} in line with the agency theory. The effects of capital structure on financial performance may vary across industries, regions, and even depending on the performance metric chosen.

However, MFIs have a double bottom line. Transformed MFIs strive to perform well financially as well as socially. Turning a profit enables MFIs to be sustainable going concerns. Social performance, on the other hand, is the source of legitimacy for MFIs, and the critical reason for receiving donations and subsidies. MFIs should offer financial services to the section of the population neglected by the mainstream financial institutions. The existence of a social mission in addition to financial goals makes it harder to evaluate the effects of capital structure on MFI performance compared to purely commercial firms.

There is an extensive body of research on the effects of the transformation of MFIs. Much of this research has compared the performance of MFIs before and after the conversion. There is a consensus that the conversion of MFIs has the potential to affect both the financial and social performance of the transformed MFIs \autocite{chahine2010social,mersland2010microfinance}. However, there is disagreement regarding the direction and magnitude of the effects of transformation \autocite{mersland2010microfinance,d2017ngos}.

Similarly, the few studies that examine the effects of the resultant capital structures of the transformed MFIs on their performance have uncovered mixed results. For instance, \textcite{bogan2012capital} established that the use of grant capital by MFIs led to decreased sustainability and operational self-sufficiency. \textcite{hoque2011commercialization} and \textcite{kar2012does} uncover a negative relationship between leverage and outreach, whereas \textcite{kyereboah2007determinants} finds the opposite using data from Ghana. This study examines the sources of capital for transformed MFIs in Africa and the extent that new capital structures impact on financial inclusion.

For MFIs that have transformed, researchers have yet to establish the factors that influence their level of sustainability and outreach. Much of the research examines financial and social performance separately instead of being the two sides of the same coin. Moreover, although a substantial number of MFIs have transformed, some still operate as NGOs. Little research that questions why some MFIs transform while others retain the NGO model. If financial sustainability for MFIs is so desirable, then it is not apparent why some MFIs would stick to an NGO, not-for-profit model that is not sustainable.

In light of the highlighted deficiencies, this study has the broad objective of evaluating the possible impacts of the institutional transformation of Microfinance institutions in Africa. The impacts include how such transformation influences sustainability and financial inclusion by MFIs in Africa. The study utlises a sample dataset comprising 775 MFIs from 40 countries in Africa available. The data is available from the Microfinance Information Exchange (MIX) and the MIX market database of the World Bank.

\hypertarget{ngos-to-banks-rationale-for-transformationing-mfis}{%
\section{NGOs to Banks: Rationale for Transformationing MFIs}\label{ngos-to-banks-rationale-for-transformationing-mfis}}

\noindent Critics of the NGO based model of MFIs cited its unsustainability and blamed it for crowding out alternative providers of MF services \autocite{kota2007microfinance}. Moreover, donors could not be relied upon to fund MFIs indefinitely. Also, dependence on donor funding left the MFIs exposed to global macroeconomic shocks \autocite{d2017ngos}, which spread across countries through the financial system \autocite{schnabl2012international}. For instance, the global financial crisis led to a decline in development assistance and capital flows to developing countries which, in turn, affected MFI financing \autocite{leach2012global,wagner2013vulnerability}.

Thus, in the absence of alternative sources of finance, MFIs are likely to experience funding shortages in crisis periods \autocite{constantinou2011financial}. Additionally, researchers have uncovered a link between the state of bilateral political relationships between countries and the flow of funds to MFIs \autocite{garmaise2013cheap}. Consequently, a diplomatic or trade row could also affect the flow of funding to MFIs, especially in the characteristically vulnerable developing countries. Thus, the NGO model is not only unsustainable but also susceptible to both political and economic dynamics.

On this backdrop was a realisation that availing financial services to the poor could be pursued as a profit-based value proposition \autocite{rhyne1999microfinance}. The introduction of the profit element meant that MFIs could carry out their services without relying extensively on donations and subsidies \autocite{duvendack2015mis}. In the long run, the sustainability arising from the transformation of MFIs would enable them to expand access to financial services to the poor and the financially excluded \autocite{brown2012microfinance,sarma2011ngo}. However, given that the central focus of MF was on the poor, researchers began to question the compatibility of the pro-poor agenda of MFIs with the profit-oriented school of thought. At the heart of the debate, which continues to date, is that commercialisation of MFIs could result in mission drift \footnote{Mission drift occurs when, upon the conversion from the NGO to the commercial model, the MFIs place more emphasis on attaining the financial objectives. This leads to a reduced focus on the social objectives of MFI of alleviating poverty by availing financial services to the poorest of the poor and the financially excluded segments of the society.} away from serving the poor in pursuit of profits \autocite{im2015profits,mia2017mission}.

Literature is abundant on the question of whether the transformation of MFIs causes mission drift. Some studies take the position that the transformation results in mission drift \autocite{mia2017mission,wagenaar2012institutional,lopatta2017sustainable,roberts2013endogeneity}. Other scholars find the transformation to be beneficial or at least not causing mission drift \autocite{im2015profits,lutzenkirchen2012microfinance,quayes2012depth,mersland2010microfinance}. Another strand of related research uncovers both positive and negative results from the transformation\autocite{kar2012does,caudill2009microfinance}. There is also lots of research on the effects of the conversion on social versus financial outcomes and efficiency \autocite{bogan2012capital,kar2012does,tchuigoua2014institutional,khachatryan2017performance}. However, little research has probed the questions raised in this proposed study. The next section is an overview of the transformation of MFIs globally and in Africa.

\hypertarget{overview-of-the-transformation-of-mfis}{%
\section{Overview of the Transformation of MFIs}\label{overview-of-the-transformation-of-mfis}}

\noindent In 1992, PRODEM, an MFI in Bolivia converted into Bancosol, a commercial bank. This change marked the beginning of the transformation of MFIs to commercial entities. Since then, numerous MFIs have transformed (Table 1.1.) . Note that most of the initial MFIs transformed into commercial banks or finance companies, apart from Card Rural Bank, which changed to a rural bank and Banco ADEMI which turned into a commercial development bank. This trend has been consistent to this day. Beyond the MFIs highlighted in Table 1.1, other MFIs that have transformed outside Africa include BRAC (Bangladesh) and ACLEDA (Cambodia), among others.

\begin{table}

\caption{\label{tab:unnamed-chunk-3}Sample of Transformed MFIs}
\centering
\fontsize{9}{11}\selectfont
\begin{tabu} to \linewidth {>{\raggedright}X>{\raggedright}X>{\raggedright}X>{\raggedright}X}
\toprule
NGO\_name & New\_name & New\_structure & CountryYear\\
\midrule
PRODEM & BANCOSOL & Commercial Bank & Bolivia, 1992\\
CORPOSOL & FINANSOL & Commercial Finance Company & Colombia, 1993\\
AMPES & Financiera Calpia & Finance Company & El Salvador, 1995\\
PRO CREDITO & Caja Los Andes & Finance Company & Bolivia, 1995\\
CARD & CARD Rural Bank & Rural Bank & The Philippines, 1995\\
\addlinespace
ADEMI & Banco-ADEMI & Commercial Development Bank & Dominican Republic, 1998\\
ACP & MIBANCO & Commercial Bank & Peru, 1998\\
K-REP & K-REP Bank & Commercial Bank & Kenya, 1999\\
\bottomrule
\multicolumn{4}{l}{\rule{0pt}{1em}Source: Campion and White (1999)}\\
\end{tabu}
\end{table}

In Africa, many MFIs have changed to commercial entities from the year 2000 and beyond. In Uganda, for instance, the Bank of Uganda granted an operating license to Uganda Microfinance Union (UMU), three years after starting off the road to the transformation. In Kenya, several MFIs have transformed into commercial entities, including Faulu (2010) and the Kenya Women Finance Trust (2010). OIBM in Malawi (2002), PRIDE (2009) in Tanzania and OI-SASL (2013) in Ghana, has also transformed into commercial entities.

The examination of the transformation of the MF industry should sensibly start with the review of the changing landscape in the national financial sectors. Typical characteristics of financial sectors in most countries include Intense competition, increased innovation, and rapid technological changes. Consequently, the MF services space is no longer the preserve of MFIs. The industry has attracted mainstream commercial banks, MF-oriented commercial banks, credit unions, building societies, and insurance companies. Financial technology (FinTech) firms have also come in (individually or in partnership with mainstream financial institutions) by offering mobile \footnote{An example of this is M-Shwari, a mobile based platform operated by Safaricom and the Commercial Bank of Africa. It allows customers to save and borrow money using their mobile phones. The savings also attract interest.}, internet-based \footnote{Kiva Microfunds (commonly referred to as kiva.org) is an example of an internet-based MF services provider. Operating in more than 80 countries, the platform aims at offering microloans to end poverty. Other examples include Branch (branch.co) and Tala (tala.co).} MF service, and peer-to-peer/crowdlending \footnote{Peer to peer lending, also called crowdlending is a system where loan applicants are connected to investors with cash to lend through an online platform. Note that the platform providers do not take deposits or lend out their money but merely link borrowers to prospective lenders. Cumplo and prosper.com are among the prominent examples of these kinds of MF services providers.} (Table 1.2). Although some researchers have argued that the digital divide could limit the effectiveness of FinTech based MFIs outfits \autocite{yartey2017subaltern,fd2017}, the prevalence of low-cost smartphones indicates that digital MF could be the future of the industry \autocite{yum2012wisdom}. To sum up this point, the changing MF landscape means that MF providers have to streamline their operations to improve their efficiency and to maintain relevance in the market.

Furthermore, although MF was initially targeted at the poorest of the poor, predominantly living in remote rural villages where mainstream banks could not reach, MFIs now target all the financially excluded individuals. Thus, MFIs (and their competitors) do offer MF services in urban areas, and even operate in developed countries {[}kota2007microfinance{]}. This paradigm shift has had several implications. For example, there has been a rise of microfinance-oriented commercial banks that were never MFIs initially. In other cases, some commercial banks have acquired MFIs, and thus incorporated into the mainstream commercial banking portfolio. In some other instances, there have also been mergers between an MFI and a commercial bank \footnote{Equity Bank in Kenya is an example of a microfinance oriented commercial bank. MFIs such as CONFIE in Nicaragua and Genesis in Guatemala arose from mergers between an MFI and a commercial bank.}.

Moreover, there have been mergers between MFIs. In the most extreme cases, MFIs have converted entirely into commercial banks and have been duly regulated under banking laws. This rapidly shifting landscape may also inform the need for MFIs to transform to efficiently compete in a market that is gaining traction amongst players from mainstream financial industries.

\begin{table}

\caption{\label{tab:unnamed-chunk-4}Sample of Internet and Mobile MF Providers}
\centering
\fontsize{9}{11}\selectfont
\begin{tabu} to \linewidth {>{\raggedright}X>{\raggedright}X>{\raggedright}X}
\toprule
MF\_Provider & Country & Platform\\
\midrule
KIVA Microfund & Global & The Internet\\
Stonehenge Telkom & Global & The Internet/ Mobile\\
M-Shwari & Kenya & Mobile\\
AYE Microfinance & India & The Internet/ Mobile\\
CUMPLO & Chile & The Internet (Peer to peer)\\
\addlinespace
Prosper.com & USA & The Internet (Peer to peer)\\
Popfunding.com & South Korea & The Internet (Peer to peer)\\
\bottomrule
\multicolumn{3}{l}{\rule{0pt}{1em}Source: Authors' Compilation from the Literature}\\
\end{tabu}
\end{table}

The transformation of MFIs is not without its challenges. In one particularly extreme case, the conversion of MFIs to for-profit entities was declared unconstitutional in Kosovo \autocite{hasani2013ustav} \footnote{The Kosovar Civil Society Foundation (KCSF), FOL Movement, Kosovo Democratic Institute (KDI) and 55 NGOs filed a suit challenging the legality of the conversion of Microfinance NGOs into joint stock companies. In 2013, the conversion was declared unconstitutional in the Republic. The full judgement can be accessed from the following site, \url{http://www.gjk-ks.org/repository/docs/KO97_12_AGJ_ANG.pdf}}. Also, three additional categories of problems arise in the process of converting MFIs into for-profit legal corporations: the integration into the formal financial system, ownership, and governance, and organisational development \autocite{campion1999institutional}. The integration of the transformed MFIs into the formal financial system raises several challenges. For example, the political and economic environment determines the timing of successful transformations. Thus, the economic and political climate is an essential factor to consider \autocite{kenya2012transforming}.

Transformation also implies setting up a board that oversees the running of the organisation. The board typically sets the mission and vision of the organisation as well as its investment strategy. Thus, an ineffective board could hinder the implementation of transformation \autocite{campion1999institutional} and even the performance of an MFI post-transformation. Lastly, issues such as the organisational culture and human resource development are critical to a successful transition. For most MFIs that have moved from the NGO model, the management has to act to alter the culture of the organisation to cater for a more commercial, and thus, a more customer-centric orientation \autocite{christen2001commercialization}. These adjustments have resulted in substantial costs of training and mentorship.

\hypertarget{motivation-for-the-study}{%
\section{Motivation for the study}\label{motivation-for-the-study}}

\noindent Different schools of thought hold differing views regarding the potential consequences of the transformation of MFIs. The sustainability perspective \footnote{The sustainability approach to the provision of microfinance is also called the financial systems approach. The approach takes the position that the integration of MF with the mainstream financial sector is the only way to ensure that MF could achieve large scale outreach without continued donor dependency \autocite{rhyne1999microfinance}. Microfinance enters the marketplace, New York, USAID.} considers the transformation as desirable for MFIs to attain financial self-sufficiency. On the other hand, the welfare standpoint sees the transformation as conflicting with the social mission of MFIs. The win-win approach attempts to reconcile both the welfare and the sustainability perspectives by bringing together the potential benefits from both schools of thought \autocite{kodongo2013individual}. The debate between the proponents of these three schools of thought has dominated the research on the institutional transformation in MFIs.

A broad range of research has documented the institutional change in the MF industry (prominent first examples include,\autocite{ledgerwood1998microfinance,ledgerwood2006transforming}. The subsequent research examined the effects of the change on the trade-off between financial sustainability and social performance. A remarkable pioneering example of research in this area is that of \textcite{frank2008stemming}, who found that transformation led to a higher client outreach, higher growth in the loan portfolio, and higher product diversification. More importantly, they established that conversion allowed more women customers to access services, although the overall percentage of women receiving the services declined. Subsequent studies support their view, for example, \autocites[ ]{hartarska2012governance,bos2015practice}{d2017ngos}.

A substantial extension of studies on the transformation of MFIs has examined the financing structures in the transformed MFIs. However, there are mixed outcomes from the research, and hence there is no consensus on the direction and magnitude of the effects of the transformation on financial and social performance. For instance, \autocite{bogan2012capital} examined the relationship between capital structure and MFI efficiency and sustainability. The study uncovers a link between capital structure, MFI size, and financial performance. Specifically, there is a negative relationship between the use of grants and financial performance. These results are close to the outcomes of the research by \autocite{hudon2011efficiency}. They also found a positive relationship between grant financing at low levels and financial performance, which turns negative beyond a certain threshold, in line with \autocite{d2017ngos}.

A related study by \textcite{kar2012does} found no relationship between debt financing and breadth of outreach and women participation as loan clients and recommended research along this line about equity financing. Subsequent research has not resolved this stalemate \autocite{hoque2011commercialization,kyereboah2007determinants,khachatryan2017performance,d2017ngos}. The mixed outcomes from previous studies motivate the focus on the effects of capital structure on the performance of MFIs in this research.

Much of the existing strand of research on MFI transformation stems from the perceived possibility of mission drift by MFIs that have undergone the change from the NGO based model to the commercial model. The primary manifestation of the transformation has been the domination of debt, deposits, and equity in the capital structure of the transformed MFIs (The Microfinance Information Exchange, 2017). Figure 1.1 shows the funding structure of 1330 MFIs across the globe that avail their data to the MIX pooled database. The regional disparity in the financing structure is particularly striking. In Latin America and the Caribbean (LAC), Eastern Europe and Central Asia (ECA), East Asia and the Pacific (EAP), and Africa, MFI source their capital mainly from deposits. In contrast, MFIs in South Asia get most of the capital from borrowings. In North Africa and the Middle East, debt and equity are equally likely to be a source of funding for MFIs.

Except for the MENA region, equity consistently consists of less than 25\% of the funding of MFIs, with the figure being lowest in Africa (18\%), LAC (18\%), and ECA (16\%). Debt is a chief source of funding compared to equity with four of the six regions having debt accounting for more than 25\% of the total funding. The exception is Africa and LAC, where borrowings account for 11\% and 18\% of the total funding. Notable also is the unusually small proportion of deposits to total capital observed in the MENA region. The regional disparity in financing patterns for MFIs provokes several questions. Why is there such a regional disparity in capital structures among MFIs?

Moreover, in the African setting, do such regional disparities exist? If the disparities exist, then what explains the disparities? This study seeks the determinants of the observed financing structures aming transformed MFIs to inform policy-making not only for the MF sector but also targeting the entire capital market. This study addresses this open issue from the capital structure perspective regarding the financing of hybrid organizations.

The examination of the capital structure of MFIs is necessary because the transformation of MFIs implies that both the proportion and the importance of donor sector funding would be declining in most MFIs. The increasing importance of commercial funding the commercial interest perspective should be the foundation from which researchers examine whether or not transformed MFIs are achieving financial sustainability and social mission. It is only when such research establishes how the business orientation affects the social mission of MFIs that the need for corrective action gets flagged timeously.

\begin{figure}
\centering
\includegraphics{_main_files/figure-latex/unnamed-chunk-5-1.pdf}
\caption{\label{fig:unnamed-chunk-5}Funding structure of MFIs Across the Globe by Region (2015)}
\end{figure}

Although the transformation of NGOs has its merits, most MFIs have still not transformed\autocite{d2017ngos}. Therefore, a key output of the proposed study will be the motives behind some MFIs transforming while other MFIs retain the NGO model. The current literature has been overly concerned with the transformed MFIs and not addressed this issue. Even among the transformed MFIs, researchers have not uncovered the drivers of the decision by MFIs to transform and the necessary preconditions for transformation. The existing research takes for granted that MFIs transform to be sustainable. Given the benefits of transformation touted in the numerous studies, then most MFIs should have already transformed.

Moreover, as \textcite{morduch2019challenges} argues, if there is no trade-off between financial performance and social performance of transformed MFIs, then NGOs would not exist. However, this is not the case. It appears, therefore that some additional factors influence the decision by MFI to either transform or retain the NGO model. As noted, there is no conclusive evidence on the effects of MFI transformation on their performance. These contradicting results could be because of numerous unknown reasons. It is in order, therefore, to establish the factors that influence the sustainability and outreach of transformed MFIs. Similarly, establishing the factors that moderate the relationship between capital structure on the one hand and the financial performance and social performance of MFIs in Africa will be a novel contribution to the research.

Recent literature also suggests that the performance of MFIs is dependent on the broader macroeconomic conditions as well \textcite{ahlin2011does}, and is country-specific \autocite{d2017ngos}. Thus, in the analysis of the institutional transformation of MFIs, cross-country and regional differences must be factored in. However, most studies have not considered the cross country and regional variations. Failure to consider regional disparities means that the extant research fails to capture local contextual peculiarities.

These regional differences motivate the choice of MFIs in Africa as the unit of analysis. By focusing on Africa, the study will isolate the regional and country heterogeneity of the effects of transformation, an issue that could have affected the results of research based on pooled global datasets. Thus, the output of the study will be unique to Africa and therefore more actionable. To sum up this section, the proposed study significantly extends the existing literature on MFI transformation. The next section outlines the purpose statement.

\hypertarget{purpose-statement}{%
\section{Purpose Statement}\label{purpose-statement}}

\noindent The primary aim of this proposed research is to evaluate how the the transformation of MFIs in Africa impacts their financial and social performance.

\hypertarget{research-questions}{%
\section{Research Questions}\label{research-questions}}

\noindent The study specifically seeks answers to the following research questions.

\begin{enumerate}
\def\labelenumi{\arabic{enumi}.}
\tightlist
\item
  Why do some MFIs in Africa transform into the commercial model while others retain the NGO Model?
\item
  To what extent has the institutional transformation of MFIs in Africa affected financial inclusion?
\item
  After transformation, what factors explain the joint level of sustainability and outreach by MFIs in Africa?
\item
  What are the factors that influence the choice of financing sources by transformed MFIs in Africa?
\item
  How is capital structure related to the performance of transformed MFIs in Africa?
\end{enumerate}

\hypertarget{significance-of-the-study}{%
\section{Significance of the Study}\label{significance-of-the-study}}

\noindent In pursuing the stated objectives, this study will fill a theoretical deficiency in the capital structure of quasi-commercial organisations that have a social dimension of value. The context of shareholder and debtholder primacy was the foundation of the development of capital structure theories. Arguments from this genesis of capital structure theories have failed to factor in corporations with extra dimensions of value, such as the achievement of social goals. As a case in point, \texttt{Modigliani\ and\ Miller} capital structure theory may not be entirely applicable to MFIs where value also has a social facet. In the Modigliani and Miller capital structure theory, the value of a firm is the sum of equity and debt components in the capital structure of a corporation. The value of a company is an increasing function of the debt proportion of capital structure up to the point where the costs of financial distress outweigh the benefits of the interest tax shield.In the process, the study will also fill an empirical deficit on the relationships between capital structure on the one hand, and financial performance, social performance, and efficiency of transformed MFIs in Africa. The empirical output has implications on policy direction relating to MFIs in Africa.

The study also has a variety of policy implications. These policy implications are related to the research objectives. For instance, what are the factors that drive MFIs to transform or fail to transform? The answer to this question would inform the crafting of policy if transformation is indeed desirable. Moreover, what are the drivers of financial inclusion, outreach, and sustainability of MFIs after the transformation? Lastly, what are the determinants of the choice of financing sources? It is worth noting that if the drive for institutional transformation is to achieve the desired effects, then the change in the MFI funding structures must not compromise the achievement of social objectives. If the conversion of MFIs negatively affects social performance, there would be no way to justify the transformation of MFIs. If indeed the transformation of MFIs was meant to improve their sustainability, this must not be to the detriment of social performance.

Also, if the transition results a decline in social performance, then the transformation could be treated as the transition point of transformed MFIs into the mainstream banking system \autocite{kent2013bankers}. It is crucial therefore to investigate the outcomes of the change to inform the design of a framework that would not expose the poor and financially excluded to the same level of exclusion by the financial system that MFIs ought to address. Doing this would be tantamount to allowing MFIs to use the underprivileged as a ladder to climb into the mainstream system, and abandoning them shortly after, which raises ethical questions.

Similarly, the identification of factors that influence the choice of financing structures has implications for the design of policies aimed at the capital market. For the same reasons, the identification of factors that influence the social performance of transformed MFIs will also be useful, not just for policy design, but also for managerial decision making. Also, the calibration of a mix of debt and equity for transformed MFIs that optimises financial and social performance will inform both policy-making and managerial decision making.

Finally, the efficiency of MFIs is a significant determinant of both the financial and social performance of MFIs. According to the efficiency theory, the positive relationship between concentration and profitability is indicative of the tendency of firms that are efficient to be successful and hence dominant in their industries \autocite{lipczynski2005industrial}. Efficiency gains could result from economies of scale and cost-saving schemes initiated by the management. Enhanced efficiency by MFIs is, therefore desirable. However, the efficiency of MFIs should encompass both financial and social dimensions. The entry of commercial sources of capital may affect the both the financial and social efficiency of MFIs in line with agency theory. It is therefore vital to establish the relationship between capital structure and efficiency to inform both the management and stakeholders of the dynamics in the MFI sector regarding effectiveness in the new era of commercial funding.

\hypertarget{outline-of-the-study}{%
\section{Outline of the Study}\label{outline-of-the-study}}

\noindent The remainder of the study will be structured as follows. Chapter two locates microfinance in the financial intermediation space and traces the genesis and evolution of the transformation. Chapter three lays out the theoretical framework of the study, focusing on the agency theory and profit incentives, capital structure theory, and the institutional theory. The subsequent five chapters each delves into one research question, laying out the methodologies applied and the results. The last chapter concludes.

\hypertarget{cites-and-refs}{%
\chapter{Transformation of Microfinance Institutions in Africa}\label{cites-and-refs}}

\chaptermark{Transformation of MFIs in Africa}

\minitoc 

\begin{center}

\textbf{Abstract}

\end{center}

We examine the factors driving non-governmental organisation microfinance institutions to convert to the commercial, profit-oriented model. Using data from the World Bank and the International Monetary Fund (IMF), we ran logit, probit, and multinomial logit models with NGOs as the base outcome. At the firm level, age and size influence transformation, whereas legal tradition, institutional quality, and stock market development are significant factors at the country level. Older firms are less likely to be NGOs, as are MFIs in civil law countries. Larger MFIs and MFIs located in countries with ``other'' legal traditions are more likely to follow the commercial model. Institutional quality raises the chances of conversion, while stock market capitalisation has a negative relationship with transformation. The models are statistically significant, and the results remain robust to removing outliers and other checks.

\newpage

\hypertarget{background}{%
\section{Background}\label{background}}

The modern micro-finance (MF) industry draws its popularity from the promise of providing appropriate and affordable financial services to the population under-served by mainstream financial intermediaries \autocite{morduch1999microfinance,morduch2000microfinance}. The motivation for reaching out to the unbanked draws from researchers and development practitioners' view that financial inclusion leads to welfare improvements. As an example, some scholars associate access to finance with more business start-ups, higher savings rates, improved health, less child mortality, and higher education attainment by the poor, although some scholars downplay these findings \autocite{klapper2015role,o2017systematic,shahriar2017lender}. Much of the initial efforts to provide financial services to the unbanked rested on Microfinance. Microfinance refers to either the practice of delivering appropriate and affordable financial facilities to the financially excluded or the providers of such micro and small denomination financial services \autocite{ledgerwood2006transforming}.

Pioneer Microfinance Institutions (MFIs), like Grameen Bank, primarily operated as Non-Governmental Organisations (NGOs), following the welfare approach, where the profitability of the institution played a second role to availing financial services to the poor \autocite{chahine2010social,d2017ngos}. However, the paradigm shift is towards the financial systems approach where MFIs operate under commercial principles, leading to charges of ``financialisation'' of poverty \autocite{mader2015financialization}, with the relatively higher interest rates charged to the clients equated to a ``poverty penalty'' \autocite{chen2017microfinance}. Some scholars and practitioners argue that MFIs following the profit-oriented commercial model are subject to ``mission drift'' though more financially sustainable''. Mission drift refers to situations where MFIs lessens their commitment to availing financial services to the financially excluded to pursue profits \autocite{jia2016commercialization,mia2017mission}.

However, given the social mission inherent in MF, MFIs following the financial systems approach risk their legitimacy in society and local and international donor community, a significant source of funding even for commercial MFIs \autocite{nason2018behavioral}. For these reasons, some researchers and development practitioners vouch for the welfare approach, where MFIs focus primarily on the mission to reach the financially excluded without emphasising profits. Most MFIs following the welfare approach are non-governmental organisations (NGOs). Though not subject to mission drift, the NGO model is over-reliant on volatile local and international donor funds and government subsidies \autocite{garmaise2013cheap,d2017aid}. Additionally, NGOs may crowd out the efficient, commercial-oriented microfinance providers \autocite{kota2007microfinance}, which may hurt aggregate welfare in the long run.

The win-win school attempts to reconcile the commercial model and welfare approach to microfinance. Adherents of the win-win school postulate that it is possible to achieve both financial sustainability (that is, turn a profit) and, at the same time, reach the financially excluded \autocite{kodongo2013individual}. As a case in point, some researchers argue that MFIs could strive to generate profits by offering financial services to the relatively well-off at market rates. The MFIs could then use the returns to subsidise the provision of financial services to the poor under a form of price discrimination, leading to ``mission expansion'' as opposed to mission drift \autocite{mersland2010microfinance}.

Globally, the shift from the pure welfare approach of MF provision is gaining ground. Most of the transformed MFIs operate at some point on the continuum between the NGO, welfare model and the commercial, profit-oriented approach \autocite{armendariz2013subsidy,d2013unsubsidized,hishigsuren2006transformation}. It all started in 1992 in Bolivia when Prodem, an NGO, converted to a commercial bank, Bancosol \autocite{fernando2004micro,creedy2018types}. Since then, many MFIs worldwide have converted from NGOs to commercial firms seeking to make profits.

This article explores the factors that drive the transformation of MFIs from NGOs to for-profit firms in Africa. To this end, we use a panel dataset of 705 MFIs in Africa from the MIX pooled database, with additional data from the World Bank like the World Development Indicators (WDI), the Worldwide Governance Index (WGI), and the Global Financial Development Database (GFDD). We focus on Africa, given the relatively low levels of financial inclusion on the continent \autocite{demirguc2018global}and the shortcomings inherent in combining data from different regions that may yield results that are not actionable. As \textcite{d2017ngos} and \textcite{wang2015ownership} suggest, the nature and performance of MFIs are country-specific. Hence, research focusing on particular regions, countries or even firms could better inform policymaking.

This article contributes to the literature in two main ways. First, the study sheds light on the drivers of the transformation of microfinance institutions in Africa. Much of the literature has not examined this phenomenon, focusing, instead, on the shift's consequences and how MFIs can balance financial and social missions \autocite{d2013unsubsidized,forkusam2014does,mia2017mission}. We believe that our analysis could form a reasonable starting point for analysing the transformation of MFIs in other regions or countries. Second, we detail the linkages between the drivers of micro-finance institutions' transformation, showing how they interact to change the likelihood of conversion. We highlight the pitfalls that bedevil analysis of pooled data from heterogeneous sources, which may mask crucial differences or similarities between the analysis units.

The rest of the article is structured as follows. Section 2.3 presents the theoretical underpinnings and empirical findings related to the study. Subsequently, in section 2.3.1, we summarise the findings of this study before delving into the analysis method and attendant results in sections 2.4 and 2.5, respectively. Section 2.6 concludes.

\hypertarget{related-literature}{%
\section{Related Literature}\label{related-literature}}

Much of the early literature on the institutional change of MFIs dealt with the theoretical, philosophical, and historical basis for transforming MFIs from NGOs to commercial firms and the potential impacts of such conversion \autocite{campion1999institutional,christen2001commercialization,gutierrez201920,zaby2019science}. Views of scholars on the transformation of MFIs drew from the institutional theory. The theory seeks to explain persistence and convergence in organisations, including change and de-institutionalisation within firms \autocite{scott2004institutional}. For MFIs, the transformation has picked pace with the dominance of neoliberalism after the cold war \autocite{ostry2017}.

The institutional theory holds that the institutional environment is more influential in developing formal structures in organisations than market pressures \autocite{maggio1991}. Coercion is one form of pressure from the institutional environment that makes organisations adopt institutional structures and practices. Institutional theorists note that stakeholders could force firms to adopt specific organisational structures and practices without critical scrutiny to gain legitimacy in the institutional environment \autocite{scott2004institutional,martinez2017coercive}.

With this hindsight, \textcite{bateman2010doesn} traces the pressure to convert MFIs from NGOs to the commercial model to the rise of neo-liberalism and the insistence that firms be financially self-sufficient instead of relying on government subsidies and, in the case of MFIs, donor funds. The wave of economic liberalisation and privatisation commenced in the early 1990s due to neo-liberalism \autocite{silva1998neoliberalism}. Researchers point to pressure from financiers of MF such as USAID and the World Bank as a significant driver for the decision for MFIs to transform \autocite{ostry2017}.

However, given that MFIs have a social mission, the transition to a profit-oriented positioning is bound to conflict with the social objectives and threaten the MF industry's legitimacy \autocite{ramus2017,nason2018behavioral}. Specifically, the quest to satisfy both financial goals and the social mission is likely to conflict, which may cause ``mission drift'' \autocite{mersland2010microfinance,mia2017mission}, where MFIs give greater priority to profitability than outreach to the unbanked.

Nonetheless, the push from donors to transform MFIs seems to contradict the slow pace of the transformation. NGOs still form a substantial proportion of MFIs in Africa, accounting for 32\%, second only to Non-Bank Financial Institutions (NBFIs) at 40\% \autocite{market2017global}. A question arises regarding the factors behind the persistence of certain organisational forms of MFI provision in Africa and globally. \textcite{pashkova2016business} tackled this question. They found that the cooperative model is prevalent in economies with civil law systems, low inflation levels, and high economic growth rates.

In contrast, NGO type MFIs feature in countries with high inflation rates and low economic growth levels, meaning that NGOs help the poor cope during challenging economic times. The commercial banking model features most in economies with common law legal systems. However, the study by Pashkova, et al.~does not explicitly address the transformation question, factors that determine the transformation of MFIs from NGOs to a commercial model, which we explicitly address in this article.

The capacity of the capital markets and their antecedents may raise the propensity for the transformation of MFIs. MFIs in countries with well-developed capital markets can efficiently or more readily issue debt and equity instruments and raise public deposits \autocite{allen2013resolving,allen2014african}. Available literature points to legal tradition, governance, and education, as drivers of financial development in a country \autocite{rajan1998financial,baltagi2009financial}. By extension, these variables drive economic growth by the financial development-economic growth nexus literature \autocite{claessens2003financial}. The size of an MFI in terms of assets base, structure, and tangibility could enhance its capital acquisition capacity in line with the trade-off theory of capital structure \autocite{barclay2005capital,gwatidzo2009corporate,ojah2016effects}.

\textcite{ledgerwood2006transforming} attribute the financing structure, hence the organisational form of an MFI, to the institutional life cycle. For instance, in the early stages, most MFIs operate as NGOs relying on donations and concessionary funds, given that commercial funders deem them too risky. Later, they use government subsidies and equity funding from NGOs and public investors to supplement their funding. In the final consolidation phase, most MFIs rely on debt, using foreign donors as guarantees. Deposits also play a prominent role in this consolidation phase as MFIs increasingly adopt the commercial model.

Thus, the age of an MFI may have a bearing on both the capital structure and organisational structure. Regulatory provisions relating to the ways MFIs can raise capital and historical legacies on saving and lending \autocite{bayai2016financing} may explain the remaining firm, country and regional disparities. The agency conflict that follows the introduction of debt and equity brings to the fore the potential conflict between optimising financial returns and sticking to the social mission of MFIs, the second central strand of research on MFI conversion \autocite{nurmakhanova2015trade,bayai2016financing,abdulai2017trade,awaworyi2018sustainability}.

Quality of institutions features prominently in explaining the investment climate in a country. Researchers have primarily explained away the ``Lucas Paradox'' using the differences in, among others, the institutions, especially the capacity to enter into and enforce contracts and guarantees against state appropriation of private property - property rights \autocite{azemar2013has,goktan2015explanation}. MFI transformation connotes the entry of private, profit-oriented capital that seeks returns and favours countries with refined institutions. Moreover, researchers have variously cited the quality of institutions as drivers of the ease with which firms, MFIs included, can access private funds by fostering a well developed financial ecosystem \autocite{huang2010political,kaidi2019financial}. In this case, the financial system would consist of equity and debt markets where private investors could buy stakes in or lend funds to the MFIs that seek to go commercial. The following section highlights the results of the study.

\hypertarget{summary-of-results}{%
\subsection{Summary of Results}\label{summary-of-results}}

The output from the data analysis shows that at the country level, it is legal tradition, stock markets development, and governance (institutional quality) that relate significantly to the likelihood of the transformation of MFIs. At the firm level, the age and size of the MFI raise the probability of conversion. There is also a robust time trend towards the commercialisation of MFIs which points to the growing acceptance of microfinance's profit orientation. Regional differences are also evident with North Africa represented by NGOs in the sample data, probably due to religious constraints. GDP growth rate and education levels are not significant determinants of the probability of transformation. However, like the stock market to GDP, private credit to GDP has a negative coefficient, suggesting sensibly that the availability of larger financial markets does not support MFIs' model conversion.

Precisely, the probability of an MFI transforming declines with the increase in age, with older MFIs more likely to follow the NGO model than younger MFIs. In contrast, bigger MFIs have a higher likelihood of transforming to the commercial model, while smaller MFIs retain the NGO status. Also, MFIs in common law countries have a higher chance of changing than civil law countries. However, MFIs located in countries with other legal traditions (that is, not civil or common law traditions - see Appendix 6) that have the highest probability of going commercial. Stock market development relates negatively with the likelihood of transformation of MFIs, likely because people in countries with well developed financial markets are more likely served by the mainstream financial system, relying less on microfinance. However, only stock market development is a significant driver of the financial development indicators, with private credit having a negative but insignificant effect of transformation. As expected, governance/ institutional quality positively relates to the chance of a conversion. Regionally, the sample data has only NGOs for North Africa, reflecting religious aversion to for-profit microfinance operating in majority Muslim countries. Importantly, there is a potent time trend towards the commercialisation of MFIs which points to a triumph of the commercial approach to microfinance over the welfare model. GDP growth rate shows mixed results but is insignificant. In the next section, we describe the methodology applied in the study.

\hypertarget{method}{%
\section{Method}\label{method}}

The article uses three empirical estimation approaches, the binary logit and probit models and the multinomial logit model, given that our dependent variable is discrete and error terms may not be normally distributed \autocite{cramer2002origins}. For the logit and probit models, MFIs following the NGO model take a code of zero and NBFIs, credit unions/ cooperatives, and rural banks forms taking a code of one. NGOs still take a zero-code for the multinomial logit model, with commercial banks, NBFIs, credit unions, and Rural Banks taking codes of one, two, three, and four, respectively. The multinomial logit model will help uncover factors that drive the choice of a particular legal form by NGOs adopting the commercial model. NGOs converting to the commercial model can turn to commercial banks, credit unions, NBFIs, or rural banks. The following section lays out and describes the models, variables, and data sources.

\hypertarget{the-model-variables-description-and-data-sources}{%
\subsection{The Model, Variables Description and Data Sources}\label{the-model-variables-description-and-data-sources}}

We use the model below to run both logit and probit regressions on a panel dataset of 705 MFIs in Africa. Assuming the error term \(\epsilon\) follows a logistic distribution \autocite{czepiel2002maximum}, we have,

\begin{equation}
y_{it} = log(\frac{p_{it}}{(1-p_{it})} ) = \alpha + x_{it} + \varepsilon_{it}
\end{equation}

where,

\begin{equation}
p_{it}  =  \frac{1}{1 +  e^{- z_{it} } } 
\end{equation}

and

\begin{equation}
1 + p_{it}  =  \frac{1}{1 +  e^{z_{it} } }, for z_{it} = f(x_{it})
\end{equation}

In the model, \(y_{it}\) is the current legal status of the MFI, the dependent variable which is a dummy with zero, representing NGOs as the base outcome. The other legal forms of MFIs take a code of one. The symbol \(x_{it}\) represents a vector of independent variables: age, size, capital market development, legal tradition, GDP growth rate, and institutional quality. Additionally, we include year dummies to cater for the trends towards commercialisation.

The multinomial logit model extends the binary logit model to more than two unordered levels (discrete choices). The data at hand meets the requirements for running a multinomial logit model as the dependent variable (the legal status of each MFI) has one outcome for each case. Also, the independent variables do not faultlessly predict the dependent variable \autocite{petrucci2009primer}. Suppose we have a dependent variable \(y\) consisting of \(K\) choices for \(K>=2\). Further, let the independent variables be \(x_1, x_2,………. x_n\), we can specify the multinomial logit model as follows.

\begin{equation}
log(\frac{prob(k/X)}{prob(K/X)}) =   \beta_{0}^{k}  x_{0}  + \beta_{1}^{k}  x_{1} +  \cdots + \beta_{p}^{k}  x_{p}, for k = 1 \cdots K-1
\end{equation}

\(y\) is the dependent variable, in this case, one of NGO, commercial bank, NBFI, credit union or rural bank and \(x\) is a vector of independent variables.

If \(K>2\), then we have a multinomial logit with \(K-1\) set of equations. Where \(K=2\), the model is the binary logit model denoted in equation one (1) where we have one equation. Note that we have arbitrarily assigned the last category (K) as the reference in this case. Any other group can serve as a reference and hence not be part of the equations set.

One of the significant drawbacks of the multinomial logit is the violation of the assumption of the independence of irrelevant alternatives (IIA). \textcite{cheng2007testing} ) illustrate this assumption using the blue bus- red bus example. If the choice between car transport and a red bus, and given that the probability of choosing a bus is \(0.8\), and \(0.2\) for a car, then the bus's odds over car transport is 4. Suppose we introduce a third alternative, the blue bus. If the probabilities of choosing a red bus, blue bus, and car transport are \(0.6\), \(0.25\), \(0.15\), respectively, the assumption holds since the odds of selecting a red bus over a car are still \(4\). If the odds are different from \(4\), then the model violates the IIA, and the multinomial model is not fit for the data. In our case, we plausibly see the assumption holding because the legal status of an individual MFI is independent of the legal status of other MFIs. Table 2.1 (next page) describes the variables in detail.

\begin{table}

\caption{\label{tab:unnamed-chunk-12}Description of Variables}
\centering
\fontsize{9}{11}\selectfont
\begin{tabu} to \linewidth {>{\raggedright}X}
\toprule
Variable\_Description\\
\midrule
1. Current Legal Status (Dependent Variable): This is the dependent variable. For logit and probit models, we create a dummy variable with the MFIs following the NGO Model getting a code of zero, and one in the case of non-bank financial institutions (NBFIs), rural banks, and credit unions/ cooperatives. We assign codes of zero to four for the multinomial logit model for NGOs, Banks, NBFIs, Cooperatives and Rural Banks, respectively. The data are available from the Microfinance Information Exchange, MIX (See source in note 1 below).\\
\\
2.  Age: The period in which the MFI has been in operation. MFIs fall into one of three groups: new (1-4 years), young (4-8 years), and mature (over 8 years). The data are available from MIX.\\
\\
3. Legal Tradition (Legal): The indicator is a dummy variable with common law countries coded 0, civil law countries 1, and 2 otherwise as per the classification by Oto-Peralías and Romero-Ávila (2014). Appendix 6 shows the classification of countries into respective legal traditions\\
\addlinespace
\\
4. Size (Assets): We proxy the size of MFI with the natural logarithm of total assets, again using MIX data.\\
\\
5. Governance/ Institutional Quality (KKM): We take the first principal component of the WGI developed by Daniel Kaufmann, Aart Kraay and Massimo Mastruzzi (KKM) that is available on the World Bank's Worldwide Governance Indicators, WGI (See source in note 3 below).\\
\\
\addlinespace
6. Private Credit to GDP (pcrdbgdp): We capture the total amount of credit advanced to the private sector by financial intermediaries as a proxy for capital markets development concerning the banking sector following Ito and Kawai (2018). The data source is the Global Financial Development Database (GFDD) of the World Bank (See note 4).\\
\\
7. Stock market capitalisation to GDP (stmktcap): We capture the  extent of stock market development using the ratio of stock market capitalisation to GDP to proxy the extent to which firms can raise equity capital. Although Africa's equity markets are thin, some relatively large stock markets like South Africa, Egypt, Nigeria, Kenya, and Ghana exist. The data are from the GFDD.\\
\\
8. GDP annual growth rate (gdp\_growth\_annual): This is the year on year growth in output adjusted for inflation and sourced from the World Development Indicators (WDI) (See note 2).\\
\addlinespace
\\
9. Education (EDUC): The indicator is a ratio of the gross enrolment in secondary school to the gross primary school enrolment as defined in the literature (Allen et al., 2013, 2014). The data are from the WDI.\\
\\
\bottomrule
\multicolumn{1}{l}{\rule{0pt}{1em}Source: Authors' construction from the literature}\\
\multicolumn{1}{l}{\rule{0pt}{1em}\textit{Notes}}\\
\multicolumn{1}{l}{\rule{0pt}{1em}\textsuperscript{1} MIX Database on www.themix.org and https://datacatalog.worldbank.org/dataset/mix-market}\\
\multicolumn{1}{l}{\rule{0pt}{1em}\textsuperscript{2} WDI on https://databank.worldbank.org/source/world-development-indicators.}\\
\multicolumn{1}{l}{\rule{0pt}{1em}\textsuperscript{3} WGI/ KKM on https://databank.worldbank.org/source/worldwide-governance-indicators.}\\
\multicolumn{1}{l}{\rule{0pt}{1em}\textsuperscript{4} GFDD on https://www.worldbank.org/en/publication/gfdr/data/global-financial-development-database}\\
\end{tabu}
\end{table}

\hypertarget{data-analysis-and-results}{%
\section{Data Analysis and Results}\label{data-analysis-and-results}}

\hypertarget{exploratory-data-analysis}{%
\subsection{Exploratory Data Analysis}\label{exploratory-data-analysis}}

In this section, we visualise the data and then describe the variables in the model.

\hypertarget{data-visualisation}{%
\subsubsection{Data visualisation}\label{data-visualisation}}

Figure 2.1 below shows the summary statistics and scatter plots for the independent numeric variables. The summaries show a high correlation between education on the one hand and private bond market capitalisation to GDP, stock market capitalisation to GDP, and private credit to GDP on the other at around 0.5. As literature shows, higher education levels coincide with greater participation of individuals in capital markets as financial inclusion levels rise \autocite{allen2013resolving,allen2014african,ito2018quantity}. The stock market capitalisation to GDP and private bond market capitalisation to GDP correlates vastly, at 0.73. Typically, debt markets mature first, followed by stock markets, and their development levels are highly correlated \autocite{levine1999stock}. Other variables that show a high correlation include private credit to GDP on the one hand and private bond market capitalisation to GDP, and stock market capitalisation to GDP at 0.32 and 0.47. Institutional quality (KKM) also correlates with education, private credit to GDP and private bond market capitalisation \autocite{yartey2008determinants}.

With this hindsight, we drop education and private bond market capitalisation from the model. Note that much of the private credit to GDP ratio's variation already reflects the stock market's GDP ratio. Also, education reflects in general capital market development as documented in the literature \autocite{allen2013resolving,allen2014african}. We also note that North Africa has only NGOs in the model, perhaps due to faith \autocite{allen2013resolving,allen2014african,hassan2018religious}. Hence, we exclude the region in the model. There could be country-specific effects that we capture using the quality of governance (KKM) \autocite{kunvcivc2014institutional} and the annual GDP growth rate \autocite{butkiewicz2006institutional}.

Figure 2.2 shows that mature MFIs dominate the dataset across all legal forms and particularly dominant among rural banks, NGOs, and cooperatives. Turning to the prevalence of MFIs by legal status, cooperatives dominate civil law countries. Simultaneously, NGOs, NBFIs, commercial banks and rural banks dominate common law countries, which researchers have documented \autocite{pashkova2016business}. For other legal traditions, NBFIs and cooperatives dominate. As noted, North Africa has only NGOs in the dataset, which could indicate the religious constraints towards interest charging financial intermediaries \autocite{hassan2018religious}. Finally, while commercial banks and NBFIs show a higher asset base, NGOs and credit unions are not large. Commercial banks assets could be more extensive due to statutory minimum capital requirement resulting from their desire to optimise economies of scale \autocite{aiyar2016does}.

Turning to Figure 2.3, country-level governance matter more for commercial MFIs- commercial banks and NBFIs, compared to NGOs and rural banks in line with the link in the literature between investment, governance, and property rights \autocite{claessens2003financial}. Commercial banks, rural banks, and NGOs dominate countries with higher stock market development levels, while NBFIs and cooperatives trail, noting that cooperatives may be relatively less inclined to obtain funding from stock markets \autocite{porter1987economic}. The result could indicate the importance of equity capital for commercial banks and NGOs, while NBFIs tend to rely more on private equity and debt. Indeed, the data shows NBFIs dominate in countries where private credit to GDP is highest, followed by cooperatives and NGOs, while commercial banks and rural banks come last. Lastly, commercial banks, NBFIs and rural banks tend to dominate countries with higher GDP growth rates. Higher GDP growth implies higher profitability that allows commercial MFIs to thrive. Low GDP growth means that the not-for-profit NGOs and member-oriented cooperatives tend to succeed as a cushion to the vulnerable in society and fill the void left by the commercial MFIs.

\newpage

\begin{landscape}

\begin{figure}
\centering
\includegraphics{_main_files/figure-latex/unnamed-chunk-13-1.pdf}
\caption{\label{fig:unnamed-chunk-13}Correlations Between Independent Variables}
\end{figure}

\begin{figure}
\centering
\includegraphics{_main_files/figure-latex/unnamed-chunk-18-1.pdf}
\caption{\label{fig:unnamed-chunk-18}Distribution and Asset Base of MFIs in Africa by Legal Status}
\end{figure}

\begin{figure}
\centering
\includegraphics{_main_files/figure-latex/unnamed-chunk-19-1.pdf}
\caption{\label{fig:unnamed-chunk-19}Governance, Capital Market Development and Legal Status of MFIs in Africa}
\end{figure}

\end{landscape}

\newpage

\hypertarget{summary-statistics}{%
\subsubsection{Summary Statistics}\label{summary-statistics}}

Categorical variables summarised in Table 2.2 have no missing values. There are 1,280 NGOs against 3502 MFIs that are either commercial banks (619), NBFIs (1318), cooperatives (1427), or rural banks (138). As noted, even with the transformation of MFIs, NGOs still form a substantial number of MFIs, with the country to country variations \autocite{d2017ngos}. While 2558 MFIs are mature, 1200 are new, and 1024 are young. The result may indicate a slowdown in the establishment of new MFIs as donations become more unreliable. 1877 MFIs are from common law countries, with Civil law countries accounting for 1849, while 1056 come from other legal traditions. It is notable, as shown in Appendix 6, that most countries in Africa are either common law (18) or civil law(19), with relatively fewer nations in the `other' legal traditions category (11) \autocite{oto2014distribution}. It is also worth noting that North Africa accounts for only 166 observations in the data against 4616 observations in the sample dataset. Table 2.3 shows the summary statistics for the numeric variables where assets, governance (KKM), and GDP growth rates account for the highest variation.

Table 2.4 and Table 2.5 shows that NGOs, NBFIs, and commercial banks dominate common law countries. In civil law countries, it is cooperatives, NGOs, and NFIs that are most prevalent. In other legal traditions, it is NBFIs and credit unions that dominate. The result could indicate that the relatively well-developed capital markets in common law countries allow private, for-profit MFIs to thrive while not displacing NGOs. It means that NGOs in common law countries mainly serve niche markets where commercial MFIs find it uneconomical to reach. The relatively weak capital markets in civil law countries mean that cooperatives are central, with NGOs playing a significant role. Commercial MFIs like commercial banks and cooperatives' presence is low due to capital constraints. Turning to age in Table 6, most of the mature MFIs in the sample data are cooperatives, NGOs, and commercial banks in that order, while most of the new MFIs are NBFIs, cooperatives and commercial banks respectively. The results could indicate the increasing acceptance of the commercial model with newer MFIs going commercial. Table 2.7 shows the correlation between age and size, with larger MFIs more likely to be older.

\begin{table}

\caption{\label{tab:unnamed-chunk-20}Summary statistics for categorical variables}
\centering
\fontsize{9}{11}\selectfont
\begin{tabu} to \linewidth {>{\raggedright}X>{\raggedright}X}
\toprule
Variable & Counts\\
\midrule
Current legal status dummy & Others: 3502, NGO: 1280\\
Current legal status & Cooperative: 1427, NBFI: 1318, NGO: 1280, Bank: 619\\
Age & Mature: 2558, New: 1200, Young: 1024\\
Legal\_tradition & Common: 1877, Civil: 1849, Others: 1056\\
Region & Sub-Saharan Africa: 4616, North Africa: 166\\
\bottomrule
\multicolumn{2}{l}{\rule{0pt}{1em}Source: Authors' construction from MIX data}\\
\end{tabu}
\end{table}

\begin{table}

\caption{\label{tab:unnamed-chunk-21}Summary statistics for numeric variables}
\centering
\fontsize{9}{11}\selectfont
\begin{tabu} to \linewidth {>{\raggedright}X>{\raggedleft}X>{\raggedleft}X>{\raggedleft}X>{\raggedleft}X>{\raggedleft}X>{\raggedleft}X>{\raggedleft}X>{\raggedleft}X}
\toprule
Variable & N & Mean & SD & Min & Q1 & Median & Q3 & Max\\
\midrule
assets & 4782 & 14.946 & 2.262 & 0.693 & 13.540 & 14.858 & 16.416 & 22.98\\
kkm & 4782 & 0.003 & 2.006 & -5.233 & -1.304 & -0.114 & 1.628 & 7.37\\
education & 4782 & 0.387 & 0.144 & 0.075 & 0.273 & 0.386 & 0.487 & 1.05\\
pcrdbgdp & 4782 & 2.719 & 0.685 & 0.298 & 2.386 & 2.758 & 3.052 & 6.88\\
stmktcap & 4782 & 1.141 & 1.473 & 0.000 & 0.000 & 0.000 & 2.428 & 5.80\\
\addlinespace
prbonds & 4782 & 0.632 & 1.093 & 0.000 & 0.000 & 0.000 & 1.130 & 4.36\\
gdp\_growth\_annual & 4782 & 5.310 & 3.590 & -46.082 & 4.000 & 5.420 & 6.723 & 33.63\\
\bottomrule
\multicolumn{9}{l}{\rule{0pt}{1em}Source: Authors' construction from MIX data}\\
\end{tabu}
\end{table}

\begin{table}

\caption{\label{tab:unnamed-chunk-22}Legal Status of MFIs in Africa Disaggregated by Legal Tradition}
\centering
\fontsize{9}{11}\selectfont
\begin{tabu} to \linewidth {>{\raggedright}X>{\raggedleft}X>{\raggedleft}X>{\raggedleft}X>{\raggedleft}X>{\raggedleft}X}
\toprule
  & NGO & Bank & NBFI & Coop & Rural Bank\\
\midrule
Common & 0.323 & 0.246 & 0.310 & 0.048 & 0.073\\
Civil & 0.304 & 0.014 & 0.158 & 0.524 & 0.001\\
Other & 0.106 & 0.126 & 0.420 & 0.348 & 0.000\\
\bottomrule
\multicolumn{6}{l}{\rule{0pt}{1em}Source: Authors' construction from MIX data}\\
\multicolumn{6}{l}{\rule{0pt}{1em}\textit{Note: }}\\
\multicolumn{6}{l}{\rule{0pt}{1em}\textsuperscript{1} Horizontals total to 100}\\
\end{tabu}
\end{table}

\newpage

\hypertarget{results-of-the-regression-model}{%
\subsection{Results of the Regression Model}\label{results-of-the-regression-model}}

Table 2.11 shows the output from both the logit and probit regression analysis. We ran the analysis using the entire dataset, then filter MFIs with three or more years and five or more years of data and rerun the regression (as indicated in the bottom panel of the regression tables). We also include a time dummy, although these results are not in the regression tables. However, there is also a robust positive time trend towards commercialisation of MFIs, indicating relatively fewer NGO type MFIs over time as more MFIs opt for the commercial model. The trend is indicative of the gains that the sustainability school has made. However, NGOs still form a substantial proportion of MFIs. In this section, we examine each variable and its relative contribution to the transformation of MFIs. Note that we use the logit model in column 1 of Table 2.9 to interpret the discussion results. However, the interpretation is also applicable to the other models presented in the Table.

\begin{table}

\caption{\label{tab:unnamed-chunk-23}Breakdown of Legal Status of MFIs by Legal Traditions, Percent}
\centering
\fontsize{9}{11}\selectfont
\begin{tabu} to \linewidth {>{\raggedright}X>{\raggedleft}X>{\raggedleft}X>{\raggedleft}X>{\raggedleft}X>{\raggedleft}X}
\toprule
legal\_tradition & NGO & Bank & NBFI & Coop & Rural Bank\\
\midrule
Common & 47.34 & 74.47 & 44.2 & 6.38 & 99.275\\
Civil & 43.91 & 4.04 & 22.2 & 67.91 & 0.725\\
Other & 8.75 & 21.49 & 33.7 & 25.72 & -\\
\bottomrule
\multicolumn{6}{l}{\rule{0pt}{1em}Source: Authors' construction from MIX data}\\
\multicolumn{6}{l}{\rule{0pt}{1em}\textit{Note: }}\\
\multicolumn{6}{l}{\rule{0pt}{1em}\textsuperscript{1} Verticals total to 100}\\
\end{tabu}
\end{table}

\hypertarget{age}{%
\subsubsection{Age}\label{age}}

New MFIs are more likely to adopt a commercial model than either young or mature ones, given the coefficients' negative sign. The observation may reflect the increasing acceptance of the financial systems approach to microfinance, making it harder for new entrants to attract donor funding \autocite{d2017ngos}. We postulate that older MFIs, being the pioneers and hence already well-acquainted with donors, find it easier to raise funds through donations and elicit state subsidies \autocite{d2013unsubsidized,mia2017mission}. Established MFIs have a historical relationship with donors. They are likely to attract funds, more so from donors that favour the welfare approach to microfinance, to reach out to the financially excluded first before pursuing profits. The results are in line with those in Table 6, which shows the distribution of MFI legal status disaggregated by age. While only 18\% of NGOs are new, 27.3\% are young, while about 30\% are mature, an upward trend. By comparison, the other legal status either decline or are relatively constant.

\begin{table}

\caption{\label{tab:unnamed-chunk-24}Legal Status of MFIs in Africa Disaggregated by Age}
\centering
\fontsize{9}{11}\selectfont
\begin{tabu} to \linewidth {>{\raggedright}X>{\raggedleft}X>{\raggedleft}X>{\raggedleft}X>{\raggedleft}X>{\raggedleft}X}
\toprule
  & NGO & Bank & NBFI & Coop & Rural Bank\\
\midrule
New & 0.188 & 0.193 & 0.297 & 0.295 & 0.027\\
Young & 0.273 & 0.099 & 0.334 & 0.275 & 0.019\\
Mature & 0.303 & 0.112 & 0.242 & 0.309 & 0.034\\
\bottomrule
\multicolumn{6}{l}{\rule{0pt}{1em}Source: Authors' construction from MIX data}\\
\end{tabu}
\end{table}

The coefficients of the logit model show that young MFIs (4-8 years) are 0.474 as likely to be in the commercial category than the new MFIs (0-4 years), ceteris paribus. This result means that new MFIs are likely to be commercial while older MFIs are most likely NGOs. \footnote{Relative risk ratios allow for easier interpretation of the logit models. To compute the ratio, we exponentiate the coefficients. For instance, the coefficient for young MFIs is -0.747, so the relative risk ratio is \(e^{-0.747}\), which gives 0.474. In other words, the odds of having the commercial model of microfinance is 1 - 0.474 = 0.526 in the sample dataset.}, we find that keeping all the other variables constant, young MFIs roughly one half less likely to be commercial than new MFIs. Likewise, mature MFIs are a third as likely to be commercial compared to new MFIs \footnote{Again, we exponentiate the coefficient -1.2 to get \(e^{-1.2} = 0.301\)}. The finding is also consistent with the intense time effect towards commercialisation which points to the increased acceptance of the commercial model of MFIs. The probit model also shows similar results.

Having started their operations before the neo-liberal tradition took hold, older MFIs have created goodwill with donors that enable them to solicit donations and subsidies easily. Mature MFIs could also have evolved business models to be financially sustainable without converting to the commercial model. For instance, mature MFIs tend to have a broader asset base meaning they have a more diverse customer base (see Table 6). Besides, they may have emphasised the social mission in their vision, mission, and organisational cultures to such an extent that both the MFI and the donor community find it hard to pull back \autocite{ramus2017,berbegal2019impact}.

\begin{table}

\caption{\label{tab:unnamed-chunk-25}Size (Assets) of MFIs in Africa Disaggregated by Age}
\centering
\fontsize{9}{11}\selectfont
\begin{tabu} to \linewidth {>{\raggedright}X>{\raggedleft}X>{\raggedleft}X>{\raggedleft}X>{\raggedleft}X}
\toprule
Age & Min\_size & Mean\_size & Median\_size & Max\_size\\
\midrule
New & 0.693 & 13.5 & 13.5 & 23.0\\
Young & 5.796 & 14.5 & 14.4 & 19.8\\
Mature & 6.361 & 15.8 & 15.7 & 22.9\\
\bottomrule
\multicolumn{5}{l}{\rule{0pt}{1em}Source: Authors' construction from MIX data}\\
\end{tabu}
\end{table}

Younger MFIs, on the other hand, cropped up when the paradigm shift to the institutional approach was taking shape. It means, therefore, that donors were reluctant to extend funds to such organisations. Hence, the MFIs had to supplement the little donor funding and government subsidies by raising funds from the capital markets. The thinking is consistent with the literature that shows the extent to which donor funding is volatile and especially sensitive to geopolitical realignments \autocite{garmaise2013cheap,d2017aid} and business cycles \autocite{wagner2013vulnerability}.

\hypertarget{legal-tradition}{%
\subsubsection{Legal Tradition}\label{legal-tradition}}

As noted, we have grouped countries in the sample data into their respective legal traditions following \textcite{oto2014distribution}. MFIs in civil law countries have a lower chance of transformation compared to those from common law countries \footnote{Appendix 6 shows a breakdown of the legal traditions in Africa}. However, MFIs in countries under ``other'' legal traditions have the highest likelihood of adopting the commercial model. The result is in line with the literature that shows the law's central place in finance \autocite{la2013law}. Specifically, holding all other variables constant, MFIs in civil law countries are 0.656 (\(e^{-0.421}\)) as likely as those in common law countries to follow the commercial model, meaning that most of them remain NGOs, following the not-for-profit, welfare approach. The odds of MFIs in civil law countries being commercial is 0.344 (\(1 - 0.656\)). On the contrary, MFIs in countries that follow other legal traditions are twice (\(e^{0.744}\)) as likely to be commercial instead of NGOs, with the odds being 1.1 (\(2.1 - 1.1\)).

Table 2.4 show the breakdown of MFI legal forms by the country's legal tradition. The table shows the dominance of NGOs (32.3\%), commercial banks (24.6\%) and NBFIs (31\%) in common law countries. Cooperatives (52.4\%), NGOs (30.4\%), and NBFIs (15.8\%) dominate civil law countries, while NBFIs (42\%), cooperatives (34.8\%), and banks (12.6\%) are more common in other legal traditions. There are very few banks (1.4\%) and NBFIs (15.8\%) in civil law countries. Given the low levels of financial development in many civil law countries, there is a commercially viable gap for profit-oriented MFIs to fill. The gap raises the odds of MFI transformation happening more frequently in civil law countries than in countries following common law and other legal traditions.

On the contrary, the prevalence of commercial MFIs in common law countries could hold due to the higher levels of capital market development, reflecting the relative ease of acquiring funds \autocite{schnyder2018twenty}. The relative ease of acquiring capital from stock and bond markets could make it less likely that NGOs would prevail, making the commercial model more likely. The substantial number of NGOs in common law countries would fill the gap left by commercial MFIs due to the infeasibility of serving some clients, for instance, due to geographic remoteness or extreme poverty. There is little literature in law and finance that examines other legal traditions, such as Portuguese/ Spanish traditions as in Mozambique, Angola, Equatorial Guinea, and countries with unique traditions like Ethiopia that was never a colony. The results for the ``other'' legal practices in Africa's setting warrant further analysis. Table 2.5 confirms these results, showing, for instance, that 47.34\% of NGOs are in common law countries, 43.91 in civil law countries and the rest in other legal traditions. Common law countries have the bulk of banks (74.47\%) and NBFIs (44.2\%). Rural banks are almost entirely a common law phenomenon.

\hypertarget{size-log-of-total-assets}{%
\subsubsection{Size (Log of Total Assets)}\label{size-log-of-total-assets}}

All else being constant, larger MFIs in terms of assets are more likely to adopt the commercial model than the relatively smaller MFIs with fewer assets. Perhaps large MFIs can sustain their operations independent of donations and subsidies \autocite{d2013unsubsidized}. They have a higher capacity to attract money from the capital markets, given their strong assets base (as collateral) and track record. Everything else remaining the same, a unit increase in the asset base of an MFI raises the probability of transformation by 1.27 (\(e^{0.240}\)), with the odds of being in the commercial model being 0.27 (\(1.27 - 1\)).

Abundant literature in Africa and beyond, such as \textcite{gwatidzo2009corporate} and \textcite{kodongo2015capital}, show that the size of an MFI is an essential determinant of firms' capital structures, the mix of long term sources of funds. In this case, larger firms could easily avail collateral for funds and tend to be more open in providing information that financial intermediaries require to assess creditworthiness. On the other hand, small firms are informationally opaque \autocite{beck2014sme,kersten2017small}. Small firms, for example, may not afford to generate audited financial reports. Moreover, larger firms are likely to be mature with a solid business record, creating goodwill among the providers of funds \autocite{beck2008finance}. The size of MFIs could also reflect the extent of property rights protection that is harder to enforce in countries with weak governance \autocite{johnson2002property,claessens2003financial}. A fragile institutional environment makes it difficult for firms to grow due, in part, to poor access to capital and the high costs of formalising business \autocite{hansen2004reconsidering}. Next, we examine country-level governance / institutional quality.

\hypertarget{country-level-governance-institutional-quality-kkm}{%
\subsubsection{Country Level Governance/ Institutional Quality (KKM)}\label{country-level-governance-institutional-quality-kkm}}

We capture governance or institutional quality by taking the first principal component of the KKM Worldwide Governance Indicators (WGI) indices \autocite{kraay2010worldwide}. Governance (KKM Index) positively relates to the odds of transforming. All else remaining the same, when the governance index in a country rises by one unit, MFIs in the given country are 1.1 (\(e^{0.095}\)) times more likely to be in the commercial model than NGOs, meaning that the odds rise by \(0.1\) (\(1.1 - 1\)). The results probably hold due to the importance of property rights in raising confidence among private investors who finance the operations of transformed MFIs (Allen et al., 2013, 2014). Where governance and property rights are weak, most MFIs would likely remain NGOs for longer as investors are reluctant to finance private ventures in line with \textcite{johnson2002property} and \textcite{claessens2003financial}.

Literature shows a positive link between country-level institutional quality and the establishment, growth of private firms \autocite{sobel2008testing}. As captured in the KKM index, institutional quality captures factors that relate directly to the ease of doing business, contract enforcement effectiveness, and the extent of property rights. Where institutional quality is high, we expect private firms to take root, mainly commercial MFIs, primarily commercial banks and NBFIs. On the other hand, where institutional quality is low, NGOs and not-for-profit oriented MFIs may be more prevalent \autocite{kuzey2021link}. Indeed, The results on governance could partly explain the prevalence of NGOs in North Africa in the sample dataset, together with religion. Table 2.8 shows that North Africa fares poorly compared to Sub-Saharan Africa in most governance metrics. Interestingly, Table 2.9 shows that commercial banks and NBFIs are more prevalent in countries with higher institutional quality.

\begin{table}

\caption{\label{tab:unnamed-chunk-26}Summary Statistics on Governance in Africa}
\centering
\fontsize{9}{11}\selectfont
\begin{tabu} to \linewidth {>{\raggedright}X>{\raggedleft}X>{\raggedleft}X>{\raggedleft}X>{\raggedleft}X}
\toprule
Region & Min & Mean & Median & Max\\
\midrule
North Africa & -3.01 & -1.61 & -1.506 & -1.01\\
Sub-Saharan Africa & -5.23 & 0.06 & -0.114 & 7.37\\
\bottomrule
\multicolumn{5}{l}{\rule{0pt}{1em}Source: Authors' construction from MIX data}\\
\end{tabu}
\end{table}

\begin{table}

\caption{\label{tab:unnamed-chunk-27}Institutional Quality (KKM) and Legal Status of MFIs in Africa}
\centering
\fontsize{9}{11}\selectfont
\begin{tabu} to \linewidth {>{\raggedright}X>{\raggedleft}X>{\raggedleft}X>{\raggedleft}X>{\raggedleft}X}
\toprule
currentlegalstatus & Min & Mean & Median & Max\\
\midrule
NGO & -5.23 & -0.494 & -0.758 & 5.68\\
Bank & -5.23 & 0.929 & 1.208 & 7.37\\
NBFI & -5.17 & 0.510 & 0.350 & 6.74\\
Coop & -3.36 & -0.166 & -0.270 & 4.92\\
Rural Bank & -3.31 & -2.652 & -3.183 & 2.38\\
\bottomrule
\multicolumn{5}{l}{\rule{0pt}{1em}Source: Authors' construction from MIX data}\\
\end{tabu}
\end{table}

\hypertarget{private-credit-to-gdp}{%
\subsubsection{Private Credit to GDP}\label{private-credit-to-gdp}}

The private credit to GDP inversely relates to the prevalence of commercial models of microfinance, with the relationship primarily insignificant. In this case, private credit refers to an aspect of capital markets development, mainly in the banking sector. It is puzzling that a well-developed credit market does not appear to enhance the prevalence of for-profit MFI models. The results could suggest a weak linkage between MFIs and private capital markets, more so credit from financial intermediaries. Indeed, MFIs exist to serve markets where mainstream intermediaries neglect, meaning the low presence of mainstream banks means a higher prevalence of MFIs to fill the void \autocite{de2007economics}. Where significant, a unit increase in private credit to GDP corresponds to a 0.894 times lower chance that an MFI will be commercial, profit-oriented (\(e^{-0.112}\)), which corresponds to an odds of -0.114.

As noted, MFIs, especially the NGO type, exist to fill a financing gap that results from the failure of credit markets to reach the financially excluded, that is, the poor, rural dwellers and women savers and borrowers. If mainstream credit markets are functional, then there is no case for the existence of commercial MFIs, because mainstream banks would fill the gap adequately, leaving no business case for commercial MFIs to exist. However, as no credit market is fully efficient, then NGOs would exist to serve niche markets where financial sustainability is unattainable due to a combination of high costs and low revenues \autocite{de2007economics}. On the other hand, if capital markets are not well developed, there exists a market gap that commercial MFIs could exploit to make a profit \autocite{d2013unsubsidized,armendariz2013subsidy}.

\hypertarget{stock-market-capitalisation-to-gdp}{%
\subsubsection{Stock market capitalisation to GDP}\label{stock-market-capitalisation-to-gdp}}

Stock market capitalisation to GDP has a significant negative relationship with the prevalence of commercial MFIs. Precisely, a unit increase of stock market capitalisation corresponds to a 0.721 odds of an MFI adopting the for-profit model. Like private credit to GDP, stock market capitalisation to GDP proxies the level of stock market development, an essential source of long-term finance for corporations, presumably including MFIs. The equity could be from the public or private equity market, which the stock market would proxy reasonably well. In the case of MFIs in the sample dataset, the capital to assets ratio, the ratio of equity capital to assets, shows the importance of equity in financing microfinance. Notably, equity is of greater importance to NGOs than commercial MFIs (see Table 2.10), with NBFIs and commercial banks following in that order. If NGOs are the dominant participants in equity markets, there are lower chances that a well-developed stock market corresponds to more commercial MFIs. The same argument follows that if stock markets are well developed, private and, public credit markets are also well-developed \autocite{schnyder2018twenty}. With well-developed capital markets, financial exclusion incidences are fewer, leaving no vacuum that commercial MFIs could profitably exploit. In such instances, NGOs following the not-for-profit welfare model best serve the few cases of financial exclusion.

\begin{table}

\caption{\label{tab:unnamed-chunk-28}Capital Asset Ratio by MFI Legal Status in Africa}
\centering
\fontsize{9}{11}\selectfont
\begin{tabu} to \linewidth {>{\raggedright}X>{\raggedleft}X>{\raggedleft}X}
\toprule
Legal Status & Mean & Median\\
\midrule
Bank & 0.306 & 0.239\\
Credit Union/ Cooperative & 0.196 & 0.208\\
NBFI & 0.388 & 0.324\\
NGO & 0.418 & 0.381\\
Rural Bank & 0.176 & 0.137\\
\bottomrule
\multicolumn{3}{l}{\rule{0pt}{1em}Source: Authors' construction from MIX data}\\
\end{tabu}
\end{table}

\hypertarget{gdp-annual-growth-rate}{%
\subsubsection{GDP Annual Growth Rate}\label{gdp-annual-growth-rate}}

The GDP growth rate is not a significant driver of transformation. Where significant, some of the coefficients are positive, while others are negative. The implication is that the macro-environment may not be a substantial driver of MFIs decisions. Most MFIs in developing countries serve the informal sector's financially excluded population with low linkage to the formal economy \autocite{ghosh2013microfinance}.

\hypertarget{time-effects}{%
\subsubsection{Time Effects}\label{time-effects}}

The regression model also incorporates year effects, although not reported in the regression results tables. There is a strong trend towards commercialisation, with the commercial model increasingly dominating Africa's MFI landscape. All the year dummies are significant in all the models, the lowest level of significance being 10\%. Numerous researchers have noted the trend towards the commercial model. Hence, the abundant research seeks to examine the potential effects of the transformation on financial inclusion targets- the financially excluded \autocite{d2017ngos}. Some scholars claim that the trend may harm financial inclusion. \autocite{meagher2006microfinance,hartarska2007regulated}. Others hold the opposing view \autocite{duvendack2015mis}. It appears that the financial sustainability school that seeks commercialisation has the upper hand in Africa, at least in the last two decades.

\hypertarget{regional-divide}{%
\subsubsection{Regional Divide}\label{regional-divide}}

It is notable that for the sample data, all the MFIs operating in North Africa are NGOs, while the rest of Africa has a MIX of all forms of MFIs \footnote{Countries in North Africa in the sample data are Morocco and Tunisia}. Religion may be at play in this case, where interest-based for-profit lending is incompatible with the Muslim faith that dominates North Africa \autocite{hassan2018religious}. Also, as noted, North Africa fares worse in governance (KKM) than sub-Saharan Africa, leading to a flawed property rights regime that discourages private investment \autocite{johnson2002property,claessens2003financial}.

\newpage

\begin{landscape}

\begin{table}[!htbp] \centering 
  \caption{Logit and Probit Models (Standard Errors in Brackets)} 
  \label{} 
\footnotesize 
\begin{tabular}{@{\extracolsep{5pt}}lcccccccc} 
\\[-1.8ex]\hline 
\hline \\[-1.8ex] 
 & \multicolumn{8}{c}{\textit{Dependent variable:}} \\ 
\cline{2-9} 
\\[-1.8ex] & \multicolumn{8}{c}{Dummy: Current Legal Status} \\ 
\\[-1.8ex] & \textit{logistic} & \textit{probit} & \textit{logistic} & \textit{probit} & \textit{logistic} & \textit{probit} & \textit{logistic} & \textit{probit} \\ 
\\[-1.8ex] & (1) & (2) & (3) & (4) & (5) & (6) & (7) & (8)\\ 
\hline \\[-1.8ex] 
 ageYoung & $-$0.747$^{***}$ & $-$0.421$^{***}$ & $-$0.418$^{***}$ & $-$0.240$^{***}$ & $-$0.452$^{***}$ & $-$0.264$^{***}$ & $-$0.766$^{***}$ & $-$0.431$^{***}$ \\ 
  & (0.114) & (0.065) & (0.132) & (0.076) & (0.157) & (0.091) & (0.112) & (0.064) \\ 
  & & & & & & & & \\ 
 ageMature & $-$1.200$^{***}$ & $-$0.691$^{***}$ & $-$0.898$^{***}$ & $-$0.522$^{***}$ & $-$0.944$^{***}$ & $-$0.551$^{***}$ & $-$1.150$^{***}$ & $-$0.662$^{***}$ \\ 
  & (0.106) & (0.060) & (0.122) & (0.071) & (0.145) & (0.084) & (0.104) & (0.059) \\ 
  & & & & & & & & \\ 
 legal\_traditionCivil & $-$0.421$^{***}$ & $-$0.239$^{***}$ & $-$0.515$^{***}$ & $-$0.313$^{***}$ & $-$0.545$^{***}$ & $-$0.338$^{***}$ & $-$0.518$^{***}$ & $-$0.289$^{***}$ \\ 
  & (0.117) & (0.068) & (0.127) & (0.074) & (0.140) & (0.082) & (0.114) & (0.066) \\ 
  & & & & & & & & \\ 
 legal\_traditionOther & 0.744$^{***}$ & 0.387$^{***}$ & 0.790$^{***}$ & 0.416$^{***}$ & 0.870$^{***}$ & 0.466$^{***}$ & 0.743$^{***}$ & 0.387$^{***}$ \\ 
  & (0.132) & (0.073) & (0.149) & (0.084) & (0.167) & (0.094) & (0.130) & (0.072) \\ 
  & & & & & & & & \\ 
 assets & 0.240$^{***}$ & 0.142$^{***}$ & 0.355$^{***}$ & 0.214$^{***}$ & 0.450$^{***}$ & 0.270$^{***}$ & 0.242$^{***}$ & 0.144$^{***}$ \\ 
  & (0.019) & (0.011) & (0.024) & (0.014) & (0.029) & (0.016) & (0.018) & (0.011) \\ 
  & & & & & & & & \\ 
 kkm & 0.095$^{***}$ & 0.057$^{***}$ & 0.102$^{***}$ & 0.063$^{***}$ & 0.139$^{***}$ & 0.087$^{***}$ & 0.115$^{***}$ & 0.067$^{***}$ \\ 
  & (0.019) & (0.011) & (0.023) & (0.013) & (0.025) & (0.015) & (0.019) & (0.011) \\ 
  & & & & & & & & \\ 
 pcrdbgdp & $-$0.112 & $-$0.049 & $-$0.127 & $-$0.047 & $-$0.221$^{**}$ & $-$0.097$^{*}$ & 0.055 & 0.036 \\ 
  & (0.076) & (0.042) & (0.083) & (0.047) & (0.090) & (0.051) & (0.070) & (0.039) \\ 
  & & & & & & & & \\ 
 stmktcap & $-$0.327$^{***}$ & $-$0.190$^{***}$ & $-$0.369$^{***}$ & $-$0.225$^{***}$ & $-$0.398$^{***}$ & $-$0.246$^{***}$ & $-$0.359$^{***}$ & $-$0.206$^{***}$ \\ 
  & (0.038) & (0.022) & (0.042) & (0.024) & (0.047) & (0.027) & (0.037) & (0.021) \\ 
  & & & & & & & & \\ 
 gdp\_growth\_annual & 0.016 & 0.012$^{*}$ & $-$0.004 & $-$0.0004 & $-$0.025$^{*}$ & $-$0.013 & 0.024$^{**}$ & 0.015$^{**}$ \\ 
  & (0.011) & (0.006) & (0.013) & (0.008) & (0.014) & (0.008) & (0.011) & (0.006) \\ 
  & & & & & & & & \\ 
\hline \\[-1.8ex] 
Year Effects & Yes & Yes & Yes & Yes & Yes & Yes & No & No \\ 
Deviance & 677*** & 664*** & 651*** & 648*** & 660*** & 659*** & 619*** & 607*** \\ 
Data & Full & Full & >3yrs & >3yrs & >5yrs & >5yrs & Full & Full \\ 
Observations & 4,782 & 4,782 & 3,840 & 3,840 & 3,165 & 3,165 & 4,782 & 4,782 \\ 
Log Likelihood & $-$2,439.000 & $-$2,446.000 & $-$2,004.000 & $-$2,005.000 & $-$1,633.000 & $-$1,633.000 & $-$2,469.000 & $-$2,475.000 \\ 
Akaike Inf. Crit. & 4,939.000 & 4,952.000 & 4,068.000 & 4,071.000 & 3,325.000 & 3,326.000 & 4,957.000 & 4,969.000 \\ 
\hline 
\hline \\[-1.8ex] 
\textit{Note:}  & \multicolumn{8}{l}{$^{*}$p$<$0.1; $^{**}$p$<$0.05; $^{***}$p$<$0.01} \\ 
\end{tabular} 
\end{table}

\end{landscape}

\newpage

\hypertarget{multinomial-logit-model}{%
\subsection{Multinomial Logit Model}\label{multinomial-logit-model}}

We extend the analysis to the multinomial logit model. To reiterate, the multinomial logit model will help uncover factors that drive the choice of a particular legal form by NGOs adopting the commercial model. We present the results of the model in the tables of Appendix 2 through to Appendix 5. As in the binary models, the results confirm the factors that drive the conversion of MFIs from NGOs to commercial models. We base our discussion on the results in Appendix 2, which easily generalises to the other appendices. The results show that as the MFI transitions from being new to young, it is less likely to be a bank, NBFI, credit union, or rural bank due to the coefficients' negative sign. Hence, older firms are more likely to be NGOs in line with the logit model for reasons expounded in the logit model's output.

Similarly, compared to NGOs, mature MFIs are less likely to be commercial banks, NBFIs, cooperatives or rural banks. The results align with the logit model showing that most commercial MFIs are more likely new or young, while NGOs are more likely mature. Again, start-up MFIs are more inclined to the commercial model than the established MFIs.

Relative to common law countries, MFIs in civil law countries are less likely to be commercial banks, NBFIs, and rural banks and are more likely to be NGOs. However, in civil law countries, MFIs are more likely to be cooperatives than NGOs. Likewise, relative to common law countries, MFIs in other legal traditions are less likely to be commercial banks and rural banks than NGOs and more likely to be credit unions or NBFIs than NGOs. The results illustrate the link between legal tradition and the legal status of MFIs, with NGOs dominating common law countries whilst cooperatives prevail in other legal traditions, including civil law tradition. As noted, the better-developed capital markets in common law countries leave little room for commercial MFIs to thrive. On the contrary, the under-served populace in civil law countries presents ample business opportunities for profit-oriented MFIs \autocite{d2013unsubsidized,mia2017mission}.

The results show that as an MFI size increases, the likelihood of shifting from the NGO model to the commercial model rises. However, as firms grow in size, they are less likely to adopt the cooperative model, although the relationship is not significant. MFIs mainly shift from NGOs to commercial banks, rural banks, and NBFIs, but rarely to cooperatives or rural banks. The results highlight the uniqueness of cooperatives and rural banking as microfinance models that serve niche markets. Strictly speaking, cooperatives are quasi-commercial entities. Their mode of operation differs from the other MFIs in terms of clientele and possible geographic reach, reducing size. Similarly, rural banks serve marginalised rural dwellers and are more prevalent in common law countries. Noting that most large MFIs are also mature, it follows that the edge granted by maturity also accrues to larger MFIs \autocite{beck2014sme,kersten2017small}.

Also, as country-level institutional quality rises, MFIs are more likely to be commercial banks and NBFIs, and less likely cooperatives and rural banks relative to NGOs. Cooperatives and rural banks are less sensitive to institutional quality matters courtesy of their unique markets, even more so than NGOs \autocite{sobel2008testing}. As noted earlier, the commercial, for-profit model could only thrive best in countries where institutional quality is high. However, cooperatives and rural banks in Africa serve unique markets. Rural banks primarily focused on informal rural economies that may have a weak linkage to the formal economy, making governance and institutional quality less relevant.

By way of both stock market to GDP and private credit to GDP, capital market development follows a similar pattern. As in the logit model, MFIs in countries scoring high in private credit to GDP and stock market capitalisation to GDP are less likely to go commercial and more likely to be NGOs. We have argued before that well developed financial market implies a smaller customer base for MFIs and hence the result. The result also concurs with the observation about legal tradition. Because capital markets in civil law countries are less developed, the void tends to be profitably filled by commercial MFIs.In common law countries, capital markets leave few profitable opportunities which NGOs serve \autocite{d2013unsubsidized,armendariz2013subsidy}.

Finally, high GDP growth rates increase the likelihood that an MFI will be a commercial bank, NBFI, or rural bank than NGOs but less likely to be a cooperative or rural bank. As is the case with institutional quality, the economic environment matters most for commercial MFIs that target profit. However, NGOs, cooperatives and rural banks may better serve communities undergoing adverse economic experiences \autocite{ghosh2013microfinance}. Cooperatives obtain capital from members and are obligated to attend to the members regardless of economic uncertainties. NGOs and rural banks specifically target marginalised people. Economic downturns are more likely to raise the level of exclusion and make these forms of MFIs even more relevant \autocite{schnyder2018twenty}.

\hypertarget{overall-model-fit}{%
\subsection{Overall Model Fit}\label{overall-model-fit}}

To assess the overall model fit, we generate the \texttt{confusion\ matrix} and statistics in Table 2.12 and figure 4, respectively. For this purpose, we use the models developed by using the whole dataset- Table 2.11 for the logit model and Appendix 2 for the multinomial logit model.

\hypertarget{logit-model}{%
\subsubsection{Logit Model}\label{logit-model}}

Overall, the models are highly significant (at 1\% significance levels, see Table 2.11), meaning that they explain why MFIs tend to adopt a given model better than guessing the most prevalent outcome- that every MFI in the sample is not an NGO. In the first row of Table 2.12, we see that the logit model predicted correctly that 304 NGOs were NGOs. The model also accurately predicts that an MFI belongs to other legal forms (Bank, NBFI, Coop, Rural Bank) when they belong to these forms. However, the model fails by predicting 976 cases of MFIs as other legal forms when they are NGOs. Similarly, the model wrongly classifies 126 cases of MFIs of other legal forms as NGOs.

Overall, the logit model accurately predicts the legal status of an MFI 77\% of the time \footnote{The accuracy is computed as (304 + 3376)/(304 + 976 + 126 + 3376) = 0.77}. The prediction is within the confidence interval captured by the entries \texttt{AccuracyLower} and \texttt{AccuracyUpper}. If we were to guess that every MFI in the dataset follows the commercial model (that is, not an NGO), we would be accurate 73.2\% of the time (referred to as the No Information Rate (NIR) in the \texttt{confusion\ matrix}) \autocite{cavalin2018confusion}. The p-value shows that the accuracy is not due to chance with over 99\% confidence, meaning that the accuracy is significantly greater than the NIR \autocite{kleinbaum2002logistic}.

The model has low \texttt{sensitivity}, though, at 23.75\%. In this case, sensitivity is a model's ability to accurately predict that an MFI is an NGO when it is an NGO \autocite{marom2010using}. The low \texttt{sensitivity} could, in part, be due to the low \texttt{prevalence} of NGOs in the dataset (at 26.77\%) relative to the commercial forms of MFIs (73.23\%). However, the model has very high \texttt{specificity} at 96.4\%. Specificity is the capacity of the model to predict that an MFI follows the commercial model (NOT an NGO) when it is following the commercial model (is NOT an NGO) \autocite{zeng2020confusion}. Hence, it appears that commercial MFIs have distinct characters that easily allow the model to distinguish them from NGOs. The other metric of interest is the \texttt{balanced\ accuracy} that averages \texttt{sensitivity} and \texttt{specificity} at 60\% \autocite{gorzalczany2016multi}. Overall, the model does better than guessing that every MFI in the sample dataset follows the commercial model (or is NOT an NGO) \autocite{hosmer2013applied}.

Figure 2.4 shows a visualization of the confusion matrix and the receiver operating characteristics (ROC) curve. Again these visualizations show that the model does well in \texttt{specificity}. The ROC curve plots \texttt{sensitivity} against \(1 - specificity\). For an ideal model, the ROC curve would pass through the point (0,1), which is the top left corner of the curve. A model with a ROC curve being a straight line passing through the origin (the dotted line) does no better than guessing. In this case, the ROC curve shows that the model has significant explanatory power. A related metric the area under the ROC curve called the \texttt{area\ under\ the\ curve\ (AUC)} shown in Table 2.13. AUC is the area between the ROC curve and the x-axis, with higher values of AUC corresponding to a better model. An AUC of 0.5 connotes a model that discriminates the units of analysis no better than guessing and is equivalent to the straight line passing through the origin \autocite{mandrekar2010receiver}. The AUC, in this case, is 0.726 out of a possible maximum of one.

\begin{table}

\caption{\label{tab:unnamed-chunk-33}Confusion Matrix and Statistics for the Logit Model}
\centering
\fontsize{9}{11}\selectfont
\begin{tabu} to \linewidth {>{\raggedright}X>{\raggedright}X>{\raggedleft}X}
\toprule
.metric & .estimator & .estimate\\
\midrule
accuracy & binary & 0.770\\
kap & binary & 0.255\\
sens & binary & 0.238\\
spec & binary & 0.964\\
ppv & binary & 0.707\\
\addlinespace
npv & binary & 0.776\\
mcc & binary & 0.312\\
j\_index & binary & 0.202\\
bal\_accuracy & binary & 0.601\\
detection\_prevalence & binary & 0.090\\
\addlinespace
precision & binary & 0.707\\
recall & binary & 0.238\\
f\_meas & binary & 0.356\\
\bottomrule
\multicolumn{3}{l}{\rule{0pt}{1em}Source: Authors' construction}\\
\multicolumn{3}{l}{\rule{0pt}{1em}\textit{Notes: }}\\
\multicolumn{3}{l}{\rule{0pt}{1em}\textsuperscript{1} Accuracy > NoInformationRate is significant at 1\% confidence level, p = 0.0000}\\
\end{tabu}
\end{table}

\newpage
\begin{landscape}

\begin{table}

\caption{\label{tab:unnamed-chunk-34}ROC Area Under Curve (ROC AUC)}
\centering
\fontsize{9}{11}\selectfont
\begin{tabu} to \linewidth {>{\raggedright}X>{\raggedright}X>{\raggedleft}X}
\toprule
.metric & .estimator & .estimate\\
\midrule
roc\_auc & binary & 0.726\\
\bottomrule
\multicolumn{3}{l}{\rule{0pt}{1em}Source: Authors' construction}\\
\end{tabu}
\end{table}

\begin{figure}
\centering
\includegraphics{_main_files/figure-latex/unnamed-chunk-34-1.pdf}
\caption{\label{fig:unnamed-chunk-34}Confusion Matrix (left) and ROC Curve}
\end{figure}

\newpage
\end{landscape}

\newpage

\hypertarget{multinomial-logit-model-1}{%
\subsubsection{Multinomial Logit Model}\label{multinomial-logit-model-1}}

The variables that are significant drivers of the transformation of MFIs under the logit and probit models are also significant in the multinomial logit model. We also generate the \texttt{confusion\ matrix} using the \texttt{multinomial\ logit} with the complete data (Table 2.14). The matrix shows that the overall \texttt{accuracy} is 56.5\%. Note that the overall accuracy is one value for the entire model, the \texttt{No\ Information\ Rate\ (NIR)}. Overall \texttt{accuracy}, in this case, is the ability to accurately predict that an MFI is an NGO when it is an NGO, a commercial bank when it is a commercial bank, and so on. If we were to guess that every MFI in the model is the cooperative - the most prevalent legal form- we would be right 39.3\% of the time, the NIR. The p-value shows that the overall \texttt{accuracy} metric is significantly different from the \texttt{NIR}. Although the multinomial logit model has a markedly lower accuracy than the logit model, it has a far more demanding task of distinguishing five legal forms of MFIs instead of 2 for the logit model \autocite{kwak2002multinomial}.

The model's \texttt{sensitivity} varies from a low of 47.1\% for NBFI to a high of 61.8\% for NGOs. The \texttt{specificity} is relatively high, ranging from 80.1\% for NBFIs to 99.22\% for rural banks \autocite{ginting2019hate}. NGOs have a \texttt{specificity} of 81.4\%, meaning that the model can predict that an MFI is not an NGO when it is not an NGO 81.4\% of the time. For rural banks, the model can correctly predict over 99\% of the time that an MFI is not a rural bank when it is not a rural bank. The \texttt{balanced\ accuracy} is also reasonably high, with the lowest being NBFIs at 63.9\% and the highest at 79.26\% for rural banks \autocite{hedeker2003mixed}.

\begin{table}

\caption{\label{tab:unnamed-chunk-35}Confusion Matrix and Statistics for the Multinomial Logit Model}
\centering
\fontsize{9}{11}\selectfont
\begin{tabular}[t]{llllll}
\toprule
  & NGO & Bank & NBFI & Coop & Rural Bank\\
\midrule
Accuracy & 0.565 & - & - & - & -\\
NoInformationRate & 0.393 & - & - & - & -\\
Kappa & 0.414 & - & - & - & -\\
sensitivity & 0.618 & 0.6154 & 0.478 & 0.584 & 0.5930\\
specificity & 0.814 & 0.9298 & 0.801 & 0.886 & 0.9922\\
\addlinespace
PosPredValue & 0.437 & 0.5170 & 0.474 & 0.768 & 0.7391\\
NegPredValue & 0.901 & 0.9520 & 0.803 & 0.767 & 0.9849\\
Prevalence & 0.189 & 0.1087 & 0.274 & 0.393 & 0.0360\\
DetectionRate & 0.117 & 0.0669 & 0.131 & 0.229 & 0.0213\\
DetectionPrevalence & 0.268 & 0.1294 & 0.276 & 0.298 & 0.0289\\
\addlinespace
BalancedAccuracy & 0.716 & 0.7726 & 0.639 & 0.735 & 0.7926\\
\bottomrule
\multicolumn{6}{l}{\rule{0pt}{1em}Source: Authors' construction}\\
\multicolumn{6}{l}{\rule{0pt}{1em}\textit{Notes: }}\\
\multicolumn{6}{l}{\rule{0pt}{1em}\textsuperscript{1} Accuracy, No information rate and Kappa are the same accross the legal forms of MFIs}\\
\multicolumn{6}{l}{\rule{0pt}{1em}\textsuperscript{2} Accuracy > NoInformationRate is significant at 1\% confidence level, p = 0.0000}\\
\end{tabular}
\end{table}

\newpage

\hypertarget{regression-diagnostics}{%
\subsection{Regression Diagnostics}\label{regression-diagnostics}}

This section examines three issues that arise in logit models: extreme values, multicollinearity, and linearity, respectively.

\hypertarget{extreme-values}{%
\subsubsection{Extreme values}\label{extreme-values}}

Figure 2.4 below shows that the data indeed has influential values. For robustness, we winsorise the data, removing the top 10\% and the bottom 10\%. Still, the results remain robust, as regression results in Table 2.14 show. It is notable that apart from the change in coefficients' value, the signs remain the same, meaning that influential observations (outliers) are not significant.

\begin{table}[!htbp] \centering 
  \caption{Regression Results - Logit and Probit Models for Winsorized Data} 
  \label{} 
\footnotesize 
\begin{tabular}{@{\extracolsep{5pt}}lcccc} 
\\[-1.8ex]\hline 
\hline \\[-1.8ex] 
 & \multicolumn{4}{c}{\textit{Dependent variable:}} \\ 
\cline{2-5} 
\\[-1.8ex] & \multicolumn{4}{c}{Dummy: Current Legal Status (Standard Errors in Brackets)} \\ 
\\[-1.8ex] & \textit{logistic} & \textit{probit} & \textit{logistic} & \textit{probit} \\ 
\\[-1.8ex] & (1) & (2) & (3) & (4)\\ 
\hline \\[-1.8ex] 
 ageYoung & $-$0.887$^{***}$ & $-$0.514$^{***}$ & $-$0.890$^{***}$ & $-$0.515$^{***}$ \\ 
  & (0.119) & (0.068) & (0.117) & (0.067) \\ 
  & & & & \\ 
 ageMature & $-$1.350$^{***}$ & $-$0.787$^{***}$ & $-$1.270$^{***}$ & $-$0.743$^{***}$ \\ 
  & (0.111) & (0.063) & (0.109) & (0.062) \\ 
  & & & & \\ 
 legal\_traditionCivil & $-$0.221$^{*}$ & $-$0.113 & $-$0.374$^{***}$ & $-$0.196$^{***}$ \\ 
  & (0.130) & (0.076) & (0.124) & (0.072) \\ 
  & & & & \\ 
 legal\_traditionOther & 0.913$^{***}$ & 0.500$^{***}$ & 0.893$^{***}$ & 0.486$^{***}$ \\ 
  & (0.146) & (0.081) & (0.142) & (0.079) \\ 
  & & & & \\ 
 assets & 0.272$^{***}$ & 0.161$^{***}$ & 0.273$^{***}$ & 0.164$^{***}$ \\ 
  & (0.021) & (0.012) & (0.020) & (0.012) \\ 
  & & & & \\ 
 kkm & 0.081$^{***}$ & 0.050$^{***}$ & 0.111$^{***}$ & 0.068$^{***}$ \\ 
  & (0.021) & (0.012) & (0.020) & (0.012) \\ 
  & & & & \\ 
 pcrdbgdp & $-$0.330$^{***}$ & $-$0.187$^{***}$ & $-$0.075 & $-$0.048 \\ 
  & (0.098) & (0.056) & (0.086) & (0.050) \\ 
  & & & & \\ 
 stmktcap & $-$0.269$^{***}$ & $-$0.152$^{***}$ & $-$0.326$^{***}$ & $-$0.183$^{***}$ \\ 
  & (0.044) & (0.025) & (0.041) & (0.024) \\ 
  & & & & \\ 
 gdp\_growth\_annual & 0.026 & 0.018$^{*}$ & 0.037$^{**}$ & 0.024$^{**}$ \\ 
  & (0.018) & (0.010) & (0.017) & (0.010) \\ 
  & & & & \\ 
 Constant & $-$2.140$^{***}$ & $-$1.310$^{***}$ & $-$1.690$^{***}$ & $-$1.050$^{***}$ \\ 
  & (0.485) & (0.288) & (0.325) & (0.191) \\ 
  & & & & \\ 
\hline \\[-1.8ex] 
Year Effects & Yes & Yes & No & No \\ 
Deviance & 664*** & 657*** & 602*** & 595*** \\ 
df & 29 & 29 & 9 & 9 \\ 
Data & Winsorized & Winsorized & Winsorized & Winsorized \\ 
Observations & 4,474 & 4,474 & 4,474 & 4,474 \\ 
Log Likelihood & $-$2,282.000 & $-$2,285.000 & $-$2,314.000 & $-$2,318.000 \\ 
Akaike Inf. Crit. & 4,623.000 & 4,631.000 & 4,648.000 & 4,655.000 \\ 
\hline 
\hline \\[-1.8ex] 
\textit{Note:}  & \multicolumn{4}{l}{$^{*}$p$<$0.1; $^{**}$p$<$0.05; $^{***}$p$<$0.01} \\ 
\end{tabular} 
\end{table}

\begin{landscape}

\begin{figure}
\centering
\includegraphics{_main_files/figure-latex/unnamed-chunk-39-1.pdf}
\caption{\label{fig:unnamed-chunk-39}Visualisation of Outliers}
\end{figure}

\end{landscape}

\hypertarget{multicollinearity}{%
\subsubsection{Multicollinearity}\label{multicollinearity}}

The problem of multicollinearity among independent variables leads to unstable coefficients. In the baseline model, however, multicollinearity is not a significant issue because, as per Table 2.15, in all cases, the variance inflation factors(VIFs) are below the 5 (sometimes 10) threshold that several researchers recommend \autocite{gujarati2012econometrics}. Table 2.15 shows the VIFs for each variable.

\begin{table}

\caption{\label{tab:unnamed-chunk-40}Variance Inflation Factors for Logit Model}
\centering
\fontsize{9}{11}\selectfont
\begin{tabu} to \linewidth {>{\raggedright}X>{\raggedleft}X>{\raggedleft}X>{\raggedleft}X}
\toprule
  & GVIF & Df & GVIF\textasciicircum{}(1/(2*Df))\\
\midrule
age & 1.41 & 2 & 1.09\\
legal\_tradition & 2.57 & 2 & 1.27\\
assets & 1.49 & 1 & 1.22\\
kkm & 1.16 & 1 & 1.08\\
pcrdbgdp & 2.01 & 1 & 1.42\\
\addlinespace
stmktcap & 2.72 & 1 & 1.65\\
gdp\_growth\_annual & 1.15 & 1 & 1.07\\
factor(year) & 1.59 & 20 & 1.01\\
\bottomrule
\multicolumn{4}{l}{\rule{0pt}{1em}Source: Authors' construction}\\
\end{tabu}
\end{table}

\hypertarget{linearity-assumptions}{%
\subsubsection{Linearity assumptions}\label{linearity-assumptions}}

Here, we check the linear relationship between independent numeric variables and the logit of the outcome by visually inspecting the scatter plot between each predictor and the logit values. As Figure 2.5 below shows, most variables could reasonably fit a linear model, though not perfectly \autocite{cheng2007testing}. The fitted line uses the Locally Weighted Scatterplot Smoothing (LOESS) method, hence the perceived non-linearity.

\begin{landscape}

\begin{figure}
\centering
\includegraphics{_main_files/figure-latex/unnamed-chunk-41-1.pdf}
\caption{\label{fig:unnamed-chunk-41}Linearity of Independent Variables}
\end{figure}

\end{landscape}

\hypertarget{other-robustness-checks}{%
\subsection{Other Robustness Checks}\label{other-robustness-checks}}

In many cases, it is unlikely that a credit union starts as an NGO, given that they primarily serve clients with a similar professional base and geographic background. For this reason, we run a regression, excluding cooperatives, with the results displayed in Appendix 1. The results remain robust in this case, noting that signs of the coefficients do not change.

\hypertarget{conclusion}{%
\section{Conclusion}\label{conclusion}}

This article examined the factors that drive the transformation of MFIs from the NGO, not-for-profit model to the commercial, for-profit model, focusing on Africa. The analysis shows that at the MFI level, there are three critical factors; age, legal tradition, and size matter. At the aggregate level, it is the country institutional quality and stock market capitalisation that matter. Specifically, older firms are most likely to follow the not-for-profit model, while newer firms are most likely commercial. We expect that older firms are better at attracting donations and subsidies and hence still follow the earlier tradition of microfinance as a welfare tool to aid the financially excluded \autocite{d2017ngos}. For legal tradition, MFIs in civil law countries are the least likely to follow the commercial model relative to those in common law countries. The results align with the finance and law literature, where civil law countries have weaker capital markets. Hence, MFIs have a market void to fill profitably, unlike in common law countries where mainstream markets already fill much of the gap, leaving relatively fewer profit opportunities \autocite{la2013law,schnyder2018twenty}. MFIs in countries following other legal traditions other than common law and civil law are most likely to follow the commercial model.

Turning to size, larger firms tend to adopt the commercial model compared to relatively younger firms. We expect that larger firms are better at attracting commercial capital courtesy of the goodwill and the collateral to pledge when seeking funds publicly. Institutional quality relates positively to adopting the commercial model at the country level, while stock market capitalisation has the opposite effect. Institutional quality affects the ease of contracting, contract enforcement, and property rights \autocite{claessens2003financial}. Private credit to GDP and GDP growth rates are not significant drivers of the conversion of MFIs. However, the coefficients' signs indicate that private credit to GDP, like the stock market to GDP ratio, is inversely related to transformation probability. At the same time, GDP growth shows mixed effects, with significant positive coefficients.

\newpage

\hypertarget{appendices}{%
\section{Appendices}\label{appendices}}

\hypertarget{appendix-1-binary-models-excl.-cooperatives}{%
\subsection{Appendix 1: Binary Models (Excl. Cooperatives)}\label{appendix-1-binary-models-excl.-cooperatives}}

\begin{table}[!htbp] \centering 
  \caption{Models Excluding Cooperatives (Standard Errors in Brackets)} 
  \label{} 
\footnotesize 
\begin{tabular}{@{\extracolsep{5pt}}lcc} 
\\[-1.8ex]\hline 
\hline \\[-1.8ex] 
 & \multicolumn{2}{c}{\textit{Dependent variable:}} \\ 
\cline{2-3} 
\\[-1.8ex] & \multicolumn{2}{c}{Dummy: Current Legal Status} \\ 
\\[-1.8ex] & \textit{logistic} & \textit{probit} \\ 
\\[-1.8ex] & (1) & (2)\\ 
\hline \\[-1.8ex] 
 ageYoung & $-$0.747$^{***}$ & $-$0.421$^{***}$ \\ 
  & (0.114) & (0.065) \\ 
  & & \\ 
 ageMature & $-$1.200$^{***}$ & $-$0.691$^{***}$ \\ 
  & (0.106) & (0.060) \\ 
  & & \\ 
 legal\_traditionCivil & $-$0.421$^{***}$ & $-$0.239$^{***}$ \\ 
  & (0.117) & (0.068) \\ 
  & & \\ 
 legal\_traditionOther & 0.744$^{***}$ & 0.387$^{***}$ \\ 
  & (0.132) & (0.073) \\ 
  & & \\ 
 assets & 0.240$^{***}$ & 0.142$^{***}$ \\ 
  & (0.019) & (0.011) \\ 
  & & \\ 
 kkm & 0.095$^{***}$ & 0.057$^{***}$ \\ 
  & (0.019) & (0.011) \\ 
  & & \\ 
 pcrdbgdp & $-$0.112 & $-$0.049 \\ 
  & (0.076) & (0.042) \\ 
  & & \\ 
 stmktcap & $-$0.327$^{***}$ & $-$0.190$^{***}$ \\ 
  & (0.038) & (0.022) \\ 
  & & \\ 
 gdp\_growth\_annual & 0.016 & 0.012$^{*}$ \\ 
  & (0.011) & (0.006) \\ 
  & & \\ 
 Constant & $-$2.030$^{***}$ & $-$1.260$^{***}$ \\ 
  & (0.446) & (0.266) \\ 
  & & \\ 
\hline \\[-1.8ex] 
Year Effects & Yes & Yes \\ 
Deviance & 664*** & 657*** \\ 
df & 29 & 29 \\ 
Data & No Credit Unions & No Credit Unions \\ 
Observations & 4,782 & 4,782 \\ 
Log Likelihood & $-$2,439.000 & $-$2,446.000 \\ 
Akaike Inf. Crit. & 4,939.000 & 4,952.000 \\ 
\hline 
\hline \\[-1.8ex] 
\textit{Note:}  & \multicolumn{2}{l}{$^{*}$p$<$0.1; $^{**}$p$<$0.05; $^{***}$p$<$0.01} \\ 
\end{tabular} 
\end{table}

\newpage

\hypertarget{appendix-2-multinomial-logit-model--full-dataset}{%
\subsection{Appendix 2: Multinomial Logit Model- Full Dataset}\label{appendix-2-multinomial-logit-model--full-dataset}}

\begin{table}[!htbp] \centering 
  \caption{Multinomial Logit Model- Full Data (Standard Errors in Brackets)} 
  \label{} 
\footnotesize 
\begin{tabular}{@{\extracolsep{5pt}}lcccc} 
\\[-1.8ex]\hline 
\hline \\[-1.8ex] 
 & \multicolumn{4}{c}{\textit{Dependent variable:}} \\ 
\cline{2-5} 
\\[-1.8ex] & Dummy: Current Legal Status & NBFI & Coop & Rural Bank \\ 
\\[-1.8ex] & (1) & (2) & (3) & (4)\\ 
\hline \\[-1.8ex] 
 ageYoung & $-$1.640$^{***}$ & $-$0.621$^{***}$ & $-$0.543$^{***}$ & $-$0.973$^{***}$ \\ 
  & (0.184) & (0.133) & (0.139) & (0.374) \\ 
  & & & & \\ 
 ageMature & $-$2.650$^{***}$ & $-$1.460$^{***}$ & $-$0.630$^{***}$ & $-$0.840$^{***}$ \\ 
  & (0.174) & (0.128) & (0.128) & (0.294) \\ 
  & & & & \\ 
 legal\_traditionCivil & $-$3.750$^{***}$ & $-$1.190$^{***}$ & 1.890$^{***}$ & $-$5.130$^{***}$ \\ 
  & (0.266) & (0.143) & (0.167) & (1.070) \\ 
  & & & & \\ 
 legal\_traditionOther & $-$0.377$^{*}$ & 0.715$^{***}$ & 2.400$^{***}$ & $-$5.880$^{***}$ \\ 
  & (0.199) & (0.149) & (0.177) & (1.750) \\ 
  & & & & \\ 
 assets & 0.798$^{***}$ & 0.360$^{***}$ & $-$0.007 & 0.420$^{***}$ \\ 
  & (0.038) & (0.026) & (0.024) & (0.070) \\ 
  & & & & \\ 
 kkm & 0.450$^{***}$ & 0.250$^{***}$ & $-$0.066$^{**}$ & $-$1.480$^{***}$ \\ 
  & (0.034) & (0.024) & (0.026) & (0.185) \\ 
  & & & & \\ 
 pcrdbgdp & $-$0.008 & $-$0.048 & $-$0.250$^{***}$ & $-$2.640$^{***}$ \\ 
  & (0.109) & (0.085) & (0.091) & (0.624) \\ 
  & & & & \\ 
 stmktcap & $-$0.266$^{***}$ & $-$0.217$^{***}$ & $-$0.364$^{***}$ & $-$1.300$^{***}$ \\ 
  & (0.061) & (0.045) & (0.049) & (0.314) \\ 
  & & & & \\ 
 gdp\_growth\_annual & 0.043$^{**}$ & 0.054$^{***}$ & $-$0.005 & 0.019 \\ 
  & (0.018) & (0.014) & (0.013) & (0.079) \\ 
  & & & & \\ 
 Constant & $-$11.400$^{***}$ & $-$4.580$^{***}$ & $-$1.250$^{**}$ & $-$3.840 \\ 
  & (0.973) & (0.577) & (0.594) & (3.310) \\ 
  & & & & \\ 
\hline \\[-1.8ex] 
Year Effects & Yes & Yes & Yes & Yes \\ 
Data & Full & Full & Full & Full \\ 
Akaike Inf. Crit. & 10,124.000 & 10,124.000 & 10,124.000 & 10,124.000 \\ 
\hline 
\hline \\[-1.8ex] 
\textit{Note:}  & \multicolumn{4}{l}{$^{*}$p$<$0.1; $^{**}$p$<$0.05; $^{***}$p$<$0.01} \\ 
\end{tabular} 
\end{table}

\newpage

\hypertarget{appendix-3-multinomial-logit-model--full-data-excluding-credit-unions-cooperatives}{%
\subsection{Appendix 3: Multinomial Logit Model- Full Data Excluding Credit Unions/ Cooperatives}\label{appendix-3-multinomial-logit-model--full-data-excluding-credit-unions-cooperatives}}

\begin{table}[!htbp] \centering 
  \caption{Multinomial Logit Model- Full Data Without Cooperatives (Standard Errors in Brackets)} 
  \label{} 
\footnotesize 
\begin{tabular}{@{\extracolsep{5pt}}lcccc} 
\\[-1.8ex]\hline 
\hline \\[-1.8ex] 
 & \multicolumn{4}{c}{\textit{Dependent variable:}} \\ 
\cline{2-5} 
\\[-1.8ex] & Dummy: Current Legal Status & NBFI & Coop & Rural Bank \\ 
\\[-1.8ex] & (1) & (2) & (3) & (4)\\ 
\hline \\[-1.8ex] 
 ageYoung & $-$1.640$^{***}$ & $-$0.621$^{***}$ & $-$0.543$^{***}$ & $-$0.973$^{***}$ \\ 
  & (0.184) & (0.133) & (0.139) & (0.374) \\ 
  & & & & \\ 
 ageMature & $-$2.650$^{***}$ & $-$1.460$^{***}$ & $-$0.630$^{***}$ & $-$0.840$^{***}$ \\ 
  & (0.174) & (0.128) & (0.128) & (0.294) \\ 
  & & & & \\ 
 legal\_traditionCivil & $-$3.750$^{***}$ & $-$1.190$^{***}$ & 1.890$^{***}$ & $-$5.130$^{***}$ \\ 
  & (0.266) & (0.143) & (0.167) & (1.070) \\ 
  & & & & \\ 
 legal\_traditionOther & $-$0.377$^{*}$ & 0.715$^{***}$ & 2.400$^{***}$ & $-$5.880$^{***}$ \\ 
  & (0.199) & (0.149) & (0.177) & (1.750) \\ 
  & & & & \\ 
 assets & 0.798$^{***}$ & 0.360$^{***}$ & $-$0.007 & 0.420$^{***}$ \\ 
  & (0.038) & (0.026) & (0.024) & (0.070) \\ 
  & & & & \\ 
 kkm & 0.450$^{***}$ & 0.250$^{***}$ & $-$0.066$^{**}$ & $-$1.480$^{***}$ \\ 
  & (0.034) & (0.024) & (0.026) & (0.185) \\ 
  & & & & \\ 
 pcrdbgdp & $-$0.008 & $-$0.048 & $-$0.250$^{***}$ & $-$2.640$^{***}$ \\ 
  & (0.109) & (0.085) & (0.091) & (0.624) \\ 
  & & & & \\ 
 stmktcap & $-$0.266$^{***}$ & $-$0.217$^{***}$ & $-$0.364$^{***}$ & $-$1.300$^{***}$ \\ 
  & (0.061) & (0.045) & (0.049) & (0.314) \\ 
  & & & & \\ 
 gdp\_growth\_annual & 0.043$^{**}$ & 0.054$^{***}$ & $-$0.005 & 0.019 \\ 
  & (0.018) & (0.014) & (0.013) & (0.079) \\ 
  & & & & \\ 
 Constant & $-$11.400$^{***}$ & $-$4.580$^{***}$ & $-$1.250$^{**}$ & $-$3.840 \\ 
  & (0.973) & (0.577) & (0.594) & (3.310) \\ 
  & & & & \\ 
\hline \\[-1.8ex] 
Year Effects & Yes & Yes & Yes & Yes \\ 
Data & Full & Full & Full & Full \\ 
Akaike Inf. Crit. & 10,124.000 & 10,124.000 & 10,124.000 & 10,124.000 \\ 
\hline 
\hline \\[-1.8ex] 
\textit{Note:}  & \multicolumn{4}{l}{$^{*}$p$<$0.1; $^{**}$p$<$0.05; $^{***}$p$<$0.01} \\ 
\end{tabular} 
\end{table}

\newpage

\hypertarget{appendix-4-multinomial-logit-model-with-full-dataset-but-no-year-effects}{%
\subsection{Appendix 4: Multinomial Logit Model With Full Dataset But No Year Effects}\label{appendix-4-multinomial-logit-model-with-full-dataset-but-no-year-effects}}

\begin{table}[!htbp] \centering 
  \caption{Multinomial Logit Model- Full Data Without Year Effects (Standard Errors in Brackets)} 
  \label{} 
\footnotesize 
\begin{tabular}{@{\extracolsep{5pt}}lcccc} 
\\[-1.8ex]\hline 
\hline \\[-1.8ex] 
 & \multicolumn{4}{c}{\textit{Dependent variable:}} \\ 
\cline{2-5} 
\\[-1.8ex] & Dummy: Current Legal Status & NBFI & Coop & Rural Bank \\ 
\\[-1.8ex] & (1) & (2) & (3) & (4)\\ 
\hline \\[-1.8ex] 
 ageYoung & $-$1.740$^{***}$ & $-$0.619$^{***}$ & $-$0.543$^{***}$ & $-$1.120$^{***}$ \\ 
  & (0.181) & (0.132) & (0.136) & (0.359) \\ 
  & & & & \\ 
 ageMature & $-$2.620$^{***}$ & $-$1.400$^{***}$ & $-$0.539$^{***}$ & $-$0.822$^{***}$ \\ 
  & (0.170) & (0.126) & (0.125) & (0.284) \\ 
  & & & & \\ 
 legal\_traditionCivil & $-$3.830$^{***}$ & $-$1.290$^{***}$ & 1.700$^{***}$ & $-$5.380$^{***}$ \\ 
  & (0.264) & (0.140) & (0.163) & (1.110) \\ 
  & & & & \\ 
 legal\_traditionOther & $-$0.359$^{*}$ & 0.720$^{***}$ & 2.390$^{***}$ & $-$11.000$^{***}$ \\ 
  & (0.195) & (0.146) & (0.175) & (0.0004) \\ 
  & & & & \\ 
 assets & 0.787$^{***}$ & 0.373$^{***}$ & $-$0.006 & 0.346$^{***}$ \\ 
  & (0.036) & (0.025) & (0.023) & (0.061) \\ 
  & & & & \\ 
 kkm & 0.471$^{***}$ & 0.260$^{***}$ & $-$0.044$^{*}$ & $-$1.420$^{***}$ \\ 
  & (0.033) & (0.024) & (0.025) & (0.138) \\ 
  & & & & \\ 
 pcrdbgdp & 0.156 & 0.081 & 0.001 & $-$2.300$^{***}$ \\ 
  & (0.102) & (0.080) & (0.082) & (0.371) \\ 
  & & & & \\ 
 stmktcap & $-$0.269$^{***}$ & $-$0.248$^{***}$ & $-$0.431$^{***}$ & $-$1.110$^{***}$ \\ 
  & (0.060) & (0.044) & (0.047) & (0.239) \\ 
  & & & & \\ 
 gdp\_growth\_annual & 0.057$^{***}$ & 0.061$^{***}$ & $-$0.0003 & $-$0.027 \\ 
  & (0.016) & (0.013) & (0.012) & (0.039) \\ 
  & & & & \\ 
 Constant & $-$10.700$^{***}$ & $-$4.540$^{***}$ & $-$0.470 & $-$0.495 \\ 
  & (0.558) & (0.371) & (0.349) & (1.010) \\ 
  & & & & \\ 
\hline \\[-1.8ex] 
Year Effects & Yes & No & No & No \\ 
Data & No Coop & No Coop & No Coop & No Coop \\ 
Akaike Inf. Crit. & 10,187.000 & 10,187.000 & 10,187.000 & 10,187.000 \\ 
\hline 
\hline \\[-1.8ex] 
\textit{Note:}  & \multicolumn{4}{l}{$^{*}$p$<$0.1; $^{**}$p$<$0.05; $^{***}$p$<$0.01} \\ 
\end{tabular} 
\end{table}

\newpage

\hypertarget{appendix-5-multinomial-logit-model--full-data-excluding-credit-unions-cooperatives-and-year-effects}{%
\subsection{Appendix 5: Multinomial Logit Model- Full Data Excluding Credit Unions/ Cooperatives and Year Effects}\label{appendix-5-multinomial-logit-model--full-data-excluding-credit-unions-cooperatives-and-year-effects}}

\begin{table}[!htbp] \centering 
  \caption{Multinomial Logit Model- Full Data Excluding Cooperatives and Year Effects (Standard Errors in Brackets)} 
  \label{} 
\footnotesize 
\begin{tabular}{@{\extracolsep{5pt}}lcccc} 
\\[-1.8ex]\hline 
\hline \\[-1.8ex] 
 & \multicolumn{4}{c}{\textit{Dependent variable:}} \\ 
\cline{2-5} 
\\[-1.8ex] & Dummy: Current Legal Status & NBFI & Coop & Rural Bank \\ 
\\[-1.8ex] & (1) & (2) & (3) & (4)\\ 
\hline \\[-1.8ex] 
 ageYoung & $-$1.740$^{***}$ & $-$0.619$^{***}$ & $-$0.543$^{***}$ & $-$1.120$^{***}$ \\ 
  & (0.181) & (0.132) & (0.136) & (0.359) \\ 
  & & & & \\ 
 ageMature & $-$2.620$^{***}$ & $-$1.400$^{***}$ & $-$0.539$^{***}$ & $-$0.822$^{***}$ \\ 
  & (0.170) & (0.126) & (0.125) & (0.284) \\ 
  & & & & \\ 
 legal\_traditionCivil & $-$3.830$^{***}$ & $-$1.290$^{***}$ & 1.700$^{***}$ & $-$5.380$^{***}$ \\ 
  & (0.264) & (0.140) & (0.163) & (1.110) \\ 
  & & & & \\ 
 legal\_traditionOther & $-$0.359$^{*}$ & 0.720$^{***}$ & 2.390$^{***}$ & $-$11.000$^{***}$ \\ 
  & (0.195) & (0.146) & (0.175) & (0.0004) \\ 
  & & & & \\ 
 assets & 0.787$^{***}$ & 0.373$^{***}$ & $-$0.006 & 0.346$^{***}$ \\ 
  & (0.036) & (0.025) & (0.023) & (0.061) \\ 
  & & & & \\ 
 kkm & 0.471$^{***}$ & 0.260$^{***}$ & $-$0.044$^{*}$ & $-$1.420$^{***}$ \\ 
  & (0.033) & (0.024) & (0.025) & (0.138) \\ 
  & & & & \\ 
 pcrdbgdp & 0.156 & 0.081 & 0.001 & $-$2.300$^{***}$ \\ 
  & (0.102) & (0.080) & (0.082) & (0.371) \\ 
  & & & & \\ 
 stmktcap & $-$0.269$^{***}$ & $-$0.248$^{***}$ & $-$0.431$^{***}$ & $-$1.110$^{***}$ \\ 
  & (0.060) & (0.044) & (0.047) & (0.239) \\ 
  & & & & \\ 
 gdp\_growth\_annual & 0.057$^{***}$ & 0.061$^{***}$ & $-$0.0003 & $-$0.027 \\ 
  & (0.016) & (0.013) & (0.012) & (0.039) \\ 
  & & & & \\ 
 Constant & $-$10.700$^{***}$ & $-$4.540$^{***}$ & $-$0.470 & $-$0.495 \\ 
  & (0.558) & (0.371) & (0.349) & (1.010) \\ 
  & & & & \\ 
\hline \\[-1.8ex] 
Year Effects & No & No & No & No \\ 
Data & Full & Full & Full & Full \\ 
Akaike Inf. Crit. & 10,187.000 & 10,187.000 & 10,187.000 & 10,187.000 \\ 
\hline 
\hline \\[-1.8ex] 
\textit{Note:}  & \multicolumn{4}{l}{$^{*}$p$<$0.1; $^{**}$p$<$0.05; $^{***}$p$<$0.01} \\ 
\end{tabular} 
\end{table}

\newpage

\hypertarget{appendix-6-legal-traditions-in-africa}{%
\subsection{Appendix 6: Legal Traditions in Africa}\label{appendix-6-legal-traditions-in-africa}}

\begin{table}[!h]

\caption{\label{tab:unnamed-chunk-47}Legal Traditions in Africa}
\centering
\fontsize{8}{10}\selectfont
\begin{tabular}[t]{>{\raggedright\arraybackslash}p{20em}>{\raggedright\arraybackslash}p{20em}>{\raggedright\arraybackslash}p{20em}}
\toprule
Civil & Common & Others\\
\midrule
Algeria                 , Benin                   , Burkina Faso            , Cameroon                , Central African Republic, Chad                    , Comoros                 , Congo, Rep.             , Cote d'Ivoire           , Gabon                   , Guinea                  , Madagascar              , Mali                    , Mauritania              , Morocco                 , Niger                   , Senegal                 , Togo                    , Tunisia & Botswana    , Eswatini    , Gambia, The , Ghana       , Kenya       , Lesotho     , Liberia     , Malawi      , Namibia     , Nigeria     , Sierra Leone, South Africa, South Sudan , Sudan       , Tanzania    , Uganda      , Zambia      , Zimbabwe & Angola                 , Burundi                , Cape Verde             , Congo, Dem. Rep.       , Egypt, Arab Republic of, Equatorial Guinea      , Eritrea                , Ethiopia               , Guinea-Bissau          , Mozambique             , Rwanda\\
\bottomrule
\multicolumn{3}{l}{\rule{0pt}{1em}Source: Oto-Peralías and Romero-Ávila (2014)}\\
\multicolumn{3}{l}{\rule{0pt}{1em}\textit{Note: }}\\
\multicolumn{3}{l}{\rule{0pt}{1em}\textsuperscript{1} Other legal traditions include Spanish, Portuguese, Belgian, and Italian}\\
\multicolumn{3}{l}{\rule{0pt}{1em}\textsuperscript{2} Ethiopia is a peculiar case of a country in Africa that was not colonised}\\
\end{tabular}
\end{table}

\begin{savequote}
``All is changed, save the river and the hill-- Even they are changed.
Only the burning sun and the quiet stars are the same. And we--we, the
memories, stand here in awe, Our eyes closed with the weariness of
tears-- In immeasurable weariness''
\qauthor{--- Edgar Lee Masters.}\end{savequote}



\hypertarget{transformation-of-microfinance-institutions-and-its-effects-on-financial-inclusion-in-africa}{%
\chapter{Transformation of Microfinance Institutions and its Effects on Financial Inclusion in Africa}\label{transformation-of-microfinance-institutions-and-its-effects-on-financial-inclusion-in-africa}}

\chaptermark{MFI Transformation and Financial Inclusion}

\minitoc 

\newpage

\begin{center}

\textbf{Abstract}

\end{center}

\begin{quote}
The shift away from the not-for-profit microfinance institutions (MFIs) model has seen the rise of commercial MFIs in forms like commercial banks, credit unions, and rural banks and the not-for-profit, non-governmental organisations (NGOs). The shift arose partly due to neo-liberalism and the need for MFIs to reach the financially excluded more sustainably than had been the case. Therefore, this article examines how the shift has affected financial inclusion in Africa, utilising data from the Microfinance Information Exchange (MIX). Our results show that the change from the NGO model to the commercial models could negatively affect the depth of financial outreach, especially given that NGOs characteristically have better outreach to women and advance smaller denomination loans on average. Also, NGOs have higher median gross loans than other legal forms of MFIs except for credit unions/ cooperatives, although commercial banks have the highest average gross loans. These results remain robust upon removing outliers and controlling for factors that affect the ability of MFIs to offer financial services to the poor.
\end{quote}

\newpage

\hypertarget{background-1}{%
\section{Background}\label{background-1}}

In 1992, PRODEM, a micro-finance institution (MFI) in Bolivia, converted from a non-governmental organisation (NGO) to a commercial bank, BANCOSOL. In fact, in the immediate past three decades, numerous NGO MFIs across the globe have adopted the commercial forms of microfinance (Table 1). In this article, we examine how the conversion of MFIs to the commercial model affects financial inclusion in terms of the depth and breadth of outreach to the financially excluded. Depth refers to the extent of the traditional financially excluded clients reached by MFIs. If an MFI serves more financially excluded people like women and the poor, it has deeper outreach.

On the other hand, breadth refers to the sheer number of clients reached regardless of their level of financial exclusion. Thus, an MFI that offers more loans to many people has more breadth of outreach. In other words, we explore how the transformation of MFIs typically drives their average loan balance per borrower, the proportion of women borrowers and gross loans. The former two metrics capture depth while the latter proxies breadth \footnote{The quote at the beginning of the chapter is from the poem \emph{Edith Conant} by Edgar Lee Masters in \textcite{masters2007spoon}}.

The study focuses on Africa, a continent that is the epicentre of financial exclusion despite remarkable economic progress of the last three decades \autocite{beck2014sme,allen2011african}. Evaluating the effects of transformation by using global metrics is likely to mask regional heterogeneity, given that these effects could manifest differently in varying settings \autocite{d2017ngos,d2013unsubsidized} \footnote{We use the terms financial sustainability/ efficiency/ profitability on the one hand and social performance/ outreach on the other interchangeably. By Financial sustainability, we refer to the capability of a firm to turn a profit, which allows it to meet its obligations without relying on donations and subsidies. Social performance/ outreach is the firm's ability to reach out and avail financial services to the financially excluded members of society, including the poor, women, and rural dwellers- referring to both breadth and depth.}.

Most pioneer microfinance institutions adopted a not-for-profit model \autocite{dichter1996questioning}, operating mainly as non-governmental organisations (NGOs). However, the dominance of neo-liberalism in organising production has seen many donors scale back and push MFIs to strive for financial sustainability \autocite{bateman2010doesn}. The arguments for the commercial approach to running microfinance activities revolve around sustainability. The financial sustainability school posits that MFIs can best serve the financially excluded when they have a degree of financial self-sufficiency \autocite{kodongo2013individual}. For instance, profit-oriented MFIs could offer financial services to the relatively well-off at market rates and use the proceeds (profits) to subsidise services to the poor more than relying on donations and subsidies alone. Hence, MFIs pursuing the for-profit model may experience mission expansion \autocite{mersland2010microfinance,louis2013financial}. Also, donor funds are volatile and subject to political and economic conditions \autocite{garmaise2013cheap,d2017aid}. In this respect, a substantial body of research finds that the transformation of microfinance institutions enhances outreach to the financially excluded \autocite{frank2008stemming,gutierrez2009social,mersland2010microfinance,quayes2012depth,mia2017mission,d2013unsubsidized}.

The proponents of MFI transformation point to the concerning possibility of mission drift. Mission drift happens when MFIs focus less on providing financial services to the financially excluded in favour of making profits. Some researchers have found this to be the case \autocite{louis2013financial,bos2015practice,d2013unsubsidized,hartarska2012governance}. Two theories can be said to underpin the MFI transformation phenomenon. The first is the agency theory on the conflicts between providers of capital and managers. In a quest to minimise agency conflicts, managers may consciously or sub-consciously place less emphasis on the social mission of MFIs, reaching out to the financially excluded. Instead, managers may more overtly focus on pursuing financial returns for shareholders and debt-holders, thus causing mission drift. The theory presumes that the motivation for all fund providers is financial returns, which is not always the case.

The second one, the institutional theory, examines the rise, persistence and decline of institutional structures over time. The central question here is; What drives the adoption and fall of certain institutional norms \autocite{powell2012new}? In this respect, some institutionalists claim that prevailing institutional culture is more potent than market forces in driving the adoption or rejection of emergent institutional structures. Institutionalists posit that one of the drivers is coercion. In the case of MFIs, some donors have put implicit or explicit pressure on financial sustainability.

Additionally, the adoption of institutional norms in most cases arises out of the need to fit into the institutional environment. The desire to be compliant may explain the prevalence of NGOs not-for-profit type MFIs in the early years of the microfinance paradigm and the trend towards the transformation of MFIs to commercial entities that is now ongoing. Notably, institutional theory sheds light on the dilemmas managers face when institutional norms change \autocite{thornton2015institutional}. For instance, how can managers of MFIs reconcile financial sustainability with the original priority of outreach to the financially excluded?

\begin{table}

\caption{\label{tab:unnamed-chunk-50}Sample of Transformed MFIs}
\centering
\fontsize{9}{11}\selectfont
\begin{tabular}[t]{llll}
\toprule
Institution & Country & Year & Converted\_to\\
\midrule
Finansol & Colombia & 1993 & Commercial Finance Company\\
OIBM & Malawi & 2002 & NBFI\\
PRIDE & Tanzania & 2009 & NBFI\\
Kenya Women Finance Trust & Kenya & 2010 & NBFI\\
Faulu & Kenya & 2010 & NBFI\\
\addlinespace
OI-SASL & Ghana & 2013 & NBFI\\
\bottomrule
\multicolumn{4}{l}{\rule{0pt}{1em}Source: Authors' construction from the literature}\\
\multicolumn{4}{l}{\rule{0pt}{1em}\textit{Note: }}\\
\multicolumn{4}{l}{\rule{0pt}{1em}\textsuperscript{1} This is a snapshot of the many MFIs that have converted over the years accross the globe}\\
\end{tabular}
\end{table}

As noted earlier, research outcomes on the effects of the transformation of microfinance institutions are mixed. \textcite{morduch2019challenges} argue that if commercial MFIs could sustainably achieve financial sustainability while also reaching the poor, NGOs would not exist. In that context, therefore, the most critical question relates to how the transformation of MFIs would affect their core mission of providing financial services to the financially excluded. The issue is vital due to the legitimacy that MFIs derive from serving the financially excluded. Besides, financial inclusion is central to alleviating poverty and achieving inclusive growth, an essential dimension of financial development. In this article, we use data from the Microfinance Information Exchange (MIX) to evaluate the ways that the transformation of MFIs affects financial inclusion in Africa.

We capture the extent of financial inclusion in Africa by using three metrics:

\begin{itemize}
\tightlist
\item
  The percentage of female borrowers.
\item
  Average loan balance per borrower.
\item
  The ratio of the gross loan portfolio to total assets of each MFI \autocite{d2017ngos}.
\end{itemize}

The first two metrics proxy the depth of outreach, with more significant outreach to women indicating deeper outreach, given that women form a substantial proportion of the financially excluded population in Africa \autocite{ayyagari2013financing}. A higher average loan balance per borrower, on the other hand, corresponds to a lower depth of outreach to any group of the financially excluded. The presumption is that financially excluded people usually borrow in smaller denominations that have drawn reservations from some researchers who argue poor people could progressively demand bigger loans as they get better off. Also, MFIs regularly use progressive lending where people who successfully pay off loans qualify for larger loan amounts. Finally, gross loans to assets capture the breadth of outreach, with higher ratios indicating more breadth.

We have organised the rest of the article as follows. Section 1.1 highlights the results of the study. In section 2, we review the background literature on MFI transformation. In section 3, we describe the methodology and, in section 4, we present and discuss the results and close with concluding remarks in section 5.

\hypertarget{summary-of-results-1}{%
\subsection{Summary of Results}\label{summary-of-results-1}}

Overall, we find that the conversion away from the NGO, not-for-profit model in Africa is harmful to financial inclusion's depth and breadth. NGO-type MFIs consistently outperform the commercial-oriented MFIs regarding the outreach to women borrowers. Additionally, NGO-type MFIs have the lowest average loan balance per borrower, indicating that they reach out to the poorest and, presumably, more financially excluded people. Turning to the ratio of gross loans to assets, NGO-type MFIs come second to credit unions/ cooperatives, thus indicating that breadth and depth of outreach are not necessarily mutually exclusive. Further examination of the trend suggests that profit-oriented MFIs reflect a measure of mission drift.

On the one hand, serving poor, financially excluded people is costly, which hurts the profitability of MFIs. On the other hand, profit orientation implies commercial capital, interest expense on debt capital and dividends on equity capital. If it is hard to reconcile these two objectives, we are inclined to back the literature that opposes the commercialisation of MFIs. As we shall see later, it seems odd that NGO-type MFIs lend more gross loans (breadth) than most commercial-oriented MFIs, though it turns out that both cooperatives and NGO-type MFIs hold the lowest volume of assets relative to other legal types. Furthermore, the other important drivers of financial inclusion are the age of MFI, operating expense to assets ratio, donations to assets ratio, capital to assets ratio, asset structure, size, education and profit margin. In the next section, we highlight the methodology and then go to the details of the results.

\hypertarget{theory-and-empirical-literature}{%
\section{Theory and Empirical Literature}\label{theory-and-empirical-literature}}

The extent to which the transformation of MFIs affects financial inclusion has been the subject of substantial research. However, there is a lack of consensus on the outcomes about its effects. Theories underlying aspects of the transformation of MFIs are the agency theory \autocite{jensen1976theory} and institutional theory \autocite{powell2012new}. Agency theory, in this case, implies that injection of commercial capital, a consequence of transformation, is likely to motivate managers to target financial return at the expense of social return to satisfy shareholders and debt-holders, the conventional providers of commercial capital. From this perspective, transformation implies that mission drift is inevitable. Indeed, \textcite{morduch2019challenges} argue that if mission drift is not an issue in microfinance, then the NGOs MFI model would not exist, meaning that NGOs (not-for-profit) MFIs exist to fill a gap left by commercial MFIs.

The institutionalists weigh how specific organisational structures dominate and ultimately decline and get discarded \autocite{powell2012new}. Institutionalists note that in certain situations, people adopt given structures without critical scrutiny to fit into the prevailing institutional environment merely. This argument could partly explain the prevalence of NGO MFI models at the early stages of microfinance evolution and the current rise of MFI commercialisation. However, the pressure to change takes several forms, with the most notable one being coercive pressure, where stakeholders put forth overt or covert pressure for MFIs to convert. In the case of MFIs, the pressure to adopt a commercial model came with the rise of neo-liberal thought around production and its funding \autocite{bateman2010doesn}, with major donors like USAID signalling their expectation that MFIs should become more financially sustainable going forward \autocite{d2013unsubsidized}. The problem for MFIs that transform is how best to balance between social goals of reaching the poor and the commercial goals that come with commercial capital and the attendant decline of donor funding.

\textcite{thornton2002rise} and \textcite{thornton2015institutional} note that ``the meaning and legitimacy of various sources of organisational identity, strategy and structure are shaped by a prevailing institutional logic''. The management of transformed MFIs can identify with microfinance as a social pursuit by emphasising social goals over profits. Alternatively, they may view microfinance as a financial venture by placing profits over social outreach. The former corresponds to the welfare model of microfinance, which posits that the social mission of microfinance is incompatible with the profit motive. The latter is the financial sustainability model, which views financial returns as a precondition for the sustainable pursuit of financial services goals of reaching the financially excluded. A third model, the win-win approach, attempts to reconcile the welfare and sustainability approach by proposing that financial and social performance are not always substituting but complementary. Different researchers have availed evidence in support of either school, as described next.

As noted, support for MFI transformation rests on two primary grounds. First, donations are subject to social, economic, and political conditions \autocite{garmaise2013cheap,armendariz2013subsidy,d2017aid}. Consequently, some researchers argue that microfinance can only be sustainable if MFIs have a level of financial self-sufficiency. In this regard, these scholars note that MFIs could advance financial services to the financially well-off and use the proceeds (profits) to reach more financially excluded people at subsidised rates, which would then lead to ``mission expansion'' as opposed to ``mission drift''. \textcite{frank2008stemming} provides empirical support for these arguments noting that transformed MFIs score higher in terms of client outreach and the number of female clients reached, although the proportion of female clients reached declines. They also find that transformed MFIs record higher growth in gross loan portfolio with better product diversification.

Similarly, \textcite{d2017ngos} finds that transformed MFIs charge a lower interest rate to micro-borrowers. \textcite{louis2013financial}, using self-organising maps and k-means clustering, find a positive relationship between financial sustainability and social performance to imply that steps to enhance financial sustainability are good. Other researchers that have found a positive link between financial and social efficiency include \textcite{gutierrez2009social}, \textcite{mersland2010microfinance}, and \textcite{quayes2012depth}.

In contrast to the above findings, several researchers have found transformation to be harmful in terms of outreach to the financially excluded. For instance, \textcite{d2017ngos} find that although transformed MFIs charge lower interest and experience a drop in operating expenses, their average loan sizes increase, indicative of mission drift. \textcite{mia2017mission} also find a trade-off between depth of outreach and the profit motive of MFIs in Bangladesh using both static and dynamic panel data methods. \textcite{d2013unsubsidized} notes that MFIs with little or no subsidies exhibit more significant mission drift. In our case, NGOs have the highest donations, implying that they may show greater social inclination. For instance, firms in Africa and Asia compensate for low subsidies by charging higher interest rates, while Latin America serves fewer women. In Europe and Central Asia, the tendency is to serve fewer indigent clients. \textcite{bos2015practice} also notes that MFIs that stay close to their original mission are the most socially efficient, while those that attempt to pursue a double bottom line are relatively inefficient. Further, they note that not all MFIs suffer mission drift the same way, arguing that MFIs with high input-output efficiency may not experience mission drift at all.

Besides, \textcite{campion1999institutional} argue that the presence or absence of mission drift in a transformed MFI is a corporate governance issue and an outcome of the challenges of the scaling up of MF services. They argue that good corporate governance allows the management to balance between financial performance and outreach. It means that MFIs could address mission drift problems through proper corporate governance regardless of whether an MFI is an NGO or commercial-type entity. Moreover, \textcite{marti2016financial} argue that different social groups such as employees, management, and MFI clients are likely to have different views, including varying definitions of social welfare. Thus, the presence or absence of mission drift may not arise from deliberate management decisions but instead from conflicting viewpoints on the meaning of social welfare between stakeholders. Given the conflicting evidence and varying views regarding mission drift in MFIs, the arguments by \textcite{morduch1999microfinance} and \textcite{morduch2000microfinance} that the microfinance industry should accommodate different legal forms of MFIs to serve different clients' needs appear to be valid.

\hypertarget{method-1}{%
\section{Method}\label{method-1}}

We run fixed and random effects models based on the results of the Hausmann Tests (see Appendix 1). The design of fixed effects models allows for the study of the causes of changes within an entity. It accomplishes this by controlling for all time-invariant differences between the individuals, so the estimated coefficients of the fixed-effects models cannot be biased because of omitted time-invariant characteristics, such as culture \autocite{torres2007panel}. On the other hand, Random effect models assist in controlling for unobserved heterogeneity when the heterogeneity is constant over time and uncorrelated with the explanatory variables. Following \textcite{roberts2013endogeneity}, we fit the following model.

\begin{equation}
y_{it} = \hat{a} + \hat{b}x_{it} + \mu_{it}
\end{equation}

In this case, \(y_{it}\) is the independent variable; interchangeably represented by per cent of female borrowers, average loan balance per borrower, and gross loan portfolio to total assets. The first two metrics capture financial depth, while gross loans capture the breadth of outreach.

Also, \(x_{it}\) is a matrix of independent variables. The variable of interest in our case is the current legal form of the MFI, which enters the model as a dummy representing NGOs, NBFIs, commercial banks, rural banks and credit unions/ cooperatives \autocite{ayyagari2013financing}. The other control variables include age dummy, a dummy for region, operating expenses to assets ratio, donations to assets ratio, equity capital to assets ratio, asset structure, size (logarithm of total assets), education, and profit margin all of which are derived from the literature \autocite{ayyagari2013financing,d2017ngos,d2013unsubsidized}.

Finally, \(\mu_{it}\) is the error term that we assume has zero mean conditional on \(x_{it}\).

Further,

\begin{equation}
\mu_{it} = c_{i} + \varepsilon_{it}
\end{equation}

In the equation, \(c_{i}\) captures the aggregate effects of the unobserved, time-invariant explanatory variables for \(y_{it}\).

In the case where \(c_{i}\) and \(x_{it}\) are correlated, then \(c_{i}\) is a fixed effect, otherwise, it is a random effect. Note that the existence of fixed effects implies the presence of endogeneity. For random effects, on the other hand, endogeneity is not a concern. However, the random-effects model affects the computation of standard errors \autocite{roberts2013endogeneity}. To eliminate the fixed effect prone to endogeneity, we run the within estimator model \autocite{clark2015should}. We present the results from the estimation of the empirical model in the next section.

\hypertarget{results}{%
\section{Results}\label{results}}

In this section, we begin by visualizing the study variables followed by summary statistics of the variables. We then run and discuss the results of the regression model.

\hypertarget{exploratory-data-analysis-1}{%
\subsection{Exploratory Data Analysis}\label{exploratory-data-analysis-1}}

In this section, we explore the data by visualizing the pertinent variables and computing their summary statistics.

\hypertarget{data-visualization}{%
\subsubsection{Data Visualization}\label{data-visualization}}

Figure 1 shows the correlation matrix and a graphical view of the relationships between the numeric variables. The highest level of correlation is between operating expense to assets ratio and donations to assets ratio, meaning that MFIs that receive more donations spend relatively more, which is suitable for financial inclusion. On the other hand, the relatively low correlations between the variables suggest that multicollinearity is not likely to be a significant concern for the regression analysis that we embark on later in the article. The main diagonal shows the distribution of the individual variables. In this case, there is high skewness exhibited by the donations to assets ratio, average loan balance per borrower and gross loans to assets ratio. It means that relatively few firms account for a considerable chunk of the donations received, in this case among NGOs, cooperatives and NBFIs. The highest correlation exists between the operating expenses to assets ratio and donations to assets ratio, meaning that donor-funded MFIs have more operating costs probably because they are less constrained by profit/ interest seeking shareholders and debt holders. This observation may imply that if it is expensive to administer and monitor small loans, then the profit-oriented model is not suitable for financial inclusion as it constrains spending. The summary statistics in Table 2, Table 3, and Table 4 that follow highlight the discussed issues but offer a more comprehensive array of statistical measures, including the mean, standard deviation and quantiles.

\newpage
\begin{landscape}

\begin{figure}
\centering
\includegraphics{_main_files/figure-latex/plots2-1.pdf}
\caption{\label{fig:plots2}Correlation Matrix for Independent Variables}
\end{figure}

\end{landscape}

Next, we visualise each of the numeric variables against the current legal forms (status) of MFIs. We use the median of the variables to stand for the variables. Figure 2 (Panel A) shows that mature MFIs form the bulk of MFIs in the sample. Among mature MFIs, NGOs and cooperatives are the majority, indicating their relatively longer operational cycle than NBFIs, banks, and rural banks. As expected, NGOs receive the highest share of donations, followed by credit unions and NBFIs, while commercial banks receive the least donations (Figure 2- Panel B). The result relating to NGOs is not surprising given they are rooted in the welfare model of microfinance. Most donors are likely to channel their funds to MFIs that place social performance over profits. When commercial capital almost entirely replaces donations, outreach to the poor may likely be affected \autocite{roberts2013profit}, given that managers may emphasise impressing shareholders and debt-holders in line with the agency theory. As noted by \textcite{d2013unsubsidized}, MFIs with little or no subsidies exhibit more significant mission drift. Hence, outreach to the poor would suffer even more where the capital providers do not have a sense of the hybrid nature of microfinance. Therefore, the rise of blended finance where commercial capital funds social causes may partly mitigate this scenario \autocite{attridge2019blended}.

Donations do not prevent NGOs and NBFIs from raising capital as they have the highest capital to assets ratio- which primarily reflects equity injections (Figure 2: Panel C). Commercial banks, credit unions and rural banks follow in that order. The observation is surprising given that NGOs and NBFIs still exhibit a high level of social performance even with a relatively high capital to assets feature. Therefore, it could imply that the profit vs social orientation of an MFI could be driven not just by the needs of the providers of funds but also by the internal governance, mission, and strategic direction of an MFI \autocite{campion1999institutional}. In this respect, an MFI's social mission could outweigh the needs of capital providers. Also, the equity capital NGOs may attract may be preferential in terms of expected returns, as with blended finance \autocite{rode2019blended}. In this case, donors could provide dedicated capital that does not pressure management to make high interest or dividends payments, allowing MFIs to remain predominantly on the social mission path \autocite{lopatta2016microfinance}. \footnote{Apart from credit unions/cooperatives and rural banks, we use acronyms for other legal forms of MFIs. NGO stands for non-governmental organisations, while NBFI connotes non-bank financial institutions. The term bank represents microfinance-oriented commercial banks.}

Appendix 10 and 11 show the visualisation for the debt to equity ratio and deposits to assets ratio. While NGOs attract more equity capital, rural banks, commercial banks, and credit unions rely more on debt, especially deposits form of debt, to finance their operations. The analysis shows that while all MFIs are raising capital, the sources are different for commercial MFIs vis-a-vis NGOs. While NGOs are inclined to using more equity, commercial MFIs appear to rely more on debt. Debt capital gives rise to fixed obligations that may exacerbate mission drift, and hence the conversion of NGOs to commercial entities could be harmful to social outreach. However, the relative inability to garner deposits maybe detrimental to NGOs' ability to access cheap, less restrictive capital.

Lastly, for asset structure (tangibility), the ratio of non-current assets to total assets, credit unions lead followed by commercial banks, NGOs, NBFIs, and rural banks. Asset structure captures the extent to which MFIs invest in physical infrastructure relative to the total asset base. Credit unions tend to serve a narrow geographic region and traditionally put up brick and mortar branches to serve their customers \autocite{mckillop2011credit}. Like credit unions, commercial banks tended to have more branches, having taken root before the advent of fintech, reducing the need for physical branches. NGOs, NBFIs, and rural banks have the lowest rates of asset tangibility, especially those of more recent origins, and using rural agents to meet customers rather than set up an expansive network of branches \autocite{kent2013bankers}.

Figure 3 (Panel A) shows that NGOs exhibit the highest median operating expense to assets ratio followed by NBFIs while credit unions trail. As we see later in the analysis, operating expenses positively relate to the depth of outreach- per cent of female borrowers and breadth of outreach in terms of gross loans to assets. Therefore, NGOs will do better in social outreach as they incur more costs to reach out to the financially excluded. Indeed, literature shows that outreach to the poor is expensive partly due to the dis-economies of scale in serving the poor, financially excluded clients \autocite{mia2017mission}. One of the efficiency enhancement opportunities from the transformation of MFIs to the for-profit approach to microfinance services delivery is that managers could trim operating expenses to increase profits hurting financial inclusion.

Turning to profitability in Figure 3 (Panel B), we find that rural and commercial banks post the highest median profits, while NBFIs and NGOs trail (Figure 3- Panel B). This result probably partially illustrates the emphasis on social performance over financial performance by the management. NGOs and NBFIs are more likely to favour the social goal. When we take this result together with the observation that NGOs tend to have more operating expenses, we conclude that the desire by managers of commercial banks and other for-profit MFIs to mitigate agency conflicts leads to less operating expenses, more profitability and, hence, lower outreach to the financially excluded \autocite{jensen1976theory}.

Overall, the pattern indicates that while NGOs spend the most in operating expenses to reach the financially excluded, these efforts come at the cost of profitability. In contrast, profit-oriented MFIs are keen to manage expenses that improve profitability, presumably at the expense of outreach to the financially excluded. It is worth noting that NGOs have a relatively low asset base and hold relatively fewer non-current assets to total assets. The observation could mean that NGOs do not invest heavily in brick and mortar branches or serve a relatively limited geographic range. Finally, NGOs have the highest capital (equity) to asset ratio despite the push towards commercial capital. Much could be from investors, who put forth dedicated equity capital because they are keen on social performance and not profits \autocite{mia2017mission}.

We now turn to Figure 4. The first graph (Figure 4- Panel A) shows that commercial banks have the most prominent asset size (total assets), while NGOs and cooperatives have the smallest in that order. Banks tend to have a much broader geographic presence and hence attract more clients, which means more assets accumulation. Again, capital adequacy requirements by central banks have implications on the assets that banks hold. Furthermore, commercial banks are generally dominant in many developing countries meaning that they have a long operating history which implies a bigger size \autocite{levine2002bank}. Figure 4 panel B shows that NGOs and NBFIs serve proportionately more women borrowers, indicating their outreach depth. Given that much of the donor funds accrues to NGOs, the conversion to the for-profit model would be detrimental to financial inclusion if coupled with a reduction in donor funding.

Commercial banks and credit unions have the highest average loan balance per borrower (depth of outreach), while NBFIs and NGOs come last in that order (Figure 4- Panel C). As an indicator of outreach to the poor, the average loan balance per borrower is better when lower, indicating that more impoverished people get access to financial services. Again, it appears that profit-orientation by commercial banks may cause MFIs to reach less financially excluded people in favour of making profits. For credit unions, the observation could arise due to the limited geographic range of operations where they serve people with a common interest like the type of occupation, meaning that their members may not be suffering from financial exclusion in the first place \autocite{armendariz2013subsidy}.

NGOs have the highest median gross loans to total assets ratio, surprising given their relatively smaller size. On the other hand, banks and rural banks, respectively, have the lowest gross loans to assets, meaning they are less efficient in converting their assets into credit (Figure 4- Panel D). Again, it shows, at least in the case of Africa, that the breadth and depth of financial outreach are not always mutually exclusive. In this case, NGOs can reach the most impoverished borrowers, going by the per cent of female borrowers and average loan balance per borrower, while at the same time seemingly reaching a broad customer base as indicated by the gross loans to assets. So far, the bulk of MFIs population leans towards the welfare orientation where profitability overtly seems incompatible with outreach to the financially excluded \autocite{lopatta2016microfinance}.

\newpage
\begin{landscape}

\begin{figure}
\centering
\includegraphics{_main_files/figure-latex/unnamed-chunk-61-1.pdf}
\caption{\label{fig:unnamed-chunk-61}Operating Expense, Donations, Capital and Asset Structure of MFIs by Legal Status}
\end{figure}

\newpage

\begin{figure}
\centering
\includegraphics{_main_files/figure-latex/unnamed-chunk-62-1.pdf}
\caption{\label{fig:unnamed-chunk-62}Asset Structure, Profit Margin, Donations, Capital to Assets Ratio by Current Legal Status}
\end{figure}

\end{landscape}

\newpage
\begin{landscape}

\begin{figure}
\centering
\includegraphics{_main_files/figure-latex/unnamed-chunk-63-1.pdf}
\caption{\label{fig:unnamed-chunk-63}Size, Profit Margin, Average Loan Balance per Borrower and Gross of MFIs by Legal Status}
\end{figure}

\end{landscape}
\newpage

\hypertarget{trends-in-dependent-and-independent-variables}{%
\subsubsection{Trends in Dependent and Independent Variables}\label{trends-in-dependent-and-independent-variables}}

Figure 4 below maps the trends, over time, for donations, capital, profits and operating expense to assets ratio, respectively. Figure 5 (\emph{Panel A}) shows that the median donations to assets ratio have been downward for 1999-2019. The observation confirms the neo-liberal shift in the MFI paradigm where donors expect MFIs to be more financially self-sustainable. The trend is expected in light of the abundance of literature on the conversion of MFIs from NGOs and subsequent drop in donor funding \autocite{d2017ngos}. As donations dry up, we expect the debt and equity capital components to fill the void.

Surprisingly, the capital to assets ratio, which captures the extent of the equity capital injection, is also on a downward trend despite the drive towards the commercialization of MFIs (\emph{Figure 5, Panel B}). The observation could be due to a rise in the total asset base of MFIs, as they scale up, with equity capital being a relatively smaller external capital component than debt. \emph{Appendix 8} shows a steady increase in the debt/equity ratio, which means that most MFIs, like commercial banks, use debt (including deposits) to finance their operations. In this respect, debt gives rise to fixed obligations which could hurt profitability during economic downturns.

MFI profitability in Figure 5 (\emph{Panel C}) shows that the median profitability for MFIs is meager and almost invariant over time, except *for a significant dip in mean profitability around 2008-2009 during the global financial crisis period. The result is consistent with the empirical regularity that shows microfinance as a low margin business, largely reflecting the primacy of social mission in microfinance \autocite{hartarska2012governance}. An important observation is that there are no distinctive profitability changes even as more MFIs adopt the commercial model. This observation means that either the commercialization was not very successful in generating profits or that the extent to which MFIs were able to make profits post-transformation is particularly peculiar to each MFI or to each of the legal forms of MFIs.

Lastly, for Figure 5 (\emph{Panel D}), the operating expenses to assets ratio remains approximately constant except for a dip in the 1999-2001 period. As the regressions show in a later section, the operating expense to assets ratio relates positively to outreach depth and breadth. In this case, the drop in the ratio between 1999 and 2001 could have worsened financial inclusion outcomes. However, the operating expenses to assets ratio have levelled off; thus, it could be indicating a sustained commitment to outreach even in the face of the desire to make profits.

Next, we turn to Figure 6, which shows a rise in MFI size, average loan balance per borrower, and gross loans to total assets, while the per cent of female borrowers is on a downward trend. The rise in the average loan's balance per borrower and the accompanying drop in the proportion of female borrowers indicates a consistent decline in the outreach of MFIs to the financially excluded, as MFIs rely less on donations and more on commercial capital. It further shows the effects of the neo-liberal, for-profit paradigm, which may be hurting the social performance of MFIs, precisely the depth of outreach. However, it appears the breadth of outreach is getting better with time as MFIs give more loans. Taken together, it implies that MFIs provide larger loans to less financially excluded people in a bid to make ample financial return to allow for financial sustainability and payment of dividends and interest to investors.

Appendices 12-14 show the breakdown, by legal forms of MFIs, of the trends for per cent of female borrowers, average loan balance per borrower and gross loans to assets. The movement is generally downwards for women borrowers except under the commercial banking legal forms, which have low outreach to women, but that is relatively constant after the initial spike in the early 2000s. The trends indicate a weakening depth of outreach. For average loan balance per borrower, NGOs and rural banks remain relatively constant. At the same time, other legal forms have rising average loan size trends, which indicates worsening outreach by banks, credit unions, and NBFIs. Lastly, gross loans are rising except for rural banks, showing a better breadth of outreach over time.

\begin{landscape}
\newpage

\hypertarget{descriptive-statistics-trends-over-2000-2020}{%
\subsection{Descriptive Statistics: Trends Over 2000-2020}\label{descriptive-statistics-trends-over-2000-2020}}

\includegraphics{_main_files/figure-latex/unnamed-chunk-65-1.pdf}

\newpage

\begin{figure}
\centering
\includegraphics{_main_files/figure-latex/unnamed-chunk-66-1.pdf}
\caption{\label{fig:unnamed-chunk-66}Trends in MFI Size, Female Borrowers, Average Loan Balances, and Gross Loans}
\end{figure}

\end{landscape}

\hypertarget{summary-statistics-1}{%
\subsection{Summary Statistics}\label{summary-statistics-1}}

Tables 2 and 3 show the summary statistics of the variables applied in the regression analysis later on. The summary indicates that there is not a wide variation in the composition of the legal forms of MFIs, with banks (619) and rural banks (138) having the least number of entities in the sample dataset. Turning to age, mature MFIs dominate in the sample dataset, followed by new MFIs (0-4 years) and young MFIs (4-8 years). Table 3 shows a summary of the three dependent variables. We have visualized and discussed the breakdown of these variables by MFI legal status in figure 4.

\begin{table}

\caption{\label{tab:unnamed-chunk-67}Summary Statistics for Categrical Independent Variables}
\centering
\fontsize{9}{11}\selectfont
\begin{tabu} to \linewidth {>{\raggedright}X>{\raggedright}X}
\toprule
Variable & Counts\\
\midrule
Current Legal Status & Credit Unions: 1427, NBFI: 1315, NGO: 1250, Bank: 619\\
Age & Mature: 2558, New: 1200, Young: 1024\\
\bottomrule
\multicolumn{2}{l}{\rule{0pt}{1em}Source: Authors' construction from the MIX data}\\
\multicolumn{2}{l}{\rule{0pt}{1em}\textit{Note: }}\\
\multicolumn{2}{l}{\rule{0pt}{1em}\textsuperscript{1} Legal status include NGO, Non-Bank Financial Institutions (NBFIs), Credit Unions, and Banks}\\
\multicolumn{2}{l}{\rule{0pt}{1em}\textsuperscript{2} Age has mature MFIs older than 8 years, young ones (4 - 8 years), and new ones that are 4 years or less}\\
\end{tabu}
\end{table}

\begin{table}

\caption{\label{tab:unnamed-chunk-68}Summary Statistics for Numeric Dependent Variables}
\centering
\fontsize{9}{11}\selectfont
\begin{tabular}[t]{lrrrrr}
\toprule
Variable & Mean & SD & Q1 & Median & Q3\\
\midrule
percent\_of\_female\_borrowers & 0.569 & 0.237 & 0.421 & 0.550 & 0.748\\
average\_loan\_balance\_per\_borrower & 895.001 & 7332.114 & 142.000 & 335.000 & 776.500\\
gross\_loan\_portfolio\_to\_total\_assets & 0.655 & 0.712 & 0.504 & 0.654 & 0.777\\
\bottomrule
\multicolumn{6}{l}{\rule{0pt}{1em}Source: Authors' construction from the MIX data}\\
\multicolumn{6}{l}{\rule{0pt}{1em}\textit{Note: }}\\
\multicolumn{6}{l}{\rule{0pt}{1em}\textsuperscript{1} The summary statistics are disaggregated by MFI legal status}\\
\end{tabular}
\end{table}

\begin{table}

\caption{\label{tab:unnamed-chunk-69}Summary Statistics for Continous Independent Variables}
\centering
\fontsize{10}{12}\selectfont
\begin{tabular}[t]{lrrrrrrr}
\toprule
Variable & Mean & SD & Min & Q1 & Median & Q3 & Max\\
\midrule
operating\_expense\_assets & 0.227 & 0.185 & 0.00e+00 & 0.124 & 0.181 & 0.269 & 2.52\\
donations\_assets\_ratio & 0.043 & 0.147 & -3.00e-03 & 0.000 & 0.001 & 0.019 & 2.60\\
capital\_asset\_ratio & 0.321 & 0.602 & -1.84e+01 & 0.153 & 0.273 & 0.478 & 12.15\\
asset\_structure & 0.076 & 0.069 & 0.00e+00 & 0.035 & 0.060 & 0.092 & 0.86\\
assets & 14.946 & 2.262 & 6.93e-01 & 13.540 & 14.858 & 16.416 & 22.98\\
\addlinespace
education & 0.387 & 0.144 & 7.50e-02 & 0.273 & 0.386 & 0.487 & 1.05\\
profit\_margin & -7.739 & 513.299 & -3.55e+04 & -0.181 & 0.048 & 0.189 & 6.20\\
\bottomrule
\multicolumn{8}{l}{\rule{0pt}{1em}Source: Authors' construction from the MIX data}\\
\multicolumn{8}{l}{\rule{0pt}{1em}\textit{Note: }}\\
\multicolumn{8}{l}{\rule{0pt}{1em}\textsuperscript{1} The summary statistics are disaggregated by MFI legal status}\\
\end{tabular}
\end{table}

\newpage

\hypertarget{regression-analysis}{%
\subsection{Regression Analysis}\label{regression-analysis}}

This section describes the results of the deeper data analysis, that is, regression models in Table 5 and 6, and Appendix 1 taken alongside the results of the exploratory data analysis. Table 5 shows the output for the fixed effects model (see results of the Hausmann test in Appendix 2) \footnote{Further, Appendix 3 and 4 show significant fixed effects. Hence, in all models, we favour the fixed-effects model. Also, both the fixed and random-effects models take precedence over the pooled OLS. We run both the fixed and random effects model to interpret the results from the dummy variables that drop out in the fixed-effects model.}. Table 6 presents the results of the random effects and the pooled OLS models. We describe the impact of transformation on each dependent variable: the MFI outreach proxies of per cent of women borrowers, average loan balance per borrower, and gross loans to assets. We are working with an unbalanced panel dataset; we run three regressions for each outcome variable. First, we run the regression using the entire dataset, then rerun the regression using data for MFIs with at least three years (panels) of data. Lastly, we run another regression for MFIs with at least five years of data \footnote{when interpreting these results, note that a smaller average loan balance per borrower indicates a better depth of outreach. Therefore, variables that have a negative coefficient have a positive influence on the depth of outreach.}. Table 5 shows the results of the fixed-effects model, while Table 6 presents the random effects and pooled OLS models. The discussion that follows mainly draws from the fixed and random-effects models.

\hypertarget{percent-of-women-borrowers}{%
\subsubsection{Percent of Women Borrowers}\label{percent-of-women-borrowers}}

The legal status of an MFI is a significant driver of outreach to women, with NGOs faring better (see Table 6). Note that NGOs are the base outcome in current legal status, while banks, NBFIs, cooperatives/credit unions, and rural banks are the alternative outcomes. The basis for picking NGOs as the base outcome stems from the trend of NGOs converting to the commercial model, the core of this article. Other positive outreach drivers include education, operating expenses to assets ratio, profit margin, and capital-to-assets ratio. North Africa fares worse in MFI outreach than Sub-Sahara Africa despite being represented entirely by NGOs in the sample, showing the importance of considering regional disparities. Table 6 of the random effects and pooled OLS model shows that NGOs outperform other legal forms of MFIs in reaching out to women. The result implies that the transformation of MFIs can negatively impact financial inclusion efforts as commercial MFIs are less keen to reach the financially excluded, which contradicts some previous research \autocite{ledgerwood1998microfinance,ledgerwood2006transforming,hartarska2012governance,bos2015practice}. Again, the results highlight the potential regional disparities even within Africa, which may question the inferences made using global datasets.

The results gain more credence when examining the operating expenses to assets ratio (Table 5 and 6). There is a positive and significant relationship between operating expense to assets ratio on the one hand and per cent of women borrowers on the other. It means that an MFI has to spend more to reach financially excluded clients, which means lesser profits at a given level of revenue. As expected, the conversion of MFIs from NGOs to the commercial model could reduce operating expenses in the quest for profits in line with the profit incentive arising from the agency theory \autocite{eisenhardt1989agency}. Hence, it arguably follows that pursuit of profit is bad for financial inclusion by causing mission drift in line with prior research \autocite{wagenaar2012institutional,roberts2013profit,lopatta2016microfinance,mia2017mission}. Except in the unlikely scenario where MFIs generate profits by raising revenue without lowering costs, Africa's MFIs and regulators should rethink the case for the transformation of MFIs into the commercial model.

Indeed, profitability has a positive relationship, albeit insignificant on women borrowers' per cent (see Table 5 and 6). With this hindsight, it would appear like the viable explanation for the profitability-operating expense-financial inclusion issue is that for commercial MFIs, it is to reduce operating expenses in the short run if that translates into higher profits. Higher profitability allows the MFIs to reach more financially excluded clients while cross-subsidising them in the medium to long term. \textcite{d2017ngos} note that for transformed MFIs, profits tend to drop in the short term but not necessarily in the long term. The drop in profitability is driven by transformed MFIs charging lower interest rates, a contentious issue in micro-finance. Therefore, it would be worth examining the dynamics between profitability, operating expense, and financial inclusion for MFIs over a more extended period.

The significant control variables are education (table 5 and 6) and region (table 6). Education varies positively with outreach to women, as does region. MFIs in Northern Africa have lower outreach to women than an otherwise similar MFI in Sub-Saharan Africa. These results could be due to religious beliefs or practices that discount financial inclusion outcomes for women or a conscious shift to the Islamic model against charging interest on capital \autocite{hassan2018religious}. This observation is despite North Africa being represented only by NGOs in the sample. Our previous findings show that NGOs have higher outreach to women than other legal forms of MFIs. The implication is that cultural and religious inclinations play a more significant role in driving financial inclusion than the operating model of MFIs in North Africa. Education also appears to play a key role in financial inclusion by empowering women to join the formal labour market and equipping women with financial literacy that allows for better financial decision making \autocite{zins2016determinants,chikalipah2017determines}.

These observed relationships hold even when we winsorise the data (see \emph{Appendix 1}). The only exception is the capital to assets ratio and profit margin, which have a significant positive relationship with outreach to women. These results suggest that larger firms are more likely to experience mission drift after conversion. However, even after winsorising the data, NGOs still do better at financing women in line with research from the welfare approach to microfinance \autocite{kodongo2013individual}. To sum this up, MFIs would best achieve the quest to improve outreach to women by targeting NGOs with capital funding, especially with the rise of blended finance, commercial capital for social projects, ceteris paribus. Some other variables in the model are not significant but are worth mentioning. For instance, older firms have lower levels of outreach to women, which could imply that firms lose focus on financial inclusion as they mature and get financially independent. Next, we examine the effects on another measure of the depth of outreach, the average loan balance per borrower.

\hypertarget{average-loan-balance-per-borrower}{%
\subsubsection{Average Loan Balance per Borrower}\label{average-loan-balance-per-borrower}}

Like the per cent of female borrowers, the average loan balance per borrower captures how deep an MFI goes to reach the financially excluded, who typically would demand smaller denominations of loans. Thus, the smaller the average loan balance, the deeper the outreach. The major criticism of the average loan balance per borrower as an indicator of financial inclusion is that a larger average loan balance could result from progressive lending or arise as clients become better off \autocite{abeysekera2014sustainability}. Again, researchers could be wrongly proclaiming mission drift for MFIs operating in countries that have relatively fewer indigent clients \autocite{armendariz2013subsidy}. Notably, the presence of a few vast loans granted to some clients could tilt the average loan balance upwards \autocite{market2014global}. Despite these shortcomings, the metric of average loan size is helpful because it is easily quantifiable, and the relevant data inputs are readily available. \footnote{when interpreting these results, note that a smaller average loan balance per borrower indicates a better depth of outreach. Therefore, variables that have a negative coefficient have a positive influence on the depth of outreach.}.

The key observation in the description of this result is that NGOs consistently offer smaller average loan sizes than other legal forms of MFIs. However, the effect is only significant for credit unions/ cooperatives. Consistent with the outreach to women, the observation would suggest that NGOs reach the financially excluded better than do commercial MFIs. It would then imply that the conversion of MFIs from NGOs to other legal forms is harmful to financial inclusion, as the welfare school argues. In fact, \textcite{d2017ngos} and \textcite{mia2017mission}, using a global dataset of MFIs, find that average loan balances go up after transformation, which is consistent with our finding. Accordingly, \textcite{peck2001commercialization} argue that commercial logic has, over time, displaced the welfare approach in addressing financial exclusion.

Furthermore, older MFIs have a lower average loan balance per borrower relative to newer ones. The relationship could hold as older firms tend to reach out to more financially excluded clients given their stable financial base, operational experience, and linkage to donors who emphasise social performance \autocite{bos2015practice}. However, the larger the MFI's asset base, the higher the average loan balance, meaning that it is older but relatively smaller firms that better focus on their mission. Overall, it would imply that the growth of an MFI's asset size comes at the expense of outreach to the financially excluded \autocite{armendariz2013subsidy}.

Moreover, the capital to asset ratio has a positive relationship with the average loan balance, while profit margin relates negatively with the average loan size. For capital, the observation would imply that entry of commercial capital negatively influences the extent of financial inclusion, given that poorer people tend to demand smaller loans \autocite{mersland2010microfinance}. As noted, however, some forms of commercial (dedicated) capital could allow MFIs to reach more financially excluded customers. It appears that the nature of capital injection - pure commercial versus preferential commercial capital may have a bearing on the extent to which an MFI focuses on profit generation relative to social outreach \autocite{d2013unsubsidized}.

As noted in the case of female borrowers, profitability is good for financial inclusion. While short-run profitability may hurt financial inclusion, it appears profitability helps extend financial inclusion in the medium to long term \autocite{louis2013financial,quayes2012depth}, which is in line with the win-win school of microfinance \autocite{kodongo2013individual}. Hence, it would be helpful to examine the inter-temporal dynamics between capital and profitability and the breadth and depth of outreach of microfinance institutions in Africa.

The insignificant drivers of average loan balance per borrower include region, operating expense to assets ratio, donations to assets ratio, asset structure, and education. That said, the relationship between operating expense ratio, donations to assets ratio and average loan size is positive, pointing to a negative but insignificant effect of these factors on financial inclusion \autocite{d2017ngos}. For education, the sign is negative, meaning that education has a positive but insignificant impact on financial inclusion, an observation consistent with education's impact on the per cent of female borrowers in section 4.4.1. Asset structure exhibits mixed results. These results here remain robust even after removing extreme values (see \emph{Appendix 1}). The next section examines the breadth of outreach captured by using gross loans to assets ratio.

\begin{landscape}

\begin{table}[!htbp] \centering 
  \caption{Output of Fixed Effects Model of MFI Outreach} 
  \label{} 
\tiny 
\begin{tabular}{@{\extracolsep{5pt}}lccccccccc} 
\\[-1.8ex]\hline 
\hline \\[-1.8ex] 
 & \multicolumn{9}{c}{Dependent Variables} \\ 
\cline{2-10} 
\\[-1.8ex] & \multicolumn{9}{c}{Social Performance} \\ 
 & FemaleClients & FemaleClients & FemaleClients & AverageLoan & AverageLoan & AverageLoan & GrossLoans & GrossLoans & GrossLoans \\ 
\\[-1.8ex] & (1) & (2) & (3) & (4) & (5) & (6) & (7) & (8) & (9)\\ 
\hline \\[-1.8ex] 
 Age: Young & $-$0.00001 & $-$0.001 & $-$0.007 & $-$147.000 & $-$159.000 & $-$1,163.000$^{***}$ & 0.134$^{***}$ & 0.138$^{***}$ & 0.128$^{***}$ \\ 
  & (0.009) & (0.010) & (0.011) & (384.000) & (399.000) & (313.000) & (0.024) & (0.025) & (0.028) \\ 
  & & & & & & & & & \\ 
 Age: Mature & $-$0.002 & $-$0.003 & $-$0.009 & $-$755.000 & $-$780.000 & $-$1,607.000$^{***}$ & 0.168$^{***}$ & 0.172$^{***}$ & 0.174$^{***}$ \\ 
  & (0.015) & (0.015) & (0.017) & (586.000) & (606.000) & (530.000) & (0.031) & (0.032) & (0.039) \\ 
  & & & & & & & & & \\ 
 Operating Expense & 0.045$^{*}$ & 0.045$^{*}$ & 0.074$^{**}$ & 584.000 & 605.000 & 1,395.000 & 0.282$^{***}$ & 0.283$^{***}$ & 0.304$^{***}$ \\ 
  & (0.026) & (0.027) & (0.032) & (1,560.000) & (1,606.000) & (1,680.000) & (0.078) & (0.081) & (0.089) \\ 
  & & & & & & & & & \\ 
 Donations/Assets & $-$0.009 & $-$0.008 & 0.0005 & 1,706.000 & 1,731.000 & 1,507.000 & $-$0.290$^{***}$ & $-$0.296$^{***}$ & $-$0.313$^{***}$ \\ 
  & (0.024) & (0.025) & (0.030) & (1,144.000) & (1,185.000) & (1,102.000) & (0.085) & (0.089) & (0.088) \\ 
  & & & & & & & & & \\ 
 Capital/Assets & 0.001 & 0.001 & 0.001 & 203.000 & 202.000 & 952.000$^{***}$ & 0.652$^{***}$ & 0.654$^{***}$ & 1.170$^{***}$ \\ 
  & (0.005) & (0.005) & (0.009) & (163.000) & (170.000) & (300.000) & (0.030) & (0.031) & (0.044) \\ 
  & & & & & & & & & \\ 
 Asset Structure & 0.028 & 0.036 & 0.069 & $-$1,028.000 & $-$1,119.000 & $-$2,649.000 & $-$0.507$^{***}$ & $-$0.516$^{**}$ & $-$0.921$^{***}$ \\ 
  & (0.060) & (0.064) & (0.081) & (4,101.000) & (4,384.000) & (4,229.000) & (0.197) & (0.209) & (0.245) \\ 
  & & & & & & & & & \\ 
 Size(Lassets) & $-$0.066 & $-$0.057 & 0.154$^{*}$ & 21,753.000$^{***}$ & 22,088.000$^{***}$ & 37,681.000$^{***}$ & $-$0.825$^{***}$ & $-$0.865$^{***}$ & $-$0.345 \\ 
  & (0.053) & (0.054) & (0.083) & (2,824.000) & (2,927.000) & (4,565.000) & (0.143) & (0.149) & (0.212) \\ 
  & & & & & & & & & \\ 
 Education & 0.161$^{**}$ & 0.164$^{**}$ & 0.180$^{**}$ & $-$4,894.000 & $-$4,791.000 & $-$2,566.000 & $-$0.096 & $-$0.096 & $-$0.024 \\ 
  & (0.078) & (0.079) & (0.087) & (3,353.000) & (3,459.000) & (3,237.000) & (0.140) & (0.146) & (0.196) \\ 
  & & & & & & & & & \\ 
 Profit Margin & 0.00000 & 0.00000 & 0.002 & $-$0.092 & $-$0.094 & $-$113.000$^{***}$ & $-$0.00001 & $-$0.00001 & 0.011$^{***}$ \\ 
  & (0.00000) & (0.00000) & (0.001) & (0.124) & (0.131) & (33.500) & (0.00001) & (0.00001) & (0.003) \\ 
  & & & & & & & & & \\ 
\hline \\[-1.8ex] 
F & 4.230*** & 4.210*** & 4.130*** & 8.200*** & 7.980*** & 7.840*** & 108.395*** & 106.966*** & 101.779*** \\ 
DF & 3240 & 3210 & 2950 & 3380 & 3340 & 3030 & 3497 & 3405 & 3056 \\ 
Data & Full & >=3 Years & >=5 Years & Full & >=3 Years & >=5 Years & Full & >=3 Years & >=5 Years \\ 
Observations & 4,782 & 4,335 & 3,460 & 4,476 & 4,172 & 3,399 & 4,678 & 4,277 & 3,434 \\ 
R$^{2}$ & 0.035 & 0.035 & 0.041 & 0.030 & 0.031 & 0.055 & 0.310 & 0.312 & 0.489 \\ 
Adjusted R$^{2}$ & $-$0.195 & $-$0.132 & $-$0.078 & $-$0.177 & $-$0.133 & $-$0.061 & 0.152 & 0.194 & 0.426 \\ 
\hline 
\hline \\[-1.8ex] 
\textit{Note:}  & \multicolumn{9}{r}{$^{*}$p$<$0.1; $^{**}$p$<$0.05; $^{***}$p$<$0.01} \\ 
\end{tabular} 
\end{table}

\end{landscape}

\begin{landscape}

\begin{table}[!htbp] \centering 
  \caption{Output of Random Effects and Pooled OLS Models of MFI Outreach} 
  \label{} 
\tiny 
\begin{tabular}{@{\extracolsep{5pt}}lcccccc} 
\\[-1.8ex]\hline 
\hline \\[-1.8ex] 
 & \multicolumn{6}{c}{Dependent Variables} \\ 
\cline{2-7} 
\\[-1.8ex] & \multicolumn{6}{c}{Social Performance} \\ 
 & FemaleClients-Random & FemaleClients-Pooled & AverageLoan- Random & AverageLoan- Pooled & GrossLoans- Random & GrossLoans- Pooled \\ 
\\[-1.8ex] & (1) & (2) & (3) & (4) & (5) & (6)\\ 
\hline \\[-1.8ex] 
 Legal: Bank & $-$0.216$^{***}$ & $-$0.201$^{***}$ & 891.000$^{*}$ & 891.000$^{*}$ & $-$0.039 & $-$0.072 \\ 
  & (0.027) & (0.029) & (468.000) & (468.000) & (0.066) & (0.046) \\ 
  & & & & & & \\ 
 Legal: NBFI & $-$0.162$^{***}$ & $-$0.171$^{***}$ & 271.000 & 271.000 & 0.096$^{*}$ & 0.003 \\ 
  & (0.023) & (0.023) & (339.000) & (339.000) & (0.058) & (0.038) \\ 
  & & & & & & \\ 
 Legal: Coop & $-$0.264$^{***}$ & $-$0.246$^{***}$ & 999.000$^{***}$ & 999.000$^{***}$ & 0.066 & 0.034 \\ 
  & (0.021) & (0.022) & (339.000) & (339.000) & (0.057) & (0.039) \\ 
  & & & & & & \\ 
 Legal: Rural Bank & $-$0.212$^{***}$ & $-$0.215$^{***}$ & 109.000 & 109.000 & $-$0.151 & $-$0.215$^{***}$ \\ 
  & (0.042) & (0.039) & (482.000) & (482.000) & (0.106) & (0.072) \\ 
  & & & & & & \\ 
 Age: Young & $-$0.001 & $-$0.0004 & 192.000 & 192.000 & 0.104$^{***}$ & 0.067$^{**}$ \\ 
  & (0.009) & (0.012) & (293.000) & (293.000) & (0.025) & (0.032) \\ 
  & & & & & & \\ 
 Age: Mature & $-$0.004 & $-$0.015 & $-$253.000 & $-$253.000 & 0.126$^{***}$ & 0.063$^{*}$ \\ 
  & (0.012) & (0.016) & (299.000) & (299.000) & (0.030) & (0.032) \\ 
  & & & & & & \\ 
 Region: SSA & 0.092$^{***}$ & 0.085 & 218.000 & 218.000 & $-$0.081 & $-$0.032 \\ 
  & (0.029) & (0.056) & (646.000) & (646.000) & (0.108) & (0.082) \\ 
  & & & & & & \\ 
 Operating Expense & 0.074$^{***}$ & 0.211$^{***}$ & 224.000 & 224.000 & 0.291$^{***}$ & 0.239$^{***}$ \\ 
  & (0.024) & (0.039) & (909.000) & (909.000) & (0.069) & (0.070) \\ 
  & & & & & & \\ 
 Donations/Assets & $-$0.005 & $-$0.007 & $-$157.000 & $-$157.000 & $-$0.369$^{***}$ & $-$0.550$^{***}$ \\ 
  & (0.024) & (0.038) & (570.000) & (570.000) & (0.082) & (0.085) \\ 
  & & & & & & \\ 
 Capital/Assets & 0.006 & 0.034$^{***}$ & $-$92.400 & $-$92.400 & 0.621$^{***}$ & 0.579$^{***}$ \\ 
  & (0.004) & (0.007) & (148.000) & (148.000) & (0.026) & (0.026) \\ 
  & & & & & & \\ 
 Asset Structure & 0.011 & $-$0.079 & $-$398.000 & $-$398.000 & $-$0.661$^{***}$ & $-$0.763$^{***}$ \\ 
  & (0.054) & (0.090) & (1,961.000) & (1,961.000) & (0.174) & (0.172) \\ 
  & & & & & & \\ 
 Size(Lassets) & $-$0.062$^{*}$ & $-$0.074 & 2,460.000$^{***}$ & 2,460.000$^{***}$ & $-$0.571$^{***}$ & $-$0.250$^{***}$ \\ 
  & (0.037) & (0.047) & (662.000) & (662.000) & (0.090) & (0.083) \\ 
  & & & & & & \\ 
 Education & 0.122$^{**}$ & 0.105$^{*}$ & 265.000 & 265.000 & 0.251$^{**}$ & 0.284$^{***}$ \\ 
  & (0.051) & (0.057) & (880.000) & (880.000) & (0.118) & (0.100) \\ 
  & & & & & & \\ 
 Profit Margin & $-$0.00000 & $-$0.00000 & 0.011 & 0.011 & $-$0.00001 & $-$0.00002$^{*}$ \\ 
  & (0.00000) & (0.00001) & (0.037) & (0.037) & (0.00001) & (0.00001) \\ 
  & & & & & & \\ 
\hline \\[-1.8ex] 
F & 27.3*** & 38.1*** & 89.4*** & 2.63*** & 3017*** & 71.5*** \\ 
Data & Full & Full & >=3 Years & >=3 Years & >=5 Years & >=5 Years \\ 
Observations & 4,782 & 4,782 & 4,476 & 4,476 & 4,678 & 4,678 \\ 
R$^{2}$ & 0.208 & 0.268 & 0.017 & 0.017 & 0.294 & 0.256 \\ 
Adjusted R$^{2}$ & 0.202 & 0.263 & 0.009 & 0.009 & 0.289 & 0.251 \\ 
\hline 
\hline \\[-1.8ex] 
\textit{Note:}  & \multicolumn{6}{r}{$^{*}$p$<$0.1; $^{**}$p$<$0.05; $^{***}$p$<$0.01} \\ 
\end{tabular} 
\end{table}

\end{landscape}

\hypertarget{gross-loans-to-assets-ratio}{%
\subsubsection{Gross Loans to Assets Ratio}\label{gross-loans-to-assets-ratio}}

Gross loans capture the breadth of outreach, the number of people reached and the volume of credit that an MFI disburses. While MFIs should enhance their depth of outreach by reaching female borrowers and micro-borrowers, the sheer scale of such lending also matters \autocite{d2013unsubsidized}. In the best-case scenario, we should have an MFI that reaches the most financially excluded borrowers and offers a higher gross volume of loans, meaning that it reaches more of the financially excluded. The gross loans to assets ratio exhibit a stronger relationship with the independent variables, with a coefficient of determination (adjusted R-Squared) of 0.467. The significant variables are MFI legal form, age, operating expenses to assets, donations to assets, capital-asset ratio, asset structure, size, and profit margin.

Although the legal form of an MFI is marginally significant in driving gross loans, NGOs have the greatest gross loans portfolios than all other legal forms except credit unions/ cooperatives and NBFIs. Hence, although NGOs exhibit more depth, it is not at the expense of breadth. Cooperatives have the highest gross loans, which may reflect their closed nature of serving a limited geographic region or people with common interests who opt to pool savings for their use \autocite{mckillop2011credit}. NBFIs, unlike NGOs, have the advantage of having access to commercial equity and other capital, which, as we see later, positively drives the breadth of outreach in terms of gross loans.

As expected, older firms have more gross loans to assets given their long presence in the market, implying a more significant market share. Size is weakly negatively related to gross loans. These results mean that larger firms have weaker intermediation. The operating expenses to assets ratio positively relate to gross loans. MFIs with a higher spending capacity give out more loans, probably due to their greater market share \autocite{gutierrez2007microfinance}. Capital to assets ratio and profit margin also positively relate to gross loans. In this respect, it appears that MFIs would best achieve broader outreach through commercial organizations that aim to maximize profits. Also, to broaden outreach, equity capital plays a positive role, meaning that commercialization could aid the expansion of gross loans to support the win-win approach to microfinance \autocite{kodongo2013individual}.

On the other hand, there is an inverse relationship between donations and gross loans to assets. In this case, it appears that donors may not be keen on breadth but rather emphasize depth, which research shows is best done through not-for-profit MFIs like NGOs \autocite{d2017ngos,bos2015practice}. MFIs that are more dependent on donations are most likely to be small and young and, hence, the low gross loans to assets. Again, asset structure has a significant negative relationship with gross loans. In this case, MFIs that tie a lot of their resources in physical assets have less breadth of outreach, which is a case for the adoption of information technology in place of brick and mortar branches \autocite{d2017aid}.

A final important aspect of this section is how gross loans to assets relate to measures of depth of outreach. Appendix 6 captures the relationship. While gross loans correlate negatively but weakly with the average loan balance per borrower, there is a substantial positive correlation between gross loans and female borrowers. But examining the scatter plots shows that outliers drive the little correlation between these variables. The positive correlation between per cent of women borrowers and the average loan balance per borrower support the claim that smaller loans indicate deeper outreach \autocite{ayyagari2013financing}. Hence MFIs can pursue both financial inclusion depth and breadth without trade-offs. However, it is not clear at what point the breadth of outreach may negatively affect the depth of outreach, especially in Africa.

\hypertarget{robustness-checks}{%
\subsubsection{Robustness Checks}\label{robustness-checks}}

Our robustness checks encompass three matters. First, the study employs three financial metrics to capture financial inclusion - per cent of women borrowers, average loan balance per borrower, and gross loans to assets ratio. The use of multiple metrics allows for triangulation, given that measuring the extent of financial inclusion is contested with different scholars favouring different metrics. The second aspect relates to outliers which could affect the regression estimates. To control for outliers, we run regressions using winsorized data. Precisely, we remove the top 10\% and the bottom 10\% of the data and run the random effects, fixed effects and pooled OLS. Outliers can bias results when there are extremely large or small values of variables than the typical observation. Overall, the results remain robust to extreme values. Lastly, we correct the standard errors by presenting panel corrected standard errors (PCSE) to cater to serial correlation and cross-sectional dependence, which is a common issue in panel data (see Appendix 5). Under cross-sectional dependence and serial correlation, the observed standard errors are different from the actual standard errors, thereby overestimating or underestimating the model's precision \autocite{pesaran2021general}.

\hypertarget{conclusion-1}{%
\section{Conclusion}\label{conclusion-1}}

MFI provides financial services to the financially excluded, including women, rural dwellers, people living in remote locations, and the poor. A paradigm shift from the NGO not-for-profit model of microfinance to the commercial, for-profit model stresses financial sustainability over and above outreach to the financially excluded. In this article, we have examined microfinance institutions' transformation in Africa and its potential effects on financial inclusion. We found that NGOs perform best in measures of financial depth, represented by the per cent of women borrowers and average loan balance per borrower. Surprisingly, NGOs do well in financial breadth, exhibiting higher median gross loans to assets ratio than other legal forms, including commercial banks. These results suggest that transformation could adversely affect financial inclusion in Africa if allowed to occur without appropriate guides and support.

Furthermore, the capital to assets ratio positively drives all aspects of financial inclusion but is only statistically significant for gross loans. Hence, microfinance institutions, including those not NGOs, could fare well in financial inclusion if affordable and dedicated external capital is available. Interestingly, profitability is positively related to gross loans, although MFI does not need to transform. Operating expenses also positively drive depth and breadth of outreach. Therefore, targeted tax breaks could, for instance, allow MFIs to incur costs of reaching the financially excluded clients without a severe dent in profitability. Donations negatively impact the breadth of outreach while education and regional location are only important in terms of depth, that is, the per cent of female borrowers. Asset structure, donations, and size of an MFI negatively relate to gross loans. Therefore, the transformation of MFIs in Africa needs an appropriate framework to mitigate possible mission drift.

\hypertarget{appendices-1}{%
\section{Appendices}\label{appendices-1}}

\hypertarget{appendix-1-regression-analysis--winsorized-data}{%
\subsection{Appendix 1: Regression Analysis- Winsorized Data}\label{appendix-1-regression-analysis--winsorized-data}}

\begin{landscape}
\newpage

\begin{table}[!htbp] \centering 
  \caption{Regression Analysis Using Winsorized Data} 
  \label{} 
\tiny 
\begin{tabular}{@{\extracolsep{5pt}}lccccccccc} 
\\[-1.8ex]\hline 
\hline \\[-1.8ex] 
 & \multicolumn{9}{c}{Dependent Variables} \\ 
\cline{2-10} 
\\[-1.8ex] & \multicolumn{9}{c}{Social Performance} \\ 
 & FemaleClients & FemaleClients & FemaleClients & AverageLoan & AverageLoan & AverageLoan & GrossLoans & GrossLoans & GrossLoans \\ 
\\[-1.8ex] & (1) & (2) & (3) & (4) & (5) & (6) & (7) & (8) & (9)\\ 
\hline \\[-1.8ex] 
 Legal: Bank & $-$0.223$^{***}$ &  & $-$0.192$^{***}$ & 368.000$^{***}$ &  & 337.000$^{***}$ & $-$0.129$^{***}$ &  & $-$0.129$^{***}$ \\ 
  & (0.028) &  & (0.032) & (72.600) &  & (77.200) & (0.024) &  & (0.024) \\ 
  & & & & & & & & & \\ 
 Legal: NBFI & $-$0.176$^{***}$ &  & $-$0.175$^{***}$ & 233.000$^{***}$ &  & 173.000$^{***}$ & $-$0.007 &  & $-$0.018 \\ 
  & (0.024) &  & (0.026) & (58.300) &  & (59.500) & (0.020) &  & (0.019) \\ 
  & & & & & & & & & \\ 
 Legal: Coop & $-$0.254$^{***}$ & 0.090 & $-$0.219$^{***}$ & 338.000$^{***}$ & $-$99.800 & 333.000$^{***}$ & $-$0.070$^{***}$ & 0.084 & $-$0.063$^{***}$ \\ 
  & (0.022) & (0.088) & (0.026) & (56.400) & (261.000) & (60.900) & (0.019) & (0.093) & (0.020) \\ 
  & & & & & & & & & \\ 
 Legal: Rural Bank & $-$0.202$^{***}$ &  & $-$0.193$^{***}$ & 10.700 &  & $-$3.250 & $-$0.320$^{***}$ &  & $-$0.315$^{***}$ \\ 
  & (0.044) &  & (0.042) & (92.400) &  & (92.100) & (0.034) &  & (0.032) \\ 
  & & & & & & & & & \\ 
 Age: Young & $-$0.008 & $-$0.009 & $-$0.002 & $-$23.400 & $-$16.700 & $-$37.000 & 0.038$^{***}$ & 0.042$^{***}$ & 0.046$^{***}$ \\ 
  & (0.010) & (0.010) & (0.014) & (24.900) & (26.800) & (34.800) & (0.010) & (0.011) & (0.012) \\ 
  & & & & & & & & & \\ 
 Age: Mature & $-$0.010 & $-$0.010 & $-$0.013 & $-$54.500 & $-$46.800 & $-$74.200$^{*}$ & 0.032$^{***}$ & 0.037$^{**}$ & 0.027$^{*}$ \\ 
  & (0.013) & (0.015) & (0.017) & (34.700) & (41.600) & (42.200) & (0.012) & (0.015) & (0.014) \\ 
  & & & & & & & & & \\ 
 Region: SSA & 0.102$^{***}$ &  & 0.088 & 65.200 &  & 26.100 & $-$0.103$^{***}$ &  & $-$0.087$^{**}$ \\ 
  & (0.035) &  & (0.058) & (79.300) &  & (128.000) & (0.036) &  & (0.043) \\ 
  & & & & & & & & & \\ 
 Operating Expense & 0.171$^{***}$ & 0.098$^{*}$ & 0.369$^{***}$ & $-$359.000$^{***}$ & $-$320.000$^{**}$ & $-$324.000$^{*}$ & 0.219$^{***}$ & 0.280$^{***}$ & 0.177$^{***}$ \\ 
  & (0.047) & (0.053) & (0.073) & (121.000) & (145.000) & (168.000) & (0.044) & (0.054) & (0.056) \\ 
  & & & & & & & & & \\ 
 Donations/Assets & 0.392$^{***}$ & 0.433$^{***}$ & 0.525$^{**}$ & 35.100 & 319.000 & $-$2,161.000$^{***}$ & 0.103 & 0.078 & 0.111 \\ 
  & (0.145) & (0.147) & (0.253) & (347.000) & (353.000) & (557.000) & (0.141) & (0.143) & (0.204) \\ 
  & & & & & & & & & \\ 
 Capital/Assets & 0.041$^{**}$ & 0.042$^{**}$ & 0.059$^{*}$ & $-$54.100 & $-$21.200 & $-$153.000$^{**}$ & $-$0.062$^{***}$ & $-$0.088$^{***}$ & 0.012 \\ 
  & (0.020) & (0.022) & (0.032) & (51.600) & (56.200) & (75.600) & (0.019) & (0.022) & (0.025) \\ 
  & & & & & & & & & \\ 
 Asset Structure & 0.030 & 0.149 & $-$0.284$^{*}$ & $-$98.700 & $-$172.000 & 104.000 & $-$0.425$^{***}$ & $-$0.318$^{***}$ & $-$0.607$^{***}$ \\ 
  & (0.090) & (0.098) & (0.162) & (241.000) & (265.000) & (388.000) & (0.089) & (0.101) & (0.125) \\ 
  & & & & & & & & & \\ 
 Size(Lassets) & 0.016 & 0.121 & $-$0.084 & 934.000$^{***}$ & 630.000$^{***}$ & 1,117.000$^{***}$ & 0.027 & 0.098 & $-$0.009 \\ 
  & (0.057) & (0.089) & (0.069) & (146.000) & (241.000) & (166.000) & (0.049) & (0.089) & (0.054) \\ 
  & & & & & & & & & \\ 
 Education & 0.096$^{*}$ & 0.077 & 0.121$^{*}$ & 303.000$^{*}$ & 293.000 & 143.000 & $-$0.030 & $-$0.056 & $-$0.002 \\ 
  & (0.057) & (0.083) & (0.065) & (155.000) & (225.000) & (163.000) & (0.049) & (0.080) & (0.050) \\ 
  & & & & & & & & & \\ 
 Profit Margin & 0.014 & 0.013 & 0.027 & 27.800 & 38.700 & 15.100 & 0.073$^{***}$ & 0.068$^{***}$ & 0.094$^{***}$ \\ 
  & (0.010) & (0.010) & (0.018) & (27.200) & (28.300) & (43.800) & (0.010) & (0.011) & (0.015) \\ 
  & & & & & & & & & \\ 
\hline \\[-1.8ex] 
F & 4.230*** & 4.210*** & 4.130*** & 8.200*** & 7.980*** & 7.840*** & 108.395*** & 106.966*** & 101.779*** \\ 
Model & Random & Fixed & Pooled & Random & Fixed & Pooled & Random & Fixed & Pooled \\ 
Observations & 3,925 & 3,925 & 3,925 & 3,466 & 3,466 & 3,466 & 3,804 & 3,804 & 3,804 \\ 
R$^{2}$ & 0.238 & 0.036 & 0.247 & 0.220 & 0.189 & 0.253 & 0.278 & 0.106 & 0.211 \\ 
Adjusted R$^{2}$ & 0.231 & $-$0.220 & 0.240 & 0.213 & $-$0.010 & 0.246 & 0.272 & $-$0.122 & 0.204 \\ 
\hline 
\hline \\[-1.8ex] 
\textit{Note:}  & \multicolumn{9}{r}{$^{*}$p$<$0.1; $^{**}$p$<$0.05; $^{***}$p$<$0.01} \\ 
\end{tabular} 
\end{table}

\end{landscape}
\newpage

\hypertarget{appendix-2-hausmann-test-fixed-versus-random-effects}{%
\subsection{Appendix 2: Hausmann Test; Fixed versus Random Effects}\label{appendix-2-hausmann-test-fixed-versus-random-effects}}

In this section, we run the Hausmann test to choose between fixed effects and the random-effects model. Also, we check for the choice between pooled OLS and random-effects models. Finally, we present the output from the regression analysis. Appendix 2 shows the results of the Hausmann test. The test favours the fixed effects model, given that the null hypothesis is the random effects.

\begin{table}[!h]

\caption{\label{tab:unnamed-chunk-82}Results of the Hausmann Test for Fixed versus Random Effects}
\centering
\fontsize{9}{11}\selectfont
\begin{tabular}[t]{lrrrl}
\toprule
Dependent\_variable & Statistic & P.value & Parameter & Alternative\\
\midrule
Percent of Female Borrowers & 37.5 & 0 & 10 & one model is inconsistent\\
Average Loan Balance per Borrower & 62.7 & 0 & 10 & one model is inconsistent\\
Gross Loan Portfolio to Total Assets & 116.0 & 0 & 10 & one model is inconsistent\\
\bottomrule
\multicolumn{5}{l}{\rule{0pt}{1em}Source: Authors' construction}\\
\multicolumn{5}{l}{\rule{0pt}{1em}\textit{Notes: }}\\
\multicolumn{5}{l}{\rule{0pt}{1em}\textsuperscript{1} The test favours the fixed effects model}\\
\end{tabular}
\end{table}

\hypertarget{appendix-3-f-test-fixed-effects-vs-pooled-ols}{%
\subsection{Appendix 3: F-Test; Fixed Effects vs Pooled OLS}\label{appendix-3-f-test-fixed-effects-vs-pooled-ols}}

The table below shows significant effects, and the test favours the fixed-effects model over the pooled OLS.

\begin{table}[!h]

\caption{\label{tab:unnamed-chunk-83}Results of the F test for individual effects for Fixed Effects versus Pooled OLS}
\centering
\fontsize{9}{11}\selectfont
\begin{tabular}[t]{llll}
\toprule
Dependent\_variable & statistic & Method & Alternative\\
\midrule
Percent of women borrowers & 23.0000*** & F test for individual effects & Significant effects\\
Average Loan Balance per Borrower & 1.0000*** & F test for individual effects & Significant effects\\
Gross Loans to Total Assets & 5.0000*** & F test for individual effects & Significant effects\\
\bottomrule
\multicolumn{4}{l}{\rule{0pt}{1em}Source: Authors' construction}\\
\multicolumn{4}{l}{\rule{0pt}{1em}\textit{Notes: }}\\
\multicolumn{4}{l}{\rule{0pt}{1em}\textsuperscript{1} The test favours the fixed effects model over pooled OLS}\\
\end{tabular}
\end{table}

\hypertarget{appendix-4-lm-test-random-effects-vs-pooled-ols}{%
\subsection{Appendix 4: LM Test; Random Effects vs Pooled OLS}\label{appendix-4-lm-test-random-effects-vs-pooled-ols}}

Again, the table below shows the Langrage multiplier test results that favour the random effects model over the pooled OLS.

\begin{table}[!h]

\caption{\label{tab:unnamed-chunk-84}Results of the Langrage Multiplier Test for Random Effects versus Pooled OLS}
\centering
\fontsize{9}{11}\selectfont
\begin{tabular}[t]{lll}
\toprule
Dependent\_variable & statistic & Alternative\\
\midrule
Percent of Women Borrowers & 85.5000*** & Significant effects\\
Average Loan Balance per Borrower & 2.2400*** & Significant effects\\
Gross Loans to Total Assets & 22.3000*** & Significant effects\\
\bottomrule
\multicolumn{3}{l}{\rule{0pt}{1em}Source: Authors' construction}\\
\multicolumn{3}{l}{\rule{0pt}{1em}\textit{Notes: }}\\
\multicolumn{3}{l}{\rule{0pt}{1em}\textsuperscript{1} The test favours the fixed effects model over pooled OLS}\\
\multicolumn{3}{l}{\rule{0pt}{1em}\textsuperscript{2} Lagrange Multiplier Test - (Honda) for unbalanced panels}\\
\end{tabular}
\end{table}

\hypertarget{appendix-5-cross-sectional-dependence}{%
\subsection{Appendix 5: Cross-Sectional Dependence}\label{appendix-5-cross-sectional-dependence}}

The table below shows that there is high cross-sectional dependence in the dataset. For this reason, we run and present the panel corrected standard errors.

\begin{table}[!h]

\caption{\label{tab:unnamed-chunk-85}Results of the PCD Test for Cross-Sectional Dependence}
\centering
\fontsize{9}{11}\selectfont
\begin{tabular}[t]{llll}
\toprule
Dependent\_variable & Model & Chisq & df\\
\midrule
Percent of women borrowers & Fixed Effects & 99564*** & 53076\\
Average loan balance per borrower & Fixed Effects & 0.0000*** & 59759\\
Gross loans to total assets & Fixed Effects & 0.0000*** & 68598\\
Percent of women borrowers & Random effects & 99764*** & 53076\\
Average loan balance per borrower & Random Effects & 0.0000*** & 59759\\
\addlinespace
Gross loans to total assets & Random effects & 0.0000*** & 68598\\
\bottomrule
\multicolumn{4}{l}{\rule{0pt}{1em}Source: Authors' construction}\\
\multicolumn{4}{l}{\rule{0pt}{1em}\textit{Notes: }}\\
\multicolumn{4}{l}{\rule{0pt}{1em}\textsuperscript{1} The test shows the existence of cross-sectional dependence}\\
\end{tabular}
\end{table}

\newpage

\begin{landscape}

\hypertarget{appendix-6-correlation-matrix-for-dependent-variables}{%
\subsection{Appendix 6: Correlation Matrix for Dependent Variables}\label{appendix-6-correlation-matrix-for-dependent-variables}}

\begin{figure}
\centering
\includegraphics{_main_files/figure-latex/unnamed-chunk-86-1.pdf}
\caption{\label{fig:unnamed-chunk-86}Correlation Between Gross Loans to Assets, Average Loan Balance per Borrower, and Percent of Female Borrowers}
\end{figure}

\end{landscape}

\begin{landscape}

\newpage

\hypertarget{appendix-7-residuals-diagnostics--full-data}{%
\subsection{Appendix 7: Residuals Diagnostics- Full Data}\label{appendix-7-residuals-diagnostics--full-data}}

\begin{figure}
\centering
\includegraphics{_main_files/figure-latex/unnamed-chunk-87-1.pdf}
\caption{\label{fig:unnamed-chunk-87}Normal QQ Plots for the Fixed and Radom Effects Regression Models}
\end{figure}

\end{landscape}

\begin{landscape}

\newpage

\hypertarget{appendix-8-residuals-diagnostics--winsorised-data}{%
\subsection{Appendix 8: Residuals Diagnostics- Winsorised Data}\label{appendix-8-residuals-diagnostics--winsorised-data}}

\begin{figure}
\centering
\includegraphics{_main_files/figure-latex/unnamed-chunk-88-1.pdf}
\caption{\label{fig:unnamed-chunk-88}Normal QQ Plots for Regressions Using Winsorized Data}
\end{figure}

\end{landscape}

\begin{landscape}

\newpage

\hypertarget{appendix-9-debt-to-equity-ratio-by-mfi-legal-status}{%
\subsection{Appendix 9: Debt to Equity Ratio by MFI Legal Status}\label{appendix-9-debt-to-equity-ratio-by-mfi-legal-status}}

\begin{figure}
\centering
\includegraphics{_main_files/figure-latex/unnamed-chunk-89-1.pdf}
\caption{\label{fig:unnamed-chunk-89}Debt to Equity Ratio by MFI Legal Status}
\end{figure}

\end{landscape}

\begin{landscape}

\newpage

\hypertarget{appendix-10-trends-in-the-percent-of-female-borrowers}{%
\subsection{Appendix 10: Trends in the Percent of Female Borrowers}\label{appendix-10-trends-in-the-percent-of-female-borrowers}}

\begin{figure}
\centering
\includegraphics{_main_files/figure-latex/unnamed-chunk-90-1.pdf}
\caption{\label{fig:unnamed-chunk-90}Trends in the Percent of Female Borrowers}
\end{figure}

\end{landscape}

\begin{savequote}
There is grandeur in this view of life, with its several powers, having
been originally breathed into a few forms or into one; and that, whilst
this planet has gone cycling on according to the fixed law of gravity,
from so simple a beginning endless forms most beautiful and most
wonderful have been, and are being, evolved.
\qauthor{--- Charles Darwin \autocite{Darwin1859}}\end{savequote}



\hypertarget{customisations-and-extensions}{%
\chapter{Customisations and extensions}\label{customisations-and-extensions}}

\minitoc 

\noindent This chapter describes a number of additional tips and tricks as well as possible customizations to the \texttt{oxforddown} thesis.

\hypertarget{front-matter}{%
\section{Front matter}\label{front-matter}}

\hypertarget{shorten-captions-shown-in-the-list-of-figures-pdf}{%
\subsection{Shorten captions shown in the list of figures (PDF)}\label{shorten-captions-shown-in-the-list-of-figures-pdf}}

You might want your list of figures (which follows the table of contents) to have shorter (or just different) figure descriptions than the actual figure captions.

Do this using the chunk option \texttt{fig.scap} (`short caption'), for example \texttt{\{r\ captain-image,\ fig.cap="A\ very\ long\ and\ descriptive\ (and\ potentially\ boring)\ caption\ that\ doesn\textquotesingle{}t\ fit\ in\ the\ list\ of\ figures,\ but\ helps\ the\ reader\ understand\ what\ the\ figure\ communicates.",\ fig.scap="A\ concise\ description\ for\ the\ list\ of\ figures"}

\hypertarget{shorten-captions-shown-in-the-list-of-tables-pdf}{%
\subsection{Shorten captions shown in the list of tables (PDF)}\label{shorten-captions-shown-in-the-list-of-tables-pdf}}

You might want your list of tables (which follows the list of figures in your thesis front matter) to have shorter (or just different) table descriptions than the actual table captions.

If you are using \texttt{knitr::kable} to generate a table, you can do this with the argument \texttt{caption.short}, e.g.:

\begin{Shaded}
\begin{Highlighting}[]
\NormalTok{knitr}\SpecialCharTok{::}\FunctionTok{kable}\NormalTok{(mtcars,}
              \AttributeTok{caption =} \StringTok{"A very long and descriptive (and potentially}
\StringTok{              boring) caption that doesn\textquotesingle{}t fit in the list of figures,}
\StringTok{              but helps the reader understand what the figure }
\StringTok{              communicates."}\NormalTok{,}
              \AttributeTok{caption.short =} \StringTok{"A concise description for the list of tables"}\NormalTok{)}
\end{Highlighting}
\end{Shaded}

\hypertarget{shorten-running-header-pdf}{%
\section{Shorten running header (PDF)}\label{shorten-running-header-pdf}}

You might want a chapter's running header (i.e.~the header showing the title of the current chapter at the top of page) to be shorter (or just different) to the actual chapter title.

Do this by adding the latex command \texttt{\textbackslash{}chaptermark\{My\ shorter\ version\}} after your chapter title.

For example, chapter \ref{cites-and-refs}`s running header is simply 'Cites and cross-refs', because it begins like this:

\begin{Shaded}
\begin{Highlighting}[]
\FunctionTok{\# Citations, cross{-}references, and collaboration \{\#cites{-}and{-}refs\} }
\NormalTok{\textbackslash{}chaptermark\{Cites and cross{-}refs\}}
\end{Highlighting}
\end{Shaded}

\hypertarget{unnumbered-chapters}{%
\section{Unnumbered chapters}\label{unnumbered-chapters}}

To make chapters unnumbered (normally only relevant to the Introduction and/or the Conclusion), follow the chapter header with \texttt{\{-\}}, e.g.~\texttt{\#\ Introduction\ \{-\}}.

When you do this, you must also follow the heading with these two latex commands:

\begin{Shaded}
\begin{Highlighting}[]
\FunctionTok{\textbackslash{}adjustmtc}
\FunctionTok{\textbackslash{}markboth}\NormalTok{\{The Name of Your Unnumbered Chapter\}\{\}}
\end{Highlighting}
\end{Shaded}

Otherwise the chapter's mini table of contents and the running header will show the previous chapter.

\hypertarget{beginning-chapters-with-quotes-pdf}{%
\section{Beginning chapters with quotes (PDF)}\label{beginning-chapters-with-quotes-pdf}}

The OxThesis LaTeX template lets you inject some wittiness into your thesis by including a block of type \texttt{savequote} at the beginning of chapters.
To do this, use the syntax \texttt{\textasciigrave{}\textasciigrave{}\textasciigrave{}\{block\ type=\textquotesingle{}savequote\textquotesingle{}\}}.\footnote{For more on custom block types, see the relevant section in \href{https://bookdown.org/yihui/bookdown/custom-blocks.html}{\emph{Authoring Books with R Markdown}}.}

Add the reference for the quote with the chunk option \texttt{quote\_author="my\ author\ name"}.
You will also want to add the chunk option \texttt{include=knitr::is\_latex\_output()} so that quotes are only included in PDF output.

It's not possible to use markdown syntax inside chunk options, so if you want to e.g.~italicise a book name in the reference use a \href{https://bookdown.org/yihui/bookdown/markdown-extensions-by-bookdown.html\#text-references}{`text reference'}: Create a named piece of text with `(ref:label-name) My text', then point to this in the chunk option with \texttt{quote\_author=\textquotesingle{}(ref:label-name)\textquotesingle{}}.

\hypertarget{highlighting-corrections-html-pdf}{%
\section{Highlighting corrections (HTML \& PDF)}\label{highlighting-corrections-html-pdf}}

For when it comes time to do corrections, you may want to highlight changes made when you submit a post-viva, corrected copy to your examiners so they can quickly verify you've completed the task.
You can do so like this:

\hypertarget{short-inline-corrections}{%
\subsection{Short, inline corrections}\label{short-inline-corrections}}

Highlight \textbf{short, inline corrections} by doing \texttt{{[}like\ this{]}\{.correction\}} --- the text between the square brackets will then be highlighted in blue in the output.

Note that pandoc might get confused by citations and cross-references inside inline corrections.
In particular, it might get confused by \texttt{"{[}what\ @Shea2014\ said{]}\{.correction\}"} which becomes \autocite[what][ said]{Shea2014}\{.correction\}
In such cases, you can use LaTeX syntax directly.
The correction highlighting uses the \href{https://ctan.org/pkg/soul}{soul} package, so you can do like this:

\begin{itemize}
\tightlist
\item
  If using biblatex for references, use \texttt{"\textbackslash{}hl\{what\ \textbackslash{}textcite\{Shea2014\}\ said\}}
\item
  If using natbib for references, use \texttt{"\textbackslash{}hl\{what\ \textbackslash{}cite\{Shea2014\}\ said\}}
\end{itemize}

Using raw LaTeX has the drawback of corrections then not showing up in HTML output at all, but you might only care about correction highlighting in the PDF for your examiners anyway!

\hypertarget{blocks-of-added-or-changed-material}{%
\subsection{Blocks of added or changed material}\label{blocks-of-added-or-changed-material}}

Highlight entire \textbf{blocks of added or changed material} by putting them in a block of type \texttt{correction}, using the syntax \texttt{\textasciigrave{}\textasciigrave{}\textasciigrave{}\{block\ type=\textquotesingle{}correction\textquotesingle{}\}}.\footnote{In the \textbf{.tex} file for PDF output, this will put the content between \texttt{\textbackslash{}begin\{correction\}} and \texttt{\textbackslash{}end\{correction\}}; in gitbook output it will be put between \texttt{\textless{}div\ class="correction"\textgreater{}} and \texttt{\textless{}/div\textgreater{}}.}
Like so:

\begin{correction}
For larger chunks, like this paragraph or indeed entire figures, you can
use the \texttt{correction} block type. This environment
\textbf{highlights paragraph-sized and larger blocks} with the same blue
colour.
\end{correction}

\emph{Note that correction blocks cannot be included in word output.}

\hypertarget{stopping-corrections-from-being-highlighted}{%
\subsection{Stopping corrections from being highlighted}\label{stopping-corrections-from-being-highlighted}}

To turn off correction highlighting, go to the YAML header of \textbf{index.Rmd}, then:

\begin{itemize}
\tightlist
\item
  PDF output: set \texttt{corrections:\ false}\\
\item
  HTML output: remove or comment out \texttt{-\ templates/corrections.css}
\end{itemize}

\hypertarget{apply-custom-font-color-and-highlighting-to-text-html-pdf}{%
\section{Apply custom font color and highlighting to text (HTML \& PDF)}\label{apply-custom-font-color-and-highlighting-to-text-html-pdf}}

The lua filter that adds the functionality to highlight corrections adds two more tricks:
you can apply your own choice of colour to highlight text, or change the font color.
The syntax is as follows:

\begin{quote}
Here's \texttt{{[}some\ text\ in\ pink\ highlighting{]}\{highlight="pink"\}}\\
Becomes: Here's \sethlcolor{pink}\hl{some text in pink highlighting}\sethlcolor{correctioncolor}.
\end{quote}

\begin{quote}
\texttt{{[}Here\textquotesingle{}s\ some\ text\ with\ blue\ font{]}\{color="blue"\}}\strut \\
Becomes: \textcolor{blue}{Here's some text with blue font}
\end{quote}

\begin{quote}
Finally --- never, ever actually do this -- \texttt{{[}here\textquotesingle{}s\ some\ text\ with\ black\ highlighting\ and\ yellow\ font{]}\{highlight="black"\ color="yellow"\}}\\
Becomes: \textcolor{yellow}{\sethlcolor{black}\hl{here's some text with black highlighting and yellow font}\sethlcolor{correctioncolor}}
\end{quote}

The file \textbf{scripts\_and\_filters/colour\_and\_highlight.lua} implements this, if you want to fiddle around with it.
It works with both PDF and HTML output.

\hypertarget{embed-pdf}{%
\section{Including another paper in your thesis - embed a PDF document}\label{embed-pdf}}

You may want to embed existing PDF documents into the thesis, for example if your department allows a `portfolio' style thesis and you need to include an existing typeset publication as a chapter.

In gitbook output, you can simply use \texttt{knitr::include\_graphics} and it should include a scrollable (and downloadable) PDF.
You will probably want to set the chunk options \texttt{out.width=\textquotesingle{}100\%\textquotesingle{}} and \texttt{out.height=\textquotesingle{}1000px\textquotesingle{}}:

\begin{Shaded}
\begin{Highlighting}[]
\NormalTok{knitr}\SpecialCharTok{::}\FunctionTok{include\_graphics}\NormalTok{(}\StringTok{"figures/sample{-}content/pdf\_embed\_example/Lyngs2020\_FB.pdf"}\NormalTok{)}
\end{Highlighting}
\end{Shaded}

In LaTeX output, however, this approach can cause odd behaviour.
Therefore, when you build your thesis to PDF, split the PDF into an alphanumerically sorted sequence of \textbf{single-page} PDF files (you can do this automatically with the package \texttt{pdftools}). You can then use the appropriate LaTeX command to insert them, as shown below (for brevity, in the \texttt{oxforddown} PDF sample content we're only including two pages).
\emph{Note that the chunk option \texttt{results=\textquotesingle{}asis\textquotesingle{}} must be set.}
You may also want to remove margins from the PDF files, which you can do with Adobe Acrobat (paid version) and likely other software.

\begin{Shaded}
\begin{Highlighting}[]
\CommentTok{\# install.packages(pdftools)}
\CommentTok{\# split PDF into pages stored in}
\NormalTok{    figures}\SpecialCharTok{/}\NormalTok{sample}\SpecialCharTok{{-}}\NormalTok{content}\SpecialCharTok{/}\NormalTok{pdf\_embed\_example}\SpecialCharTok{/}\NormalTok{split}\SpecialCharTok{/}
\CommentTok{\#}
\NormalTok{    pdftools}\SpecialCharTok{::}\FunctionTok{pdf\_split}\NormalTok{(}\StringTok{"figures/sample{-}content/pdf\_embed\_example/Lyngs2020\_FB.pdf"}\NormalTok{,}
\CommentTok{\# output = "figures/sample{-}content/pdf\_embed\_example/split/")}

\CommentTok{\# grab the pages}
\NormalTok{pages }\OtherTok{\textless{}{-}} \FunctionTok{list.files}\NormalTok{(}\StringTok{"figures/sample{-}content/pdf\_embed\_example/split"}\NormalTok{,}
    \AttributeTok{full.names =} \ConstantTok{TRUE}\NormalTok{)}

\CommentTok{\# set how wide you want the inserted PDFs to be:}
\CommentTok{\# 1.0 is 100 per cent of the oxforddown PDF page width;}
\CommentTok{\# you may want to make it a bit bigger}
\NormalTok{pdf\_width }\OtherTok{\textless{}{-}} \FloatTok{1.2}

\CommentTok{\# for each PDF page, insert it nicely and}
\CommentTok{\# end with a page break}
\FunctionTok{cat}\NormalTok{(stringr}\SpecialCharTok{::}\FunctionTok{str\_c}\NormalTok{(}\StringTok{"}\SpecialCharTok{\textbackslash{}\textbackslash{}}\StringTok{newpage }\SpecialCharTok{\textbackslash{}\textbackslash{}}\StringTok{begin\{center\}}
\StringTok{    }\SpecialCharTok{\textbackslash{}\textbackslash{}}\StringTok{makebox[}\SpecialCharTok{\textbackslash{}\textbackslash{}}\StringTok{linewidth][c]\{}\SpecialCharTok{\textbackslash{}\textbackslash{}}\StringTok{includegraphics[width="}\NormalTok{, pdf\_width,}
    \StringTok{"}\SpecialCharTok{\textbackslash{}\textbackslash{}}\StringTok{linewidth]\{"}\NormalTok{, pages, }\StringTok{"\}\} }\SpecialCharTok{\textbackslash{}\textbackslash{}}\StringTok{end\{center\}"}\NormalTok{))}
\end{Highlighting}
\end{Shaded}

\newpage \begin{center} \makebox[\linewidth][c]{\includegraphics[width=1.2\linewidth]{figures/sample-content/pdf_embed_example/split/_000000000000001.pdf}} \end{center} \newpage \begin{center} \makebox[\linewidth][c]{\includegraphics[width=1.2\linewidth]{figures/sample-content/pdf_embed_example/split/_000000000000011.pdf}} \end{center}

\hypertarget{embed-rmd}{%
\section{Including another paper in your thesis - R Markdown child document}\label{embed-rmd}}

Sometimes you want to include another paper you are currently writing as a chapter in your thesis.
Above \ref{embed-pdf}, we described the simplest way to do this: include the other paper as a pdf.
However, in some cases you instead want to include the R Markdown source from this paper, and have it compiled within your thesis.
This is a little bit more tricky, because you need to keep careful track of your file paths, but it is possible by \href{https://bookdown.org/yihui/rmarkdown-cookbook/child-document.html}{including the paper as a child document}.
There are four main steps:

\begin{enumerate}
\def\labelenumi{\arabic{enumi}.}
\tightlist
\item
  Include the paper as a child document
\item
  Make file paths compatible with knitting the article on its own, as well as when it's include in your thesis
\item
  Make header levels correct
\item
  Make figure widths correct
\end{enumerate}

\hypertarget{an-example-paper-in-another-folder}{%
\subsection{An example paper in another folder}\label{an-example-paper-in-another-folder}}

Take this simple example (files for this are in \href{https://github.com/ulyngs/oxforddown-external-article}{this GitHub repository}):

\begin{Shaded}
\begin{Highlighting}[]
\NormalTok{|{-}{-}paper\_to\_include}
\NormalTok{|  |{-}{-}my\_paper.Rmd}
\NormalTok{|  |{-}{-}data}
\NormalTok{|  |  |{-}{-}cat\_salt.csv}
\NormalTok{|  |{-}{-}figures}
\NormalTok{|  |  |{-}{-}cat.jpg}
\NormalTok{|}
\NormalTok{|{-}{-}thesis}
\end{Highlighting}
\end{Shaded}

As the chart suggests, you have another folder, \textbf{paper\_to\_include/} living in the same containing folder as your thesis folder.
In the \textbf{paper\_to\_include} folder, the file \textbf{my\_paper.Rmd} is where you write the paper.
In \textbf{my\_paper.Rmd}, you read in a CSV file found in the subfolder \textbf{data/cats.csv}, and also an image from the subfolder \textbf{figures/cat.jpg}.

\hypertarget{step-1-include-paper-as-a-child-document}{%
\subsection{Step 1: Include paper as a child document}\label{step-1-include-paper-as-a-child-document}}

In your thesis folder, create an Rmd file for the chapter where you want to include another paper.
Add one or more code chunks that include R Markdown files from that paper as child documents:

\begin{Shaded}
\begin{Highlighting}[]
\FunctionTok{\# Including an external chapter }

\InformationTok{\textasciigrave{}\textasciigrave{}\textasciigrave{}\{r child = "../paper\_to\_include/my\_paper.Rmd"\}}
\InformationTok{\textasciigrave{}\textasciigrave{}\textasciigrave{}}
\end{Highlighting}
\end{Shaded}

\hypertarget{step-2-make-file-paths-compatible}{%
\subsection{Step 2: Make file paths compatible}\label{step-2-make-file-paths-compatible}}

Use \href{https://rmarkdown.rstudio.com/lesson-6.html}{parameters} to adjust the file path of images based on values you set in the YAML header of an R Markdown file.
In \textbf{my\_paper.Rmd}, create a parameter called \texttt{other\_path} and set it to an empty string:

\begin{Shaded}
\begin{Highlighting}[]
\PreprocessorTok{{-}{-}{-}}
\FunctionTok{title}\KeywordTok{:}\AttributeTok{ }\StringTok{"A fabulous article in a different folder"}
\FunctionTok{params}\KeywordTok{:}
\AttributeTok{  }\FunctionTok{other\_path}\KeywordTok{:}\AttributeTok{ }\StringTok{""}
\PreprocessorTok{{-}{-}{-}}
\end{Highlighting}
\end{Shaded}

In \textbf{my\_paper.Rmd}, put this at the start of the filepath when you read in data or include images:

\begin{Shaded}
\begin{Highlighting}[]
\FunctionTok{library}\NormalTok{(tidyverse)}
\FunctionTok{library}\NormalTok{(knitr)}

\NormalTok{cat\_data }\OtherTok{\textless{}{-}} \FunctionTok{read\_csv}\NormalTok{(}\FunctionTok{str\_c}\NormalTok{(params}\SpecialCharTok{$}\NormalTok{other\_path, }\StringTok{"data/cats.csv"}\NormalTok{))}
\FunctionTok{include\_graphics}\NormalTok{(}\FunctionTok{str\_c}\NormalTok{(params}\SpecialCharTok{$}\NormalTok{other\_path, }\StringTok{"figures/cat.jpg"}\NormalTok{))}
\end{Highlighting}
\end{Shaded}

Finally, in your thesis folder's \textbf{index.Rmd} file, also create the parameter \texttt{other\_path}.
But here, set it to where the \textbf{paper\_to\_include/} folder is relative to your thesis folder:

\begin{Shaded}
\begin{Highlighting}[]
\FunctionTok{params}\KeywordTok{:}
\AttributeTok{  }\FunctionTok{other\_path}\KeywordTok{:}\AttributeTok{ }\StringTok{"../paper\_to\_include/"}
\end{Highlighting}
\end{Shaded}

\hypertarget{note-on-html-output}{%
\subsubsection{Note on HTML output}\label{note-on-html-output}}

Note that if you want to host an HTML version on your thesis online, you will need to include graphics in the content that you host online - the internet obviously won't be able to see filepaths that are just referring to stuff in another folder on your computer!

\hypertarget{step-3-make-sure-header-levels-are-correct}{%
\subsection{Step 3: Make sure header levels are correct}\label{step-3-make-sure-header-levels-are-correct}}

Unless the paper you want to include is also written as a book, your header levels are probably going to be off.
That is, the level 1 headers (\# Some header) you use for main sections in the other paper turns into chaper titles when included in your thesis.

To avoid this, first \emph{increment all heading levels by one in \textbf{paper\_to\_include/my\_paper.Rmd}} (\# Some header -\textgreater{} \#\# Some header).
Then in \textbf{paper\_to\_include/} create a \href{https://bookdown.org/yihui/rmarkdown-cookbook/lua-filters.html\#lua-filters}{lua filter} that decrements header levels by one: Create a text file, save it as \textbf{reduce\_header\_level.lua}, and give it the content below.

\begin{Shaded}
\begin{Highlighting}[]
\KeywordTok{function}\NormalTok{ Header}\OperatorTok{(}\NormalTok{el}\OperatorTok{)}
  \ControlFlowTok{if} \OperatorTok{(}\NormalTok{el}\OperatorTok{.}\NormalTok{level }\OperatorTok{\textless{}=} \DecValTok{1}\OperatorTok{)} \ControlFlowTok{then}
    \FunctionTok{error}\OperatorTok{(}\StringTok{"I don\textquotesingle{}t know how to decrease the level of h1"}\OperatorTok{)}
  \ControlFlowTok{end}
\NormalTok{  el}\OperatorTok{.}\NormalTok{level }\OperatorTok{=}\NormalTok{ el}\OperatorTok{.}\NormalTok{level }\OperatorTok{{-}} \DecValTok{1}
  \ControlFlowTok{return}\NormalTok{ el}
\KeywordTok{end}
\end{Highlighting}
\end{Shaded}

In the YAML header of \textbf{paper\_to\_include/my\_paper.Rmd}, use this filter:

\begin{Shaded}
\begin{Highlighting}[]
\PreprocessorTok{{-}{-}{-}}
\FunctionTok{title}\KeywordTok{:}\AttributeTok{ }\StringTok{"A fabulous article in a different folder"}
\FunctionTok{params}\KeywordTok{:}
\AttributeTok{  }\FunctionTok{other\_path}\KeywordTok{:}\AttributeTok{ }\StringTok{""}
\FunctionTok{output}\KeywordTok{:}
\AttributeTok{  }\FunctionTok{pdf\_document}\KeywordTok{:}\AttributeTok{ }
\AttributeTok{    }\FunctionTok{pandoc\_args}\KeywordTok{:}\AttributeTok{ }\KeywordTok{[}\StringTok{"{-}{-}lua{-}filter=reduce\_header\_level.lua"}\KeywordTok{]}
\PreprocessorTok{{-}{-}{-}}
\end{Highlighting}
\end{Shaded}

Now, your header levels will be correct both when you knit the paper on its own and when its included in your thesis.

\hypertarget{step-4.-make-sure-figure-widths-are-correct}{%
\subsection{Step 4. Make sure figure widths are correct}\label{step-4.-make-sure-figure-widths-are-correct}}

It might be that your figure widths when knitting your paper on its own, and when including it in your thesis, need to be different.
You can again use parameters to set figure widths.

Imagine you want figure width to be 80\% of the page width when knitting your paper on its own, but 100\% in your thesis.
In \textbf{paper\_to\_include/my\_paper.Rmd}, first add a parameter we could call \texttt{out\_width} and set it to the string ``80\%'':

\begin{Shaded}
\begin{Highlighting}[]
\PreprocessorTok{{-}{-}{-}}
\FunctionTok{title}\KeywordTok{:}\AttributeTok{ }\StringTok{"A fabulous article in a different folder"}
\FunctionTok{params}\KeywordTok{:}
\AttributeTok{  }\FunctionTok{other\_path}\KeywordTok{:}\AttributeTok{ }\StringTok{""}
\AttributeTok{  }\FunctionTok{out\_width}\KeywordTok{:}\AttributeTok{ }\StringTok{"80\%"}
\FunctionTok{output}\KeywordTok{:}
\AttributeTok{  }\FunctionTok{pdf\_document}\KeywordTok{:}\AttributeTok{ }
\AttributeTok{    }\FunctionTok{pandoc\_args}\KeywordTok{:}\AttributeTok{ }\KeywordTok{[}\StringTok{"{-}{-}lua{-}filter=reduce\_header\_level.lua"}\KeywordTok{]}
\PreprocessorTok{{-}{-}{-}}
\end{Highlighting}
\end{Shaded}

Then, make sure use that parameter to set the output width when you include figures in \textbf{paper\_to\_include/my\_paper.Rmd}:

\begin{Shaded}
\begin{Highlighting}[]
\InformationTok{\textasciigrave{}\textasciigrave{}\textasciigrave{}\{r, out.width=params$out\_width, fig.cap="A very funny cat"\}}
\InformationTok{include\_graphics(str\_c(params$other\_path, "figures/cat.jpg"))}
\InformationTok{\textasciigrave{}\textasciigrave{}\textasciigrave{}}
\end{Highlighting}
\end{Shaded}

Finally, create the parameter \texttt{out\_width} in your thesis' \textbf{index.Rmd} file:

\begin{Shaded}
\begin{Highlighting}[]
\FunctionTok{params}\KeywordTok{:}
\AttributeTok{  }\FunctionTok{other\_path}\KeywordTok{:}\AttributeTok{ }\StringTok{"../paper\_to\_include/"}
\AttributeTok{  }\FunctionTok{out\_width}\KeywordTok{:}\AttributeTok{ }\StringTok{"80\%"}
\end{Highlighting}
\end{Shaded}

Now, the output width of your figure will be 80\% when knitting your paper on its own, and 100\% when knitting it as child document of your thesis.

\hypertarget{customizing-referencing}{%
\section{Customizing referencing}\label{customizing-referencing}}

\hypertarget{using-a-.csl-file-with-pandoc-instead-of-biblatex}{%
\subsection{Using a .csl file with pandoc instead of biblatex}\label{using-a-.csl-file-with-pandoc-instead-of-biblatex}}

The \texttt{oxforddown} package uses biblatex in LaTeX for referencing.
It is also possible to use pandoc for referencing by providing a .csl file in the YAML header of \textbf{index.Rmd} (likely requiring commenting out the biblatex code in \textbf{templates/template.tex}).
This may be helpful for those who have a .csl file describing the referencing format for a particular journal.
However, note that this approach does not support chapter bibliographies (see Section \ref{biblatex-custom}).

\begin{Shaded}
\begin{Highlighting}[]
\FunctionTok{csl}\KeywordTok{:}\AttributeTok{ ecology.csl}
\end{Highlighting}
\end{Shaded}

\hypertarget{biblatex-custom}{%
\subsection{Customizing biblatex and adding chapter bibliographies}\label{biblatex-custom}}

This section provides one example of customizing biblatex. Much of this code was combined from searches on Stack Exchange and other sources (e.g.~\href{https://tex.stackexchange.com/questions/10682/suppress-in-biblatex}{here}).

In \textbf{templates/template.tex}, one can replace the existing biblatex calls with the following to achieve referencing that looks like this:

(Charmantier and Gienapp 2014)

Charmantier, A. and P. Gienapp (2014). Climate change and timing of avian breeding and migration: evolutionary versus plastic changes. Evolutionary Applications 7(1):15--28. doi: 10.1111/eva.12126.

\begin{Shaded}
\begin{Highlighting}[]
\BuiltInTok{\textbackslash{}usepackage}\NormalTok{[backend=biber,}
\NormalTok{    bibencoding=utf8,}
\NormalTok{    refsection=chapter, }\CommentTok{\% referencing by chapter}
\NormalTok{    style=authoryear, }
\NormalTok{    firstinits=true,}
\NormalTok{    isbn=false,}
\NormalTok{    doi=true,}
\NormalTok{    url=false,}
\NormalTok{    eprint=false,}
\NormalTok{    related=false,}
\NormalTok{    dashed=false,}
\NormalTok{    clearlang=true,}
\NormalTok{    maxcitenames=2,}
\NormalTok{    mincitenames=1,}
\NormalTok{    maxbibnames=10,}
\NormalTok{    abbreviate=false,}
\NormalTok{    minbibnames=3,}
\NormalTok{    uniquelist=minyear,}
\NormalTok{    sortcites=true,}
\NormalTok{    date=year}
\NormalTok{]\{}\ExtensionTok{biblatex}\NormalTok{\}}
\FunctionTok{\textbackslash{}AtEveryBibitem}\NormalTok{\{}\CommentTok{\%}
  \FunctionTok{\textbackslash{}clearlist}\NormalTok{\{language\}}\CommentTok{\%}
  \FunctionTok{\textbackslash{}clearfield}\NormalTok{\{note\}}
\NormalTok{\}}

\FunctionTok{\textbackslash{}DeclareFieldFormat}\NormalTok{\{titlecase\}\{}\FunctionTok{\textbackslash{}MakeTitleCase}\NormalTok{\{\#1\}\}}

\FunctionTok{\textbackslash{}newrobustcmd}\NormalTok{\{}\FunctionTok{\textbackslash{}MakeTitleCase}\NormalTok{\}[1]\{}\CommentTok{\%}
  \FunctionTok{\textbackslash{}ifthenelse}\NormalTok{\{}\FunctionTok{\textbackslash{}ifcurrentfield}\NormalTok{\{booktitle\}}\FunctionTok{\textbackslash{}OR\textbackslash{}ifcurrentfield}\NormalTok{\{booksubtitle\}}\CommentTok{\%}
    \FunctionTok{\textbackslash{}OR\textbackslash{}ifcurrentfield}\NormalTok{\{maintitle\}}\FunctionTok{\textbackslash{}OR\textbackslash{}ifcurrentfield}\NormalTok{\{mainsubtitle\}}\CommentTok{\%}
    \FunctionTok{\textbackslash{}OR\textbackslash{}ifcurrentfield}\NormalTok{\{journaltitle\}}\FunctionTok{\textbackslash{}OR\textbackslash{}ifcurrentfield}\NormalTok{\{journalsubtitle\}}\CommentTok{\%}
    \FunctionTok{\textbackslash{}OR\textbackslash{}ifcurrentfield}\NormalTok{\{issuetitle\}}\FunctionTok{\textbackslash{}OR\textbackslash{}ifcurrentfield}\NormalTok{\{issuesubtitle\}}\CommentTok{\%}
    \FunctionTok{\textbackslash{}OR\textbackslash{}ifentrytype}\NormalTok{\{book\}}\FunctionTok{\textbackslash{}OR\textbackslash{}ifentrytype}\NormalTok{\{mvbook\}}\FunctionTok{\textbackslash{}OR\textbackslash{}ifentrytype}\NormalTok{\{bookinbook\}}\CommentTok{\%}
    \FunctionTok{\textbackslash{}OR\textbackslash{}ifentrytype}\NormalTok{\{booklet\}}\FunctionTok{\textbackslash{}OR\textbackslash{}ifentrytype}\NormalTok{\{suppbook\}}\CommentTok{\%}
    \FunctionTok{\textbackslash{}OR\textbackslash{}ifentrytype}\NormalTok{\{collection\}}\FunctionTok{\textbackslash{}OR\textbackslash{}ifentrytype}\NormalTok{\{mvcollection\}}\CommentTok{\%}
    \FunctionTok{\textbackslash{}OR\textbackslash{}ifentrytype}\NormalTok{\{suppcollection\}}\FunctionTok{\textbackslash{}OR\textbackslash{}ifentrytype}\NormalTok{\{manual\}}\CommentTok{\%}
    \FunctionTok{\textbackslash{}OR\textbackslash{}ifentrytype}\NormalTok{\{periodical\}}\FunctionTok{\textbackslash{}OR\textbackslash{}ifentrytype}\NormalTok{\{suppperiodical\}}\CommentTok{\%}
    \FunctionTok{\textbackslash{}OR\textbackslash{}ifentrytype}\NormalTok{\{proceedings\}}\FunctionTok{\textbackslash{}OR\textbackslash{}ifentrytype}\NormalTok{\{mvproceedings\}}\CommentTok{\%}
    \FunctionTok{\textbackslash{}OR\textbackslash{}ifentrytype}\NormalTok{\{reference\}}\FunctionTok{\textbackslash{}OR\textbackslash{}ifentrytype}\NormalTok{\{mvreference\}}\CommentTok{\%}
    \FunctionTok{\textbackslash{}OR\textbackslash{}ifentrytype}\NormalTok{\{report\}}\FunctionTok{\textbackslash{}OR\textbackslash{}ifentrytype}\NormalTok{\{thesis\}\}}
\NormalTok{    \{\#1\}}
\NormalTok{    \{}\FunctionTok{\textbackslash{}MakeSentenceCase}\NormalTok{\{\#1\}\}\}}
    
\CommentTok{\% \textbackslash{}renewbibmacro\{in:\}\{\}}
\CommentTok{\% suppress "in" for articles}
\CommentTok{\% }
\FunctionTok{\textbackslash{}renewbibmacro}\NormalTok{\{in:\}\{}\CommentTok{\%}
  \FunctionTok{\textbackslash{}ifentrytype}\NormalTok{\{article\}\{\}\{}\FunctionTok{\textbackslash{}printtext}\NormalTok{\{}\FunctionTok{\textbackslash{}bibstring}\NormalTok{\{in\}}\FunctionTok{\textbackslash{}intitlepunct}\NormalTok{\}\}\}}
\CommentTok{\%{-}{-} no "quotes" around titles of chapters/article titles}
\FunctionTok{\textbackslash{}DeclareFieldFormat}\NormalTok{[article, inbook, incollection, inproceedings, misc, thesis, unpublished]}
\NormalTok{\{title\}\{\#1\}}
\CommentTok{\%{-}{-} no punctuation after volume}
\FunctionTok{\textbackslash{}DeclareFieldFormat}\NormalTok{[article]}
\NormalTok{\{volume\}\{\{\#1\}\}}
\CommentTok{\%{-}{-} puts number/issue between brackets}
\FunctionTok{\textbackslash{}DeclareFieldFormat}\NormalTok{[article, inbook, incollection, inproceedings, misc, thesis, unpublished]}
\NormalTok{\{number\}\{}\FunctionTok{\textbackslash{}mkbibparens}\NormalTok{\{\#1\}\} }
\CommentTok{\%{-}{-} and then for articles directly the pages w/o any "pages" or "pp." }
\FunctionTok{\textbackslash{}DeclareFieldFormat}\NormalTok{[article]}
\NormalTok{\{pages\}\{\#1\}}
\CommentTok{\%{-}{-} for some types replace "pages" by "p."}
\FunctionTok{\textbackslash{}DeclareFieldFormat}\NormalTok{[inproceedings, incollection, inbook]}
\NormalTok{\{pages\}\{p. \#1\}}
\CommentTok{\%{-}{-} format 16(4):224{-}{-}225 for articles}
\FunctionTok{\textbackslash{}renewbibmacro*}\NormalTok{\{volume+number+eid\}\{}
  \FunctionTok{\textbackslash{}printfield}\NormalTok{\{volume\}}\CommentTok{\%}
  \FunctionTok{\textbackslash{}printfield}\NormalTok{\{number\}}\CommentTok{\%}
  \FunctionTok{\textbackslash{}printunit}\NormalTok{\{}\FunctionTok{\textbackslash{}addcolon}\NormalTok{\}}
\NormalTok{\}}
\end{Highlighting}
\end{Shaded}

If you would like chapter bibliographies, in addition insert the following code at the end of each chapter, and comment out the entire REFERENCES section at the end of template.tex.

\begin{Shaded}
\begin{Highlighting}[]
\FunctionTok{\textbackslash{}printbibliography}\NormalTok{[segment=}\FunctionTok{\textbackslash{}therefsection}\NormalTok{,heading=subbibliography]}
\end{Highlighting}
\end{Shaded}

\hypertarget{customizing-the-page-headers-and-footers-pdf}{%
\section{Customizing the page headers and footers (PDF)}\label{customizing-the-page-headers-and-footers-pdf}}

This can now be done directly in \textbf{index.Rmd}'s YAML header.
If you are a LaTeX expert and need further customisation that what's currently provided, you can tweak the relevant sections of \textbf{templates/template.tex} - the relevant code is beneath the line that begins \texttt{\textbackslash{}usepackage\{fancyhdr\}}.

\hypertarget{diving-in-to-the-oxthesis-latex-template-pdf}{%
\section{Diving in to the OxThesis LaTeX template (PDF)}\label{diving-in-to-the-oxthesis-latex-template-pdf}}

For LaTeX minded people, you can read through \textbf{templates/template.tex} to see which additional customisation options are available as well as \textbf{templates/ociamthesis.cls} which supplies the base class.
For example, \textbf{template.tex} provides an option for master's degree submissions, which changes identifying information to candidate number and includes a word count.
At the time of writing, you must set this directly in \textbf{template.tex} rather than from the YAML header in \textbf{index.Rmd}.

\hypertarget{customising-to-a-different-university}{%
\section{Customising to a different university}\label{customising-to-a-different-university}}

\hypertarget{the-minimal-route}{%
\subsection{The minimal route}\label{the-minimal-route}}

If the front matter in the OxThesis LaTeX template is suitable to your university, customising \texttt{oxforddown} to your needs could be as simple as putting the name of your institution and the path to your university's logo in \textbf{index.Rmd}:

\begin{Shaded}
\begin{Highlighting}[]
\FunctionTok{university}\KeywordTok{:}\AttributeTok{ University of You}
\FunctionTok{university{-}logo}\KeywordTok{:}\AttributeTok{ figures/your{-}logo{-}here.pdf}
\end{Highlighting}
\end{Shaded}

\hypertarget{replacing-the-entire-title-page-with-your-required-content}{%
\subsection{Replacing the entire title page with your required content}\label{replacing-the-entire-title-page-with-your-required-content}}

If you have a \textbf{.tex} file with some required front matter from your university that you want to replace the OxThesis template's title page altogether, you can provide a filepath to this file in \textbf{index.Rmd}.
\texttt{oxforddown}'s sample content includes and example of this --- if you use the YAML below, your front matter will look like this:

\begin{Shaded}
\begin{Highlighting}[]
\FunctionTok{alternative{-}title{-}page}\KeywordTok{:}\AttributeTok{ front{-}and{-}back{-}matter/alt{-}title{-}page{-}example.tex}
\end{Highlighting}
\end{Shaded}

\noindent
\fbox{\includegraphics[width=0.32\linewidth]{figures/sample-content/alt_frontmatter_example/split/_000001.pdf}} \fbox{\includegraphics[width=0.32\linewidth]{figures/sample-content/alt_frontmatter_example/split/_000002.pdf}} \fbox{\includegraphics[width=0.32\linewidth]{figures/sample-content/alt_frontmatter_example/split/_000003.pdf}} \fbox{\includegraphics[width=0.32\linewidth]{figures/sample-content/alt_frontmatter_example/split/_000004.pdf}} \fbox{\includegraphics[width=0.32\linewidth]{figures/sample-content/alt_frontmatter_example/split/_000005.pdf}} \fbox{\includegraphics[width=0.32\linewidth]{figures/sample-content/alt_frontmatter_example/split/_000006.pdf}}

\hypertarget{troubleshooting}{%
\chapter{Troubleshooting}\label{troubleshooting}}

This chapter describes common errors you may run into, and how to fix them.

\hypertarget{error-failed-to-build-the-bibliography-via-biber}{%
\section{Error: Failed to build the bibliography via biber}\label{error-failed-to-build-the-bibliography-via-biber}}

This can happen if you've had a failed build, perhaps in relation to RStudio shutting down abruptly.

Try doing this:

\begin{enumerate}
\def\labelenumi{\arabic{enumi}.}
\tightlist
\item
  type \texttt{make\ clean-knits} in the terminal tab (or run \texttt{file.remove(list.files(pattern\ =\ "*.(log\textbar{}mtc\textbar{}maf\textbar{}aux\textbar{}bbl\textbar{}blg\textbar{}xml)"))} in the R console) to clean up files generated by LaTeX during a build
\item
  restart your computer
\end{enumerate}

If this does not solve the problem, try using the \href{https://www.overleaf.com/learn/latex/Bibliography_management_with_natbib}{natbib} LaTeX package instead of \href{https://www.overleaf.com/learn/latex/Articles/Getting_started_with_BibLaTeX}{biblatex} for handling references.
To do this, go to \textbf{index.Rmd} and

\begin{enumerate}
\def\labelenumi{\arabic{enumi}.}
\tightlist
\item
  set \texttt{use-biblatex:\ false} and \texttt{use-natbib:\ true}
\item
  set \texttt{citation\_package:\ natbib} under
\end{enumerate}

\begin{Shaded}
\begin{Highlighting}[]
\FunctionTok{output}\KeywordTok{:}
\AttributeTok{  bookdown:}\FunctionTok{:pdf\_book}\KeywordTok{:}
\AttributeTok{    }\FunctionTok{citation\_package}\KeywordTok{:}\AttributeTok{ natbib}
\end{Highlighting}
\end{Shaded}

\begin{savequote}
Alles Gescheite ist schon gedacht worden.\\
Man muss nur versuchen, es noch einmal zu denken.

All intelligent thoughts have already been thought;\\
what is necessary is only to try to think them again.
\qauthor{--- Johann Wolfgang von Goethe \autocite{von_goethe_wilhelm_1829}}\end{savequote}



\hypertarget{conclusion-2}{%
\chapter*{Conclusion}\label{conclusion-2}}
\addcontentsline{toc}{chapter}{Conclusion}

If we don't want Conclusion to have a chapter number next to it, we can add the \texttt{\{-\}} attribute.

\hypertarget{more-info}{%
\section*{More info}\label{more-info}}
\addcontentsline{toc}{section}{More info}

And here's some other random info:
the first paragraph after a chapter title or section head \emph{shouldn't be} indented, because indents are to tell the reader that you're starting a new paragraph.
Since that's obvious after a chapter or section title, proper typesetting doesn't add an indent there.

This paragraph, by contrast, \emph{will} be indented as it should because it is not the first one after the `More info' heading.
All hail LaTeX. (If you're reading the HTML version, you won't see any indentation - have a look at the PDF version to understand what in the earth this section is babbling on about).

\startappendices

\hypertarget{the-first-appendix}{%
\chapter{The First Appendix}\label{the-first-appendix}}

This first appendix includes an R chunk that was hidden in the document (using \texttt{echo\ =\ FALSE}) to help with readibility:

\textbf{In 02-rmd-basics-code.Rmd}

\textbf{And here's another one from the same chapter, i.e.~Chapter \ref{code}:}

\hypertarget{the-second-appendix-for-fun}{%
\chapter{The Second Appendix, for Fun}\label{the-second-appendix-for-fun}}


%%%%% REFERENCES

% JEM: Quote for the top of references (just like a chapter quote if you're using them).  Comment to skip.
% \begin{savequote}[8cm]
% The first kind of intellectual and artistic personality belongs to the hedgehogs, the second to the foxes \dots
%   \qauthor{--- Sir Isaiah Berlin \cite{berlin_hedgehog_2013}}
% \end{savequote}

\setlength{\baselineskip}{0pt} % JEM: Single-space References

{\renewcommand*\MakeUppercase[1]{#1}%
\printbibliography[heading=bibintoc,title={\bibtitle}]}


\end{document}
